%% MATTER_Tunnel.tex
%% 2024/10/7
%% by Dongwook Kim, Jisu Shin, Giram Park, Seoyoon Jung

\documentclass[conference]{IEEEtran}

\usepackage{needspace}
\usepackage{enumitem}

\hyphenation{op-tical net-works semi-conduc-tor}

\begin{document}

\title{MATTER Tunnel}

\author{
\IEEEauthorblockN{Dongwook Kim}
\IEEEauthorblockA{\textit{College of Engineering} \\
\textit{Hanyang University}\\
\textit{Dept.of Information Systems}\\
Seoul, Korea \\
dongwook1214@gmail.com}
\and
\IEEEauthorblockN{Jisu Shin}
\IEEEauthorblockA{\textit{College of Engineering} \\
\textit{Hanyang University}\\
\textit{Dept.of Information Systems}\\
Seoul, Korea \\
sjsz0811@hanyang.ac.kr}
\and
\IEEEauthorblockN{Giram Park}
\IEEEauthorblockA{\textit{College of Engineering} \\
\textit{Hanyang University}\\
\textit{Dept.of Information Systems}\\
Seoul, Korea \\
kirammida@hanyang.ac.kr}
\and
\IEEEauthorblockN{Seoyoon Jung}
\IEEEauthorblockA{\textit{College of Engineering} \\
\textit{Hanyang University}\\
\textit{Dept.of Information Systems}\\
Seoul, Korea \\
yoooonnn@naver.com}
}
\maketitle

\begin{abstract}
Our team introduces "MATTER Tunnel," which enables the MATTER protocol to operate on a blockchain basis. MATTER is a protocol that provides interoperability between IoT devices from various manufacturers, allowing control of multiple brands of IoT devices from a single application. However, due to current network constraints such as NAT and firewalls, a dedicated MATTER hub is required when using MATTER devices.
The mandatory use of MATTER hubs degrades user experience and causes many issues, including user dependence on specific platforms. MATTER Tunnel resolves these current limitations of MATTER by utilizing blockchain technology. This technology operates as if creating a virtual private network between applications and IoT devices on a blockchain.
MATTER Tunnel will enhance user experience, enable complete end-to-end encryption (E2EE), and lower the entry barriers for small development teams into the MATTER ecosystem. With the introduction of MATTER Tunnel, security of IoT devices will be strengthened, and users will be able to enjoy a more unrestricted IoT experience.
\end{abstract}

\begin{IEEEkeywords}
MATTER Tunnel, MATTER, MATTER hub, Blockchain, user experience, E2EE
\end{IEEEkeywords}

\begin{table}[h]
\caption{Role Assignments}
\def\arraystretch{1.24} \small
    \begin{tabular}{|p{1.8cm}|p{1.4cm}|p{4.4cm}|}
        \hline
         Roles & Name & Task description and etc. \\ \hline
         
         Development \par manager \par Blockchain \par Developer & Dongwook \par Kim & The role involves designing and implementing solutions utilizing blockchain technology. This position sets the technical direction for projects and applies the latest blockchain trends and technologies. Responsibilities include developing smart contracts, optimizing blockchain protocols, managing team members' work, and developing their skills.\\ \hline

         User, \par Customer, \par Front-end \par Developer & Jisu \par Shin & From user and customer perspective, considers what features would enhance the IoT experience and how to improve user interaction. From a development perspective, designs and implements user interfaces for IoT applications, focusing on intuitive and responsive front-end solutions that integrate with MATTER protocol and blockchain technology.\\ \hline
        
    \end{tabular}
\end{table}

\needspace{10cm}

\begin{table}
\def\arraystretch{1.24} \small
    \begin{tabular}{|p{1.8cm}|p{1.4cm}|p{4.4cm}|}
        \hline
        User, \par Customer, \par Embedded \par Developer & Giram \par Park & From a user standpoint, evaluates how IoT devices can better serve everyday needs. As a Embedded developer, works on firmware and software for IoT devices compatible with the MATTER protocol and proposed blockchain solution. Focuses on implementing user-centric features that enhance device functionality, improve reliability, and simplify setup processes.\\ \hline
        
        User, \par Customer, \par Front-end \par Developer & Seoyoon \par Jung & From user and customer perspective, considers what features would enhance the IoT experience and how to improve user interaction. From a development perspective, designs and implements user interfaces for IoT applications, focusing on intuitive and responsive front-end solutions that integrate with MATTER protocol and blockchain technology. Aims to create user-friendly interfaces that simplify device control and management.\\ \hline
    \end{tabular}
\end{table}

\section{Introduction}

\subsection{Motivation}
The rapid advancement of Internet of Things (IoT) technology has led to significant growth in the smart home market. However, compatibility issues among IoT devices from various manufacturers have been a persistent challenge. To address this problem, the MATTER protocol was developed, offering an approach that enables control of IoT devices from multiple brands through a single application.

Despite the introduction of the MATTER protocol, the current implementation still harbors several crucial issues. These problems prevent the full realization of MATTER's original goals: true interoperability, security, and protection of user privacy. Therefore, we believe a new approach is necessary to overcome these limitations and maximize the potential of the MATTER protocol.

\subsection{Problem Statement}
The current implementation of the MATTER protocol presents the following key issues:

Difficulty in P2P Communication Due to Network Constraints: Network limitations such as NAT and firewalls make direct P2P communication between IoT devices and applications challenging in typical households. This necessitates the use of dedicated MATTER hubs.

Platform Dependency: The need for platform-specific solutions, such as Apple's "HomeKit" and "HomePod" or Google's "Google Home" and "Nest Hub," results in consumers being locked into particular platforms.

Limited End-to-End Encryption (E2EE): The advantage of E2EE in the MATTER protocol is confined to operation within private networks, failing to ensure complete security throughout the entire communication process.

Threats to User Privacy: The communication structure that routes through cloud services potentially threatens user privacy.

Centralized Authentication System: MATTER devices must be certified by a centralized root certificate authority (CA), which makes it difficult for small development teams to participate and limits consumer choices.

These issues hinder the original purposes of the MATTER protocol: interoperability, security, and openness. Therefore, we have determined that a new approach is necessary to overcome these limitations and fully realize the potential of the MATTER protocol. We propose a solution utilizing blockchain technology to address these challenges.

\subsection{Research on related software}

\begin{enumerate}[itemsep=2ex, parsep=1ex]
    \item MATTER
    
    We use a variety of IoT devices, and several manufacturers develop and sell IoT devices with different names and appearances. It is Matter that enables these various smart home devices to be connected and managed at once.
    
    Matter is an IP-based smart home interworking standard that is compatible with all devices, designed to overcome the manufacturer-dependent limitations of smart home devices. It was launched in 2019 by four IoT giants Apple, Amazon, Google, Samsung SmartThings, and the global association CSA, formerly the Zigbee Alliance, and renamed Matter in 2021.
    
    Matter has the following technical features:
    
    Unlike existing ZigBee and Z-Wave, Matter operates based on IP protocols. The fact that Matter is based on IP, a network layer protocol, means that as long as it supports only IP, the communication protocols under it will not matter, and eventually, all processing will be done at the application layer. In other words, the transmission method varies depending on what application is, but as long as you use IP, that method is not important. Therefore, devices with the Matter logo can work together regardless of different brands or supported transmission protocols. In addition, the reason why it is important to use IP is that IP protocols are already proven in the market in terms of interoperability and security.
    
    Matter is interoperable between devices. Matter allows each device to interact using the same protocol, even if it is from a different manufacturer. For example, Samsung Electronics' products have been linked to SmartThings, and LG Electronics only to the ThinQ platform, but now Samsung Electronics' products can be connected to ThinQ. Matter is a very desirable standard from a user's point of view because most homes use a mixture of products from multiple brands.
    
    Matter supports both Wi-Fi and Thread, a low-power mesh network protocol, and supports various network protocols, such as using BLE in the device setting process.
    
    In addition, Matter has the characteristics of Multi-Admin, which uses the same device in conjunction with multiple platforms, AES authentication prescribed by NIST in the United States regarding data encryption, and PKI and certificates for device authentication.
    
    An open ecosystem is being created with the introduction of Matter with these characteristics, and the trend of automation and intelligence of residential environments is spreading through integration with Generative AI technology. Korea is also promoting active efforts to build and expand a smart home ecosystem by preparing support plans in line with global trends. The Korean government is expanding policy support by promoting 'AI@Home', a project centered on Matter and Generative AI, to support the creation of a smart home ecosystem.
    
    However, privacy protection, application of smart home technology of existing houses, and high installation costs are challenges that limit the growth of the market, so it is necessary to proactively prepare countermeasures.
    
    \item Network Constraints in P2P Communication
    
    In a P2P (Peer-to-Peer) structure, there is minimal reliance on always-on infrastructure servers. Instead, the application allows pairs of intermittently connected hosts, called peers, to communicate directly with each other. Peers are desktops and laptops controlled by users rather than owned by service providers, and most peers are located in homes, universities, and offices. Since communication occurs directly between peers without passing through a specific server, this structure is referred to as Peer-to-Peer. 

    Network constraints in P2P communication negatively impact user experience by assigning additional complex tasks to users.
    
    NAT (Network Address Translation) and Firewalls: 
    When many peers are behind NAT, NAT converts the private network's IP addresses into public IP addresses, allowing communication with external networks. However, NAT can block incoming connections, making it difficult for peers behind NAT to be accessed directly from the outside.
    Firewalls also block incoming connections from external networks, hindering communication between peers.
    To address these constraints, NAT traversal technologies like STUN, TURN, and ICE are required, or users may need to manually configure complicated network settings, such as port forwarding, to facilitate incoming connections. These manual configurations can bypass NAT and firewall restrictions, enabling direct access to specific services within the internal network from external sources.
    
    \item Matter hub
    
    Matter Hub is a central component of the Matter ecosystem, designed to facilitate seamless communication and interoperability between smart home devices from various manufacturers. Matter aims to unify different smart home technologies, allowing devices to work together regardless of brand.
    
    Smart Home Hubs serve as central controllers for smart home devices, enabling communication between Matter-compatible devices from different manufacturers. Samsung SmartThings and Amazon Echo are representative examples.
    
    MATTER Hubs connect Matter devices to the internet and other networks, allowing for remote access and control. Notable examples include Google Nest Hub, which integrates with Google services, and Apple HomePod, which utilizes Siri for voice commands.
    
    While Matter Hubs play a crucial role in enhancing interoperability within the smart home ecosystem, it's important to note that using Matter devices typically requires a home hub. Each application may dictate the specific Matter hub that must be used, which can strictly lock users into particular platforms. This limitation highlights the need for greater flexibility and broader compatibility in the Matter ecosystem to ensure a truly open and user-friendly IoT environment.

    \item Blockchain
    
    To effectively integrate blockchain technology into the IoT industry, it is crucial to consider blockchains with high transaction processing speeds (TPS) and enterprise-friendly features. 
    
    Several blockchain platforms stand out for their high TPS capabilities, including:
    
	Solana: Capable of processing up to 65,000 transactions per second
 
	XRP: Achieves around 1,500 TPS
 
	Hyperledger Fabric: Can process 2,000 to 20,000 TPS depending on the network configuration
 
    Among enterprise-friendly blockchain platforms, the following are noteworthy:
    
    	Hyperledger Fabric: Designed specifically for enterprise use
     
    	Quorum: An enterprise-focused version of Ethereum
     
    	Hyperledger  Besu: An Ethereum client designed for enterprise deployments, offering both public and private network capabilities
     
    After careful consideration of these options, Hyperledger Fabric is judged as the most suitable blockchain platform for the IoT industry. It offers a combination of high TPS and enterprise-grade features that are essential for large-scale IoT implementations.
    Furthermore, Hyperledger Fabric is compatibility with Monachain, a blockchain platform developed by LG CNS based on Hyperledger Fabric. This compatibility allows for seamless integration and immediate application in existing systems, potentially accelerating adoption and reducing implementation barriers.
    
\end{enumerate}

\section{Requirements}
The solution proposed in this study must meet the following key requirements:

User-Friendliness: It should support a smooth transition for existing MATTER protocol users without forcing the use of a Matter hub, and provide an interface that end-users can easily set up and manage.

Platform Independence: The solution should not be tied to any specific platform and should allow IoT devices from various manufacturers to interact seamlessly.

Interoperability: It must maintain compatibility with the existing MATTER protocol while simultaneously supporting MATTER Tunnel.

Development Convenience: It should support MATTER Tunnel with minimal code changes to existing MATTER devices and provide a simple API that developers can easily understand and implement.

Enhanced End-to-End Encryption (E2EE): Complete E2EE should be guaranteed even outside private networks, maintaining data confidentiality throughout the entire communication process.

User Privacy Protection: It should reduce reliance on centralized cloud services for communication and minimize the collection and use of user data while managing it transparently.

By meeting these requirements, the proposed solution is expected to overcome the limitations of the current Matter hub-based MATTER protocol and provide a better user experience, security, and privacy.

\end{document}


