%% Matter_Tunnel.tex
%% 2024/11/7
%% by Dongwook Kim, Jisu Shin, Giram Park, Seoyoon Jung

\documentclass[conference]{IEEEtran}

\usepackage{needspace}
\usepackage{enumitem}

\hyphenation{op-tical net-works semi-conduc-tor}

\begin{document}
	\title{Matter Tunnel}

	\author{ \IEEEauthorblockN{Dongwook Kim} \IEEEauthorblockA{\textit{College of Engineering} \\ \textit{Hanyang University}\\ \textit{Dept.of Information Systems}\\ Seoul, Korea \\ dongwook1214@gmail.com}
	\and \IEEEauthorblockN{Jisu Shin} \IEEEauthorblockA{\textit{College of Engineering} \\ \textit{Hanyang University}\\ \textit{Dept.of Information Systems}\\ Seoul, Korea \\ sjsz0811@hanyang.ac.kr}
	\and \IEEEauthorblockN{Giram Park} \IEEEauthorblockA{\textit{College of Engineering} \\ \textit{Hanyang University}\\ \textit{Dept.of Information Systems}\\ Seoul, Korea \\ kirammida@hanyang.ac.kr}
	\and \IEEEauthorblockN{Seoyoon Jung} \IEEEauthorblockA{\textit{College of Engineering} \\ \textit{Hanyang University}\\ \textit{Dept.of Information Systems}\\ Seoul, Korea \\ yoooonnn@naver.com}
	}
	\maketitle

	\begin{abstract}
		Our team introduces `Matter Tunnel', which enables the Matter protocol to
		operate on a blockchain basis. Matter is a protocol that provides interoperability
		between IoT devices from various manufacturers, allowing control of multiple
		brands of IoT devices from a single application. However, due to current
		network constraints such as NAT and firewalls, a dedicated Matter hub is required
		when using Matter devices. Matter Tunnel resolves the current limitations of
		Matter by utilizing blockchain technology, operating as if creating a virtual
		private network between applications and IoT devices. Primarily, it eliminates
		the mandatory use of Matter hubs, significantly enhancing user experience
		and flexibility. This innovation also extends the operational range of Matter
		devices, allowing them to be placed and controlled beyond the confines of a
		home. Users can easily manage devices in various environments such as home,
		workspaces, and vehicles through a single application. From an enterprise
		perspective, all device interactions and transactions are permanently recorded
		on the blockchain, providing businesses with reliable and immutable data for
		tracking device usage patterns and extracting valuable operational insights.
		Furthermore, Matter Tunnel ensures platform independence, freeing users from
		being locked into specific ecosystems. It strengthens security by enabling
		complete end-to-end encryption (E2EE) and enhances user privacy. Additionally,
		by lowering entry barriers, Matter Tunnel opens up opportunities for small development
		teams to participate in the Matter ecosystem. With these improvements, users
		can enjoy a more secure, versatile, and unrestricted IoT experience.
	\end{abstract}

	\begin{IEEEkeywords}
		Matter Tunnel, Matter, Blockchain, Matter hub, user experience, E2EE
	\end{IEEEkeywords}

	\begin{table}[h]
		\caption{Role Assignments}
		\def\arraystretch{1.24} \small
		\begin{tabular}{|p{1.8cm}|p{1.4cm}|p{4.4cm}|}
			\hline
			Roles                                                   & Name              & Task description and etc.                                                                                                                                                                                                                                                                                                                                  \\
			\hline
			Development \par manager \par Blockchain \par Developer & Dongwook \par Kim & The role involves designing and implementing solutions utilizing blockchain technology. This position sets the technical direction for projects and applies the latest blockchain trends and technologies. Responsibilities include developing smart contracts, optimizing blockchain protocols, managing team members' work, and developing their skills. \\
			\hline
		\end{tabular}
	\end{table}

	\needspace{10cm}

	\begin{table}
		\def\arraystretch{1.24} \small
		\begin{tabular}{|p{1.8cm}|p{1.4cm}|p{4.4cm}|}
			\hline
			User, \par Customer, \par Front-end \par Developer & Jisu \par Shin    & From user and customer perspective, considers what features would enhance the IoT experience and how to improve user interaction. From a development perspective, designs and implements user interfaces for IoT applications, focusing on intuitive and responsive front-end solutions that integrate with Matter protocol and blockchain technology.                                                                                      \\
			\hline
			User, \par Customer, \par Embedded \par Developer  & Giram \par Park   & From a user standpoint, evaluates how IoT devices can better serve everyday needs. As an Embedded developer, works on firmware and software for IoT devices compatible with the Matter protocol and proposed blockchain solution. Focuses on implementing user-centric features that enhance device functionality, improve reliability, and simplify setup processes.                                                                       \\
			\hline
			User, \par Customer, \par Front-end \par Developer & Seoyoon \par Jung & From user and customer perspective, considers what features would enhance the IoT experience and how to improve user interaction. From a development perspective, designs and implements user interfaces for IoT applications, focusing on intuitive and responsive front-end solutions that integrate with Matter protocol and blockchain technology. Aims to create user-friendly interfaces that simplify device control and management. \\
			\hline
		\end{tabular}
	\end{table}

	\section{Introduction}

	\subsection{Motivation}
	The rapid advancement of Internet of Things (IoT) technology has led to significant
	growth in the smart home market. However, compatibility issues among IoT devices
	from various manufacturers have been a persistent challenge. To address this
	problem, the Matter protocol was developed, offering an approach that enables
	control of IoT devices from multiple brands through a single application.

	Despite the introduction of the Matter protocol, the current implementation still
	harbors several crucial issues. These problems prevent the full realization of
	Matter's original goals: true interoperability, security, and protection of user
	privacy.

	Therefore, we believe a new approach is necessary to overcome these
	limitations and maximize the potential of the Matter protocol.

	\subsection{Problem Statement}
	The current implementation of the Matter protocol presents the following key
	issues.
	\begin{enumerate}[itemsep=2ex, parsep=1ex]
		\item Mandatory use of Matter hubs

			Network limitations such as NAT and firewalls make direct P2P
			communication between IoT devices and applications challenging in typical
			households. This necessitates the use of dedicated Matter hubs.

		\item Limited Operational Range

			As a consequence of the mandatory use of Matter hubs, the operational range
			of Matter is confined to the home. This limitation restricts the potential
			for broader IoT applications and remote device management beyond the
			immediate household environment.

		\item Platform Dependency

			The need for platform-specific solutions, such as Apple's `HomeKit' and `HomePod'
			or Google's `Google Home' and `Nest Hub', results in consumers being
			locked into particular platforms.

		\item Limited End-to-End Encryption (E2EE)

			The advantage of E2EE in the Matter protocol is confined to operation
			within private networks, failing to ensure complete security throughout the
			entire communication process.

		\item Threats to User Privacy

			The communication structure that routes through cloud services potentially
			threatens user privacy.

		\item Centralized Authentication System

			Matter devices must be certified by a centralized root certificate
			authority (CA), which makes it difficult for small development teams to participate
			and limits consumer choices.
	\end{enumerate}

	These issues hinder the original purposes of the Matter protocol: interoperability,
	security, and openness. Moreover, the existing centralized server-based IoT architecture
	presents significant challenges for businesses.

	\begin{enumerate}[itemsep=2ex, parsep=1ex]
		\item Data Reliability and Trust Issues

			The centralized server architecture creates vulnerability in data
			integrity, making it challenging for businesses to maintain reliable records.
			Without an immutable ledger system, businesses cannot ensure the authenticity
			and accuracy of device interaction data, which is crucial for operation
			analysis and service improvement.

		\item Data Analysis and Tracking Problems

			Traditional centralized IoT systems lack comprehensive data tracking
			mechanisms. This limitation makes it difficult for businesses to effectively
			analyze device usage patterns, maintain accurate maintenance histories, and
			extract valuable insights for business optimization. The fragmented nature
			of data across different platforms and environments further complicates
			the analysis process.
	\end{enumerate}

	Therefore, we have determined that a new approach is necessary to overcome these
	limitations and fully realize the potential of the Matter protocol. We propose
	a solution utilizing blockchain technology to address these challenges.

	\subsection{Research on related technical elements}

	\begin{enumerate}[itemsep=2ex, parsep=1ex]
		\item Matter

			We use a variety of IoT devices, and several manufacturers develop and sell
			IoT devices with different names and appearances. It is Matter that
			enables these various smart home devices to be connected and managed at
			once.

			Matter is an IP-based smart home interworking standard that is compatible
			with all devices, designed to overcome the manufacturer-dependent limitations
			of smart home devices. It was launched in 2019 by four IoT giants Apple, Amazon,
			Google, Samsung SmartThings, and the global association CSA, formerly the Zigbee
			Alliance, and renamed Matter in 2021.

			Matter has the following technical features:

			Unlike existing ZigBee and Z-Wave, Matter operates based on IP protocols. Since
			Matter is based on IP, which is a network layer protocol, the communication
			protocol below it does not matter, and eventually all processing is done
			at the application layer. In other words, the transmission method varies
			depending on what the application is, but as long as IP is used, the
			method is not important. Therefore, devices with the Matter logo can work
			together regardless of brands or supported transmission protocols. In addition,
			the reason why it is important to use IP is that IP protocols are already
			proven in the market in terms of interoperability and security.

			Matter is interoperable between devices. Matter allows each device to interact
			using the same protocol, even if it is from a different manufacturer. For example,
			Samsung Electronics' products have been linked to SmartThings, and LG
			Electronics only to the ThinQ platform, but now Samsung Electronics'
			products can be connected to ThinQ. Matter is a very desirable standard from
			a user's point of view because most homes use a mixture of products from multiple
			brands.

			Matter supports both Wi-Fi and Thread, a low-power mesh network protocol, and
			supports various network protocols, such as using BLE in the device
			setting process.

			In addition, Matter has the characteristics of Multi-Admin, which uses the
			same device in conjunction with multiple platforms, AES authentication prescribed
			by NIST in the United States regarding data encryption, and PKI and
			certificates for device authentication.

			An open ecosystem is being created with the introduction of Matter with
			these characteristics, and the trend of automation and intelligence of residential
			environments is spreading through integration with Generative AI technology.
			Korea is also promoting active efforts to build and expand a smart home
			ecosystem by preparing support plans in line with global trends. The Korean
			government is expanding policy support by promoting `AI@Home', a project
			centered on Matter and Generative AI, to support the creation of a smart home
			ecosystem.

			However, privacy protection, application of smart home technology of existing
			houses, and high installation costs are challenges that limit the growth
			of the market, so it is necessary to proactively prepare countermeasures.

		\item Network Constraints in P2P Communication

			In a Peer-to-Peer (P2P) structure, there is minimal reliance on always-on infrastructure
			servers. Instead, the application allows pairs of intermittently connected
			hosts, called peers, to communicate directly with each other. Peers are
			desktops and laptops controlled by users rather than owned by service
			providers, and most peers are located in homes, universities, and offices.
			Since communication occurs directly between peers without passing through
			a specific server, this structure is referred to as Peer-to-Peer.

			Network constraints in P2P communication, particularly due to NAT (Network
			Address Translation) and firewalls, negatively impact user experience by introducing
			additional complexity. NAT, which is used to map private IP addresses to public
			ones, can prevent peers behind it from being directly accessible from
			external networks, as it often blocks incoming connections. Firewalls
			similarly restrict inbound traffic, further complicating direct peer-to-peer
			communication. To address these issues, NAT traversal techniques such as STUN,
			TURN, and ICE are commonly employed. Alternatively, users may need to manually
			configure network settings, such as port forwarding, to bypass NAT and firewall
			restrictions, allowing for direct communication between peers. These manual
			configurations can be challenging for users, ultimately affecting the
			overall user experience.

		\item Matter Hub

			Matter Hub is a central component of the Matter ecosystem, designed to facilitate
			seamless communication and interoperability between smart home devices from
			various manufacturers. Matter aims to unify different smart home
			technologies, allowing devices to work together regardless of brand.

			Smart Home Hubs serve as central controllers for smart home devices,
			enabling communication between Matter-compatible devices from different
			manufacturers. Samsung SmartThings and Amazon Echo are representative examples.

			Matter Hubs connect Matter devices to the internet and other networks,
			allowing for remote access and control. Notable examples include Google Nest
			Hub, which integrates with Google services, and Apple HomePod, which
			utilizes Siri for voice commands.

			While Matter Hubs play a crucial role in enhancing interoperability within
			the smart home ecosystem, it's important to note that using Matter devices
			typically requires a home hub. Each application may dictate the specific
			Matter Hub that must be used, which can strictly lock users into particular
			platforms. This limitation highlights the need for greater flexibility and
			broader compatibility in the Matter ecosystem to ensure a truly open and
			user-friendly IoT environment.

		\item Blockchain

			To effectively integrate blockchain technology into the IoT industry, it is
			crucial to consider blockchains with high transaction processing speeds (TPS)
			and enterprise-friendly features.

			Several blockchain platforms stand out for their high TPS capabilities,
			including:

			Solana: Solana is an innovative platform designed for mainstream adoption.
			The core development team, including co-founder Anatoly Yakovenko, focused
			on scalability and efficiency based on their experience in building
			telecommunications networks. By implementing Proof of Stake (PoS) and Proof
			of History (PoH), they achieved a throughput of up to 65,000 TPS and
			realized very low transaction fees (\$0.00025). It's also highly energy
			efficient, a single Solana transaction uses 0.00051 kWh.

			XRP: XRP (Ripple) uses the RPCA consensus algorithm and provides
			approximately 1,500 TPS throughput. Specialized in international remittance,
			it has established partnerships with many banks and financial institutions,
			featuring fast transaction completion times of 3-5 seconds.

			Hyperledger Fabric: Hyperledger Fabric is an enterprise blockchain
			platform that provides 2,000-20,000 TPS throughput. Through various network
			configurations, organizations can adjust the throughput and degree of
			centralization according to their needs. For example, by modifying
			parameters such as the number of organizations, ordering service nodes,
			peer nodes, and channels, administrators can find the optimal balance
			between performance and decentralization.

			Among enterprise-friendly blockchain platforms, the following are noteworthy:

			Hyperledger Fabric: Hyperledger Fabric features a modular architecture and
			permissioned blockchain characteristics, supporting channel-based data partitioning.
			It has various enterprise use cases including supply chain management, asset
			tracking, identity management, and healthcare data management.

			Quorum: Quorum, developed by JP Morgan, is an enterprise blockchain based
			on Ethereum, featuring enhanced privacy features and high throughput. It
			supports private transactions, voting-based consensus mechanisms, role-based
			access control, and multi-signature contracts.

			Hyperledger Besu: Hyperledger Besu is a Java-based blockchain platform
			compatible with Ethereum. It supports both public and private networks and
			provides enterprise-grade governance. It is being utilized in various
			fields including financial services, supply chain management, digital
			asset management, and inter-enterprise collaboration platforms.

			After careful consideration of these options, Hyperledger Fabric is judged
			as the most suitable blockchain platform for the IoT industry. It offers a
			combination of high TPS and enterprise-grade features that are essential for
			large-scale IoT implementations. Furthermore, Hyperledger Fabric is compatible
			with Monachain, a blockchain platform developed by LG CNS based on
			Hyperledger Fabric. This compatibility allows for seamless integration and
			immediate application in existing systems, potentially accelerating
			adoption and reducing implementation barriers.

		\item Arduino

			Arduino is an open-source electronics platform based on easy-to-use
			hardware and software. In the context of Matter IoT, Arduino plays a
			significant role due to its flexibility, ease of use, and strong community
			support. When considering Arduino for Matter IoT applications, the
			following aspects are crucial:

			Processing power: Matter protocol requires sufficient computational resources
			to handle encryption and network communication.

			Connectivity options: Wi-Fi or Ethernet capability is essential for Matter,
			as it's IP-based.

			Memory capacity: Adequate RAM and flash memory to run Matter stack and
			application code.

			Power efficiency: For battery-operated IoT devices, low power consumption
			is critical.

			Compatibility with Matter SDK: The board should be capable of running the Matter
			SDK.

			Some Arduino boards suitable for Matter IoT projects include:

			Arduino Nano 33 IoT: This compact board features Wi-Fi connectivity and a
			powerful SAMD21 microcontroller, making it suitable for small Matter devices.

			Arduino MKR WiFi 1010: With its low power consumption and robust Wi-Fi
			capabilities, it's excellent for battery-operated Matter devices.

			Arduino Portenta H7: This high-performance board with dual-core processor
			and multiple connectivity options is ideal for more complex Matter
			applications.

			ESP32-S3: While not an official Arduino board, the ESP32-S3 is widely used
			in the Arduino ecosystem and offers powerful processing capabilities, Wi-Fi
			and Bluetooth connectivity, and ample memory.

			These boards offer various combinations of processing power, connectivity,
			and memory, allowing developers to choose the most suitable option for their
			specific Matter IoT application.

		\item Web Assembly

			Web Assembly (Wasm) is a binary instruction format designed for efficient
			execution in web browsers. It serves as a portable target for compilation of
			high-level languages like C, C++, and Rust, enabling deployment on the web
			for client and server applications.

			Key features and benefits of Web Assembly in the context of IoT and Matter
			include:

			Language Versatility: Web Assembly allows developers to use languages like
			C, C++, or Rust in web environments. This is particularly beneficial for
			IoT applications, as these languages are commonly used in embedded systems
			development.

			Performance: Wasm provides near-native performance, making it suitable for
			computationally intensive tasks often required in IoT applications.

			Code Reusability: It enables the use of the same codebase across different
			platforms - from embedded devices to web interfaces. This is especially valuable
			for functions like encryption and decryption, where consistent
			implementation across platforms is crucial.

			Frontend Capabilities: Web Assembly empowers frontend applications to perform
			complex operations typically associated with backend or embedded environments.
			This can include data processing, encryption, and other intensive
			computations directly in the browser.

			In the context of Matter Tunnel, Web Assembly can play a significant role in
			creating consistent, high-performance interfaces for controlling and
			managing IoT devices across different platforms. It allows developers to implement
			complex Matter protocol operations in web applications, maintaining
			consistency with the implementations on the devices themselves. This consistency
			across platforms is particularly valuable for ensuring that security measures,
			device interactions, and data handling are uniform across the entire Matter
			ecosystem.

		\item gRPC

			gRPC is a modern open source high performance Remote Procedure Call (RPC)
			framework that can run in any environment. It can efficiently connect services
			in and across data centers with pluggable support for load balancing,
			tracing, health checking and authentication.

			High Performance and Efficiency: gRPC is designed based on HTTP/2, allowing
			it to support features like multiplexing, server push, and streaming.
			These characteristics enable efficient management of connections between
			multiple clients and servers, minimizing latency and optimizing bandwidth.
			This ensures that high performance is maintained even in large-scale
			distributed systems.

			Protocol Buffers: gRPC uses Protocol Buffers for service definition.
			Protocol Buffers is a powerful binary serialization tool developed by
			Google, allowing the definition of data structures and their conversion into
			various programming languages. This increases the efficiency of data
			transmission and maintains consistency in APIs.

			Easy Installation and Scalability: gRPC offers simple installation,
			allowing developers to set up runtime and development environments with
			just a single command. It also provides scalability that can handle millions
			of RPCs per second, making it suitable for large-scale applications. This allows
			developers to quickly build and operate services without complex infrastructure
			setups.

			Cross-Language and Platform Support: gRPC works across multiple
			programming languages and platforms. It can automatically generate idiomatic
			client and server stubs for various languages such as Java, C++, Python,
			Go, and Ruby. This facilitates collaboration between teams using different
			tech stacks and enhances code reusability.

			Bi-Directional Streaming: gRPC supports bi-directional streaming between clients
			and servers. This allows clients to send data to the server while
			simultaneously receiving streamed data from the server. This feature is particularly
			useful in applications that require real-time data transmission.

			Integrated Authentication and Security: gRPC enhances security by
			integrating full pluggable authentication features at the HTTP/2 transport
			layer. This simplifies the implementation of user authentication and authorization,
			ensuring safe transmission of sensitive data.

			Due to these features, gRPC is widely used in various fields, including
			microservices architecture, IoT applications, and mobile backend services.
			By using gRPC, developers can build efficient and scalable systems, making
			it easier to manage communication between services.
	\end{enumerate}

	\subsection{Research on related study}

	\begin{enumerate}[itemsep=2ex, parsep=1ex]
		\item Benefits of Blockchain for Data Mining

			The integration of blockchain technology and data mining techniques for anomaly detection provides efficient methods through their combined application. Data stored on the blockchain can be treated as big data, and data mining techniques enable the extraction of hidden patterns and insights.

            This paper explores analytical approaches to blockchain data and practical applications, demonstrating how these technologies enhance anomaly detection and fraud prevention. Blockchain's transparent recording system facilitates data tracking and monitoring. It serves as an effective tool for corporate data analysis, with a key advantage being rapid detection of operational changes or fraudulent activities.

			Key applications from the literature include:

			\begin{enumerate}[itemsep=2ex, parsep=1ex]
				\item Analysis of Bitcoin Transaction Networks

					Zola et al. (2019) analyzed changes in Bitcoin transaction patterns by utilizing the time-series data of the blockchain. They used data from WalletExplorer and the Bitcoin mainnet over the past three years, calculating F1 scores through k-fold cross-testing. By analyzing the transaction data linked in chronological order on the blockchain, it is possible to detect cybersecurity threats and identify changes in behavioral patterns.

				\item Blockchain Data Collection and Analysis

					Brinckman et al. (2019) presented techniques for crawling, collecting, and analyzing blockchain data. They demonstrated a method for clustering transactions and extracting account characteristics to identify fraudulent accounts, which serves as a good example of understanding the data structure of the blockchain and effectively analyzing it.

				\item Time-Series Transaction Data Analysis

					Zhao et al. (2021) analyzed the entire dataset of the Ethereum blockchain from a temporal perspective. They utilized the ethereum blockchain dataset from the Bigquery Public Data Repository to examine changes in transaction patterns over time, comparing the accuracy of Random Forest and Logistic Regression, and visualizing the temporal evaluation of the collected data.
			\end{enumerate}

			These application examples demonstrate that effective analysis is possible by leveraging the connected data structure and temporal characteristics of the blockchain. In particular, the data structure of the blockchain can be effectively utilized to identify patterns in sensitive transaction data and detect anomalous behaviors, which can be considered a significant advantage of blockchain-based data analysis.

		\item Benefits of Blockchain for Data Integrity and Accessibility

			Blockchain technology has emerged as an innovative solution in healthcare data management, addressing important challenges in data security, integrity, and interoperability. Here are three representative implementations that demonstrate the applications of blockchain in healthcare:

			\begin{enumerate}[itemsep=2ex, parsep=1ex]
				\item MedRec

					MedRec is a blockchain project in healthcare that enables comprehensive management of medical data, including data provision by medical institutions, patient licensing, and data utilization by research institutes. MedRec 2.0 is implemented using Go-ethereum and Solidity languages, utilizing smart contracts on the Ethereum blockchain for intermediary-free data exchange. Like other blockchains, MedRec ensures security through its blockchain nature. Due to decentralization, data is maintained across all network nodes and stored in nodes of patients and their service providers. The blockchain's consensus mechanism prevents security issues from single-point vulnerabilities. Additionally, if one node attempts to modify a specific transaction in a block, that modified node becomes inconsistent with others and is excluded from consensus, maintaining record integrity.

				\item Estonian e-Health

					The Estonian e-Health Foundation and Guardtime have strengthened security and patient monitoring by implementing KSI blockchain technology. The healthcare system integrates medical service data through X-Road, a data exchange platform, enabling comprehensive medical service data analysis. This system provides personal healthcare data integrity verification, with healthcare providers transmitting data integrity to the KSI server for permanent blockchain and offline recording. Through the "e-prescription service," doctors and pharmacies can verify prescription integrity. Additionally, the "electronic health registration service" allows doctors to access medical images like X-rays using only the patient's ID code. These systems enable secure storage, efficient sharing, and integrity verification of medical data.

				\item Mediblock

					Mediblock is a blockchain-based integrated management platform for personal medical data that aims to integrate distributed patient medical data. Mediblock secures data by granting patients access to medical data and allowing only them to decrypt the entire data, while recording data hash values on the blockchain for integrity. It improves data reliability by restricting medical record creation to certified medical personnel, and ensures transparency by recording data access information and rights on the blockchain. Additionally, it enables safe data sharing through relay services and secondary backup storage, while facilitating secure data transactions through an encrypted data trading market within the platform.
			\end{enumerate}

			These implementations demonstrate how blockchain technology enhances healthcare data management through improved integrity, security, accessibility, and transparency. These advantages can be applied to blockchain-based Matter tunnels, potentially replacing Matter hubs to enhance the reliability and efficiency of the Matter protocol.

		\item Hyperledger Performance

			The performance of Hyperledger Fabric is often questioned, particularly regarding
			transaction processing speeds (TPS) in comparison to other blockchain platforms.
			The performance of a Fabric network is complex due to the involvement of
			multiple organizations with varying hardware and networking
			infrastructures, as well as factors such as the number of channels and chaincode
			implementation.

			Hyperledger Fabric 2.x has introduced performance improvements over
			version1.4, which is no longer in long-term support (LTS). Fabric 2.5 is the
			latest LTS version and includes a new peer gateway service. This service, along
			with the new gateway SDK, is expected to enhance the performance of
			applications.

			\begin{enumerate}[itemsep=2ex, parsep=1ex]
				\item Hardware and Topology

					In Hyperledger Fabric, the topology used consists of two peer organizations
					(PeerOrg0 and PeerOrg1), each with one peer node, along with a single ordering
					service organization (OrdererOrg0) that utilizes Raft consensus with one
					ordering service node. TLS is enabled on each node.

					All nodes utilized the same hardware configuration:

					- Intel(R) Xeon(R) Silver 4210 CPU @ 2.20GHz

					- 40 Cores made up of 2 CPUs. Each CPU has 10 physical cores supporting
					20 Threads in total

					- 64Gb Samsung 2933Mhx Memory

					- MegaRAID Tri-Mode SAS3516 (MR9461-16i) disk controller

					- Intel 730 and DC S35x0/3610/3700 Series SSD attached to disk
					controller

					- Ethernet Controller X710/X557-AT 10GBASE-T

					- Ubuntu 20.04

					All machines were connected to the same switch. Hyperledger Fabric was
					deployed natively across three physical machines, meaning that the native
					binaries were installed and executed without using container technologies
					such as Docker or Kubernetes.

				\item Hyperledger Fabric Application Configuration

					State Database: LevelDB was used as the state database.

					Gateway Service: The concurrency limit was set to 20,000.

					Application Channel: A single application channel was created with two
					peers and an orderer, configured to use V1\_4 capabilities for lifecycle
					deployment while other capabilities were set to V2\_0.

					Configuration: Only the application channel existed, no private data
					was used, and default policies applied. No range or JSON queries were
					conducted, and the network was TLS-enabled without mutual TLS.

					Chaincode: Go chaincode (fixed-asset-base) was deployed without the
					Contract API, and an endorsement policy of “1 Of Any” was specified.

				\item Load Generator

					Hyperledger Caliper 0.5.0 served as both the load generator and for generating
					report outputs. It was configured to work with Fabric 2.4, utilizing the
					peer Gateway Service to initiate and assess transactions, with all
					transactions originating from a single organization directed to its
					gateway peer. The load was defined based on the fixed-asset benchmark from
					Hyperledger Caliper-Benchmarks.

					Four bare-metal machines were employed to host remote Caliper workers and
					a single Caliper manager, which was responsible for generating the
					load on the Hyperledger Fabric network. To create sufficient workload on
					the Fabric network, multiple Caliper workers were necessary, corresponding
					to the number of clients currently connected to the network. The
					results section includes details on the number of Caliper workers
					utilized.

				\item Key Points

					The results presented here were generated using the latest builds of
					Hyperledger Fabric 2.5, employing the default node and channel configuration
					values. The block cutting parameters used were as follows:

					\textit{Block Cut Time: 2 seconds}

					\textit{Block Size: 500}

					\textit{Preferred Maximum Bytes: 2 MB}

					To avoid hitting concurrency limits while pushing enough workload
					through, the gateway concurrency limit was set to 20,000.

					Only a single channel was utilized, meaning the peer did not leverage its
					full resource potential.

					The chaincode was optimized for these tests, and real-world chaincode
					is expected to perform less efficiently.

					The Caliper workload generator was also optimized for transaction
					throughput, whereas real-world applications would involve client implementations
					that may introduce latency.

					Transactions were sent to a single gateway peer from the same organization;
					real-world scenarios would likely involve multiple organizations
					sending transactions concurrently, potentially leading to higher TPS results.

					Using the gateway service allowed for better performance as blocks were
					not received via the delivery service to confirm transaction completion,
					enhancing both client and network performance compared to the legacy
					node SDK.

				\item Result

					\begin{table}[h]
						\caption{Hyperledger Fabric Benchmark Result}
						\def\arraystretch{1.4} \small
						\begin{tabular}{|p{1.7cm}|p{1.5cm}|p{1.5cm}|p{1.7cm}|}
							\hline
							Name                                        & Max \par Latency & Average \par Latency & Throughput \\
							\hline
							create \par asset 100                       & 2.13             & 0.33                 & 2946.7     \\
							\hline
							create \par asset 1000                      & 3.21             & 1.52                 & 2938.9     \\
							\hline
							read write \par assets previously read 100  & 0.21             & 0.06                 & 2544.3     \\
							\hline
							read write \par assets previously read 1000 & 0.26             & 0.11                 & 1527.0     \\
							\hline
						\end{tabular}
					\end{table}

					\begin{enumerate}[itemsep=2ex, parsep=1ex]
						\item Blind Write of a Single Key 100 Byte Asset Size

							Caliper test configuration:

							- workers: 200

							- fixed-tps : 3000

							The TPS value represents the peak performance achieved during the tests.
							Attempts to exceed this throughput resulted in unexpected failures
							and a decrease in overall throughput.

						\item Blind Write of a Single key 1000 Byte Asset Size

							Caliper test configuration:

							- workers: 200

							- fixed-tps : 3000

							The throughput remains the same as that observed with the 100-byte
							blind write benchmark; however, latency increases.

						\item Read Write of a Single Key 100 Byte Asset Size

							Caliper test configuration:

							- workers: 200

							- fixed-tps, tps: 2550

							The above results were achieved under the expectation of no failures.
							The latency remained very low, indicating that the Fabric network was
							reaching its capacity during this test.

						\item Read Write of a Single Key 1000 Byte Asset Size

							Caliper test configuration:

							- workers: 200

							- fixed-tps, tps: 1530
					\end{enumerate}
			\end{enumerate}

			The above results were achieved under the expectation of no failures. The
			latency remained very low, indicating that the Fabric network was reaching
			its capacity during this test.
	\end{enumerate}

	\section{Requirements}

	\subsection{Core Requirements}
	The solution proposed in this study must meet the following key requirements

	\begin{enumerate}[itemsep=2ex, parsep=1ex]
		\item Eliminate the mandatory use of Matter Hubs

			It must eliminate the mandatory use of Matter Hubs, enabling direct and secure
			device-to-application communication without relying on intermediary hardware.
			This removal of hub dependency not only reduces system complexity and cost
			but also enhances system reliability by eliminating single points of
			failure.

		\item Extended Operational Range

			The solution should overcome the limitation of traditional Matter Hub-based
			systems, which confine device operation to a home network. It must enable
			secure and efficient management and control of IoT devices from remote
			locations, expanding the utility of Matter-compatible devices beyond the immediate
			household environment. This extended range should not compromise security
			or user privacy.

		\item Enhanced data tracking mechanisms

			The solution must incorporate enhanced data tracking mechanisms that
			provide comprehensive visibility into device operations, interactions.
			This tracking system should maintain tamper-proof records of all device
			activities, enabling businesses to analyze usage patterns, monitor
			performance metrics, and optimize their operations effectively. The
			implementation should support both real-time monitoring and historical
			data analysis while maintaining user privacy and data security.

		\item Improved system reliability

			To ensure system reliability and trust, all data interactions and transactions
			must be recorded on an immutable ledger, creating a verifiable and
			transparent history of device operations. This trustworthy data foundation
			is crucial for both operational intelligence and regulatory compliance,
			allowing businesses to make data-driven decisions with confidence. The system
			should provide mechanisms for data verification and validation while maintaining
			appropriate access controls and privacy measures.

		\item Decentralization

			The solution should embrace decentralization by removing dependencies on
			centralized certificate authorities (CAs) and platform-specific ecosystems.
			This decentralization should establish a more democratic and open IoT ecosystem
			where small developers and manufacturers can participate freely, fostering
			innovation and competition. Furthermore, the solution allow IoT devices
			from various manufacturers to interact seamlessly.

		\item Enhanced End-to-End Encryption (E2EE)

			E2EE should be guaranteed even outside private networks, maintaining data confidentiality
			throughout the entire communication process.

		\item User Privacy Protection

			It should reduce reliance on centralized cloud services for communication
			and minimize the collection and use of user data while managing it
			transparently.

		\item Interoperability

			The solution must maintain full compatibility with the existing Matter
			protocol while concurrently supporting Matter Tunnel. It should provide backward
			compatibility for legacy devices and forward compatibility for future Matter
			protocol updates, ensuring a cohesive ecosystem that evolves without
			disrupting existing setups.

		\item Real-time Performance

			The solution must support real-time communication and responsiveness, even
			when utilizing blockchain technology. It should ensure that blockchain
			integration does not introduce significant latency or delays in device
			interactions. The system should maintain quick response times for user commands
			and device state updates, while leveraging the benefits of blockchain for
			enhanced security and decentralization.

		\item Development Convenience

			It should support Matter Tunnel with minimal code changes to existing
			Matter devices and provide a simple API that developers can easily
			understand and implement.

			By meeting these requirements, the proposed solution is expected to overcome
			the limitations of the current Matter hub-based Matter protocol and provide
			a better user experience, security, and privacy.
	\end{enumerate}

	\subsection{Development Requirements}
	{\centering \textbf{Client} \par}
	\begin{enumerate}[itemsep=2ex, parsep=1ex]
		\item User Authentication

			The login system should prioritize security, simplicity, and compatibility
			with the Matter protocol. To achieve this, we propose implementing a login
			mechanism based on asymmetric cryptography, specifically using the secp256k1
			elliptic curve algorithm, which is also employed by Matter. Users can easily
			log in by entering their secp256k1 private key instead of using social login
			methods.

			\begin{enumerate}[itemsep=2ex, parsep=1ex]
				\item Sign Up

					Users can initiate the registration process by clicking the Sign Up
					button on the login page of the application. During the sign-up process,
					users will create their private key, which will serve as their unique identifier
					for logging in. Generated private key will be securely stored in local
					storage. If desired, users can retrieve their private key at any time
					from the application's account management section. This feature allows
					users to back up or transfer their private key to another device if needed.

				\item Login

					Users can log in by entering their private key. Upon successful login,
					a public key is derived from the private key, and users are directed to
					a page where they can register Matter devices. If the entered key
					doesn't match the required format, an error message starting ``Your key
					is incorrectly formatted" will be displayed.
			\end{enumerate}

		\item Add Device

			To register a Matter-compatible device, the user clicks the ‘+’ button to add
			a new device. User can scan the QR code or enter the setup code provided
			by the device to proceed with the registration. The device information will
			be stored in the local storage. Once the device is successfully registered,
			the user gets registration confirmation message and will be directed to a screen
			displaying the device status and features.

		\item Remove Device

			To remove an unnecessary device, the user select the device and click the ‘Remove’
			button. The user will be prompted to confirm choice before the device is
			removed from the system. Upon confirming the removal, the system will
			delete the device from the local storage and the user will receive a
			message confirming the successful deregistration. If an error occurs during
			the process, an error message will be provided.

		\item Device Control

			A user-friendly interface will be designed for controlling each device, focusing
			on intuitive navigation and clear functionality. The interface will
			display available control options. When the user issues a command to control
			a device, the command will be executed through communication with the
			blockchain and the device. Feedback will be provided to the user upon successsful
			command execution. User can set devices to operate automatically based on
			specific time or conditions. The application will allow users to configure
			and manage their automations.

		\item Data Display

			Matter devices transmit a variety of signals to the application through
			the Matter Tunnel. These signals are structured in various formats to accommodate
			diverse data types and use cases. The formats include, but are not limited
			to, JSON, binary data. Each format serves a specific purpose, allowing for
			flexible and efficient data transmission.

			The application is responsible for processing these diverse signals and presenting
			them to users in a manner that aligns with their respective data formats.
			Upon receiving the signals, the application will parse and interpret the data
			to ensure it is accurately represented. The transformed data will then be
			displayed in a user-friendly and intuitive interface that enhances the
			overall user experience.

			The application will be designed to automatically update the user
			interface in real-time, reflecting any changes in the device status or incoming
			data. This ensures that users have access to the most current information
			available, allowing for informed decision-making and timely responses.

			By supporting various signal formats and providing a clear, interactive
			display, the application aims to enhance the usability and effectiveness of
			Matter devices within the connected ecosystem.
	\end{enumerate}

	{\centering \textbf{Dashboard} \par}
	\begin{enumerate}[itemsep=2ex, parsep=1ex]
		\item Integrated Dashboard

			The Integrated Dashboard provides a unified view of the status of all
			devices and the network. Through visual representations of real-time device
			statuses, transaction logs, and key metrics, users can monitor and
			understand system performance at a glance, enabling quick situation assessment.

		\item Natural Language Query System

			The Natural Language Query System supports intuitive natural language
			input for querying blockchain data. The AI model converts user questions into
			precise data query commands, allowing users without programming knowledge
			to access and analyze blockchain data. This interface enhances accessibility,
			making data analysis easy for both technical and non-technical users.

		\item Data Visualization and Analysis Tools

			Data Visualization and Analysis Tools improve the platform’s usability by
			presenting blockchain data results in accessible visual formats. Charts,
			graphs, and other visuals provide clear insights, while advanced analytics
			functionalities let users analyze transaction logs, and real-time performance
			indicators (KPIs) to generate meaningful insights for decision-making.

		\item Insight and Report Service

			The Insight and Report Service features automatically generates daily, weekly,
			and monthly operational reports. These reports allow decision-makers to track
			performance trends, identify areas needing attention, and access actionable
			insights and recommendations, supporting effective planning and strategic
			problem-solving.
	\end{enumerate}

	\section{Development environment}

	\subsection{software development platform}

	\begin{enumerate}[itemsep=2ex, parsep=1ex]
		\item JavaScript

			JavaScript is a programming language used to make web pages dynamic, allowing
			for content changes in response to user interactions. It evolved from
			historically static web pages and is now utilized in both client-side and
			server-side development, with various libraries and frameworks expanding its
			functionality. JavaScript is interpreted by the browser, modifying the DOM
			in response to user events on the client side and generating dynamic content
			by interacting with databases on the server side. Additionally, when combined
			with HTML and CSS, it enhances the UI of web applications and allows for
			efficient task execution. As a client-side scripting language, JavaScript
			is one of the core technologies of the World Wide Web. Its features can improve
			the user experience of websites, from refreshing social media feeds to
			animations and interactive map displays. For example, when browsing the
			internet, if you encounter an image slideshow, a dropdown menu that
			appears upon clicking, or dynamic color changes of objects on a webpage,
			you are witnessing the effects of JavaScript in action. Its selection for this
			project is driven by the team's familiarity with it, which enhances
			efficiency and productivity.

		\item React

			React is a JavaScript library developed by Facebook, primarily used for building
			user interfaces (UI). It has a component-based structure, allowing developers
			to create reusable components for UI construction. React uses a Virtual
			DOM to efficiently handle updates and optimize performance, making it widely
			used in the front-end development of web applications. The decision to use
			React for this project was influenced by the fact that the team conducted
			a study on React together during their vacation, enhancing their familiarity
			and readiness to implement it effectively.

		\item Electron

			Electron is a framework for building desktop applications using JavaScript,
			HTML, and CSS. By embedding Chromium and Node.js into its binary, Electron
			enables developers to maintain a single JavaScript codebase and create
			cross-platform applications that work on Windows, macOS, and Linux. Popular
			desktop applications like Slack and Visual Studio Code are developed using
			Electron. Implementing the dashboard as a desktop application is
			advantageous for local use, particularly when want to use AI function in local,
			which makes Electron a suitable choice for this project. This decision reflects
			the need for a robust and efficient local application to meet the project’s
			requirements.

		\item Go

			Go (or Golang) is an open-source programming language developed by Google,
			designed to be fast and concise while supporting concurrency, making it
			ideal for network applications and server-side programming. Its straightforward
			syntax and ease of error handling contribute to its popularity in projects
			that require high performance and efficiency. Additionally, Go boasts a large
			ecosystem of partners, communities, and tools, making it easy to learn and
			fostering effective team collaboration. The decision to use Go for this project
			is influenced by the requirement that Hyperledger Fabric must be developed
			using Go, ensuring compatibility and optimal performance within the
			blockchain framework.

		\item gRPC

			gRPC is a high-performance, open-source Remote Procedure Call (RPC) framework
			initially developed by Google. It uses HTTP/2 for transport and Protocol
			Buffers as the interface description language, enabling efficient communication
			between distributed systems across different languages and platforms. gRPC
			excels in scenarios requiring high-throughput and low-latency
			communication, making it particularly suitable for microservices architectures.
			The framework's strong typing system, bidirectional streaming capabilities,
			and built-in support for authentication enhance the reliability and
			security of service-to-service communication. The decision to use gRPC for
			this project was influenced by its essential role in communicating with the
			Hyperledger Fabric gateway. Since the Electron-based frontend needs to
			interact with the Hyperledger Fabric network through its gateway, gRPC provides
			the necessary protocol and tools to establish this communication efficiently
			and securely.

		\item C++

			C++ is a high-performance, object-oriented programming language developed by
			Bjarne Stroustrup as an extension of the C language. It is widely used in
			various applications, including game engines, system software, IoT,
			embedded systems, and graphics processing. C++ provides a clear program
			structure and enables code reuse, which helps reduce development costs,
			while also offering portability for creating applications that can adapt
			to multiple platforms. Furthermore, C++ offers a high level of control
			over system resources and memory. This project leverages C++ for building Arduino
			and WebAssembly applications, capitalizing on its efficiency and
			versatility in these domains.

		\item Arduino

			Arduino is an open-source electronics platform based on easy-to-use
			hardware and software, primarily used in IoT and embedded systems projects.
			Arduino boards can read inputs—such as light from a sensor, a finger press
			on a button, or a Twitter message—and turn them into outputs, like activating
			a motor, lighting an LED, or publishing data online. Programmed using C/C++,
			Arduino enables rapid prototyping by connecting various sensors and actuators,
			and is also popular for educational and DIY projects. Users can control their
			boards by sending instructions to the microcontroller, allowing for
			flexible and dynamic applications. Arduino was chosen for this project because
			it offers a simple way to develop embedded systems, streamlining the
			development process and enhancing efficiency.

		\item Python

			Python is a high-level programming language with concise and easy-to-read
			syntax. It is used in various fields, including data science, artificial intelligence,
			web development, automation, and scripting. Thanks to its rich libraries and
			community support, Python enables rapid prototyping and highly productive
			development. In this project, Python is chosen specifically for AI
			development, leveraging its capabilities to create efficient and effective
			AI solutions.

		\item Visual Studio Code

			Visual Studio Code is a code editor redefined and optimized for building
			and debugging modern web and cloud applications. Developed by Microsoft,
			it is a free and open-source editor that supports a wide range of
			programming languages, including JavaScript, Python, and C++. With
			features like extensive extensibility through a marketplace of plugins, built-in
			debugging tools, and seamless integration with version control systems
			like Git, VS Code provides a user-friendly interface that enhances productivity.
			Additionally, it is available on multiple platforms, including Windows, macOS,
			and Linux, making it accessible to developers regardless of their operating
			system. Visual Studio Code was chosen for this project because it is the
			most commonly used code editor, offering familiarity and reliability for efficient
			development

		\item GoLand

			GoLand is an integrated development environment (IDE) specifically
			designed for the Go programming language, developed by JetBrains. It
			offers smart code assistance with advanced code completion, navigation,
			and refactoring tools, making it easier to write clean and efficient code.
			GoLand features a powerful integrated debugger for setting breakpoints and
			inspecting variables, as well as built-in support for unit testing and code
			coverage analysis. Additionally, it integrates seamlessly with version control
			systems like Git, allowing developers to manage their repositories
			directly within the IDE. With a customizable interface and cross-platform compatibility
			for Windows, macOS, and Linux, GoLand enhances the development experience,
			enabling developers to write, test, and deploy Go applications more effectively.
			GoLand was chosen for this project to facilitate the use of the Go
			programming language, providing the necessary tools and environment for optimal
			development.

		\item LaTeX

			LaTeX is a high-quality typesetting system. It includes features designed for
			the production of technical and scientific documentation. LaTeX is the de
			facto standard for the communication and publication of scientific
			documents.

		\item GitHub

			GitHub is a web-based platform that uses Git version control for managing
			and sharing code repositories. It enables developers to collaborate on projects
			by allowing them to track changes, manage branches, and resolve conflicts seamlessly.
			With features like pull requests, code reviews, and issue tracking, GitHub
			facilitates efficient team collaboration and project management.
			Additionally, it hosts a vast repository of open-source projects,
			providing developers with resources to learn from and contribute to. GitHub's
			integration with various CI/CD tools and support for GitHub Actions enhances
			its capabilities, making it an essential tool for modern software
			development.

		\item Notion

			Notion is an all-in-one workspace that combines note-taking, task management,
			databases, and collaboration tools, allowing teams to organize and share
			information effectively. With its flexible structure, users can create
			custom templates and pages tailored to their specific needs, promoting productivity
			and collaboration. Notion's rich formatting options, including tables,
			kanban boards, and calendars, enable users to visualize and manage their
			work dynamically. Additionally, its real-time collaboration features allow
			multiple users to edit and comment simultaneously, making it a powerful tool
			for project management and team communication.

		\item macOS

			macOS is a widely used operating system for software development, known for
			its user-friendly interface and exceptional versatility. It equips
			developers with essential tools and integrated development environments (IDEs)
			for creating a variety of applications, including web, desktop, mobile,
			and gaming software. The platform supports multiple programming languages and
			frameworks, offering the flexibility to adapt to specific project
			requirements. Its intuitive design simplifies the setup of development environments
			and project management. Additionally, an active macOS developer community fosters
			collaboration and knowledge sharing. With continuous updates, developers have
			access to the latest technologies and tools, enabling them to modernize
			their applications effectively. Overall, macOS is recognized as a crucial
			platform for software development, playing a significant role in turning innovative
			ideas into reality.
	\end{enumerate}

	\vspace{4cm}

	\subsection{Computer resources}

	\begin{table}[h]
		\caption{Computer Resources}
		\def\arraystretch{1.4} \small
		\begin{tabular}{|p{1.8cm}|p{2.7cm}|p{3.1cm}|}
			\hline
			Name              & Computer \par Resource                 & Version of OS, SW    \\
			\hline
			Dongwook \par Kim & Apple M3 Pro Chip \par 18GB RAM memory & macOS Sequoia 15.0.1 \\
			\hline
			Jisu Shin         & Apple M1 Chip \par 16GB RAM memory     & macOS Sequoia 15.0.1 \\
			\hline
			Giram park        & Apple M2 Chip \par 16GB RAM memory     & macOS Sequoia 15.0.1 \\
			\hline
			Seoyoon Jung      & Apple M2 Chip \par 8GB RAM memory      & macOS Sequoia 15.0.1 \\
			\hline
		\end{tabular}
	\end{table}

	\par

	\subsection{Cost Estimation}

	Although it is different from the actual application in the industry, we
	envision a test network operating two Hyperledger Fabric peers and one orderer
	on a single computer. For this configuration, we plan to use an AWS EC2 t2.medium
	instance (2vCPU, 4GB RAM). This t2.medium instance is deemed suitable for a
	test network of this scale, as it meets the minimum specifications required for
	running peers and orderer while providing a cost-effective option for development
	and testing purposes. According to the AWS pricing calculator, operating a t2.medium
	instance in the Seoul region with a long-term commitment would cost approximately
	\$18.47 per month. Adding the cost of a required 50GB EBS volume at approximately
	\$4.56, the total estimated monthly cost would be \$23.03. This represents the
	minimum cost for establishing a test environment, and an actual production
	environment would likely require higher-specification instances and additional
	infrastructure configuration. However, if we can utilize the company's
	existing underutilized server resources, we expect to significantly reduce these
	cloud cost.

	\subsection{Software in use}

	\begin{enumerate}[itemsep=2ex, parsep=1ex]
		\item ADEPT

			ADEPT, which stands for Autonomous Decentralized Peer-to-Peer Telemetry,
			is a blockchain platform designed for IoT devices that operates on a peer-to-peer
			basis. The idea behind ADEPT was introduced in 2015 as a result of collaboration
			between Samsung and IBM. It incorporates technologies like BitTorrent, Telehash,
			and Ethereum. ADEPT organizes IoT devices based on their capacities,
			allowing them to independently manage, analyze, and share their data. This
			platform is being implemented in wearable technology and household appliances.
			For instance, Samsung's smart washing machine employs ADEPT technology to automatically
			order essential supplies, such as detergent, when they are running low.

		\item IoT Chain

			IoT Chain operates on blockchain technology and incorporates various
			mechanisms like PBFT (Practical Byzantine Fault Tolerance), DAG (Directed
			Acyclic Graph), SPV (Simple Payment Verification), and CPS (Cyber Physical
			System). Its primary goal is to improve security within the IoT ecosystem
			while utilizing ICT (IoT Chain Token) for accessing IoT products. By leveraging
			the decentralized security of conventional blockchains, IoT Chain
			overcomes challenges related to transaction speed and scalability through
			PBFT and DAG. The architecture consists of a main chain and a side chain;
			the side chain executes smart contracts using coins generated from the
			main chain. The main chain employs PBFT for rapid transaction validation,
			while the side chain utilizes DAG for efficient transaction processing. SPV
			allows payment verification by checking only the headers of blocks, rather
			than all their components, which reduces verification fees and decreases user
			overhead. IoT Chain finds applications in shared economies and smart home
			technologies. In November 2018, initiatives were launched to create a
			developer ecosystem, with plans to publicly release IoT Chain in December.

		\item SLOCK.IT

			SLOCK.IT, a startup based in Germany, focuses on creating a sharing
			economy infrastructure utilizing Ethereum technology. They are in the process
			of developing the Universal Sharing Network, which integrates an automated
			payment system with Ethereum. This platform allows individuals to share and
			trade unused resources like homes or cars via blockchain technology,
			ensuring trust between parties. SLOCK.IT provides a smart lock feature,
			allowing users to unlock their assets for others by paying with tokens to
			execute Ethereum smart contracts. Additionally, users can control the keys
			required for transactions through a mobile application.

		\item JD.COM

			JD.com offers blockchain gateway services, blockchain node services, and
			blockchain consensus network services. The platform utilizes a BFT-like consensus
			algorithm and employs an authentication protocol to manage the number of accesses
			to the blockchain network. The system consists of three types of peers: consensus
			peers, gateway peers, and IoT devices. Gateway peers operate within the
			middleware layer to integrate inputs and protocols from the lower layers.
	\end{enumerate}

	\vspace{3cm}

	\subsection{Task distribution }

	\begin{table}[h]
		\caption{Task distribution}
		\def\arraystretch{1.4} \small
		\begin{tabular}{|p{3cm}|p{4.6cm}|}
			\hline
			Name         & Task                   \\
			\hline
			Dongwook Kim & Blockchain Development \\
			\hline
			Jisu Shin    & Front-end Development  \\
			\hline
			Giram Park   & Embedded Development   \\
			\hline
			Seoyoon Jung & Front-end Development  \\
			\hline
		\end{tabular}
	\end{table}

	\section{Specifications}

	\subsection{Core Requirements Specifications}

	\begin{enumerate}[itemsep=2ex, parsep=1ex]
		\item Hub Elimination and Range Extension

			To eliminate the mandatory use of Matter Hubs and extend operational range,
			we propose replacing traditional Matter hubs with Hyperledger Fabric's chaincode
			functionality. This transformation fundamentally changes how Matter
			devices communicate and operate, freeing them from the physical constraints
			of home networks.

			The core of this solution lies in implementing a message queue system within
			the blockchain. Instead of relying on a physical hub for communication, each
			Matter device interacts with a dedicated queue in the blockchain. This
			queue serves as a virtual communication channel, enabling devices to operate
			beyond the traditional home network boundaries.

			However, implementing a traditional queue structure in Hyperledger Fabric
			would be inefficient due to the complexity of transaction operations. Reading
			and writing to an array-based queue would require reading the entire array
			for each push operation, creating unnecessary overhead. To optimize this process,
			we propose using a Key-Value Store structure where the key is formatted as
			\emph{devicePK-mCTR} (device public key combined with a message counter) and
			the value contains the message payload.

			This architecture provides several advantages:
			\begin{enumerate}[itemsep=2ex, parsep=1ex]
				\item Eliminates the need for physical Matter hubs by virtualizing their
					functionality through blockchain

				\item Enables device operation from any location with internet connectivity

				\item Maintains secure and reliable communication through blockchain's inherent
					security features

				\item Optimizes performance through efficient key-value based message handling
			\end{enumerate}

		\item Data Analytics and System Reliability

			For enhanced data tracking and improved system reliability, we propose a revolutionary
			approach to data analytics and visualization that leverages blockchain's inherent
			advantages in data integrity and accessibility. Traditional IoT data
			analysis systems typically involve multiple intermediaries - data analysts
			processing raw data and relay servers transmitting processed information
			to decision-makers. This multi-step process introduces potential points of
			data corruption and creates opportunities for malicious administrators to compromise
			data integrity.

			Our solution implements a direct data access approach through a desktop application
			built with Electron, which connects directly to Hyperledger Fabric via
			gRPC. Additionally, the solution incorporates generative AI to transform
			natural language inputs into executable queries, enabling administrators without
			programming expertise to interact with and analyze raw data directly. This
			AI-powered query generation system allows non-technical users to extract
			meaningful insights from the blockchain data through intuitive language-based
			interactions.

			This approach significantly improves system reliability by ensuring that
			decision-makers work with authentic, unaltered data. The combination of
			blockchain's immutable data storage, direct access through gRPC, and AI-assisted
			query generation creates a powerful platform for data-driven decision
			making that maintains data integrity while being accessible to users
			regardless of their technical expertise.

		\item Decentralization

			Our approach to decentralization fundamentally reimagines the Matter
			protocol's architecture while maintaining its core benefits. Unlike the traditional
			Matter protocol that relies on a centralized Certificate Authority (CA) for
			device certification, Matter Tunnel eliminates this requirement while
			preserving protocol compatibility. By removing the centralized authentication
			system, we lower barriers to entry for device manufacturers and smaller
			development teams, fostering innovation and competition in the IoT ecosystem.

			This decentralized approach maintains the Matter protocol's ability to control
			multiple vendors' devices through a single application, ensuring that the key
			benefit of interoperability remains intact. The elimination of platform
			dependencies further enhances true decentralization, freeing users from vendor
			lock-in and creating a more open IoT environment.

		\item Enhanced Security and Privacy Protection

			Our solution significantly enhances End-to-End Encryption (E2EE) and user
			privacy protection by fundamentally restructuring the communication
			architecture of Matter devices. Traditional Matter implementations rely on
			a cloud-based communication model where applications communicate with cloud
			services, and Matter hubs poll these services for updates. This structure inherently
			compromises both E2EE and privacy. Matter Tunnel addresses these
			limitations through a blockchain-based approach.

			Key Security and Privacy Enhancements:

			\begin{enumerate}[itemsep=2ex, parsep=1ex]
				\item Direct Blockchain Communication

					By eliminating cloud service intermediaries, this system enables direct
					and encrypted communication between applications and devices. The
					removal of vulnerable points in the communication chain enhances
					security, while ensuring true end-to-end encryption throughout the entire
					process.

				\item Enhanced Privacy Protection

					This system maintains privacy by only exposing device public keys on
					the blockchain, while keeping sensitive information like IP addresses and
					location data completely private. This approach enables anonymous
					device operation, significantly reducing the attack surface for potential
					privacy breaches.

				\item Secure Message Counter Implementation

					The system leverages Matter devices' built-in message counter (mCTR) to
					effectively prevent message replay attacks. By maintaining
					synchronized encryption states across devices and broadcasting mCTR
					values through the blockchain for coordination, it ensures secure and synchronized
					communication between devices.
			\end{enumerate}

			By leveraging blockchain technology and existing Matter security features,
			our solution creates a more robust and private IoT ecosystem. The combination
			of anonymous operation, secure message counting, and direct blockchain communication
			ensures that both E2EE and user privacy are maintained at the highest possible
			level.

		\item Interoperability and Development Convenience

			Matter Tunnel is designed with a strong focus on interoperability and
			development convenience, offering a seamless path for integrating blockchain
			capabilities into existing Matter implementations. Our approach minimizes
			development overhead while maximizing compatibility and flexibility in
			implementation.

			The core strength of our solution lies in its simplicity of integration. Developers
			can enable blockchain support by adding just a few lines of code to their existing
			Matter applications. This minimal modification approach significantly
			reduces the barrier to entry for blockchain adoption while maintaining the
			integrity of current implementations. The system provides a straightforward
			API that abstracts the complexities of blockchain interaction, allowing
			developers to focus on their application logic rather than blockchain
			implementation details.

			A key architectural feature is the ability to separate blockchain polling
			threads from Matter processes. This separation enables applications to simultaneously
			support both traditional Matter communication and blockchain-based operations.
			Developers can implement and validate blockchain features while
			maintaining their current Matter communication channels, ensuring continuous
			service availability and reducing implementation risks.

		\item Real-Time Performance with Hyperledger Fabric

			To ensure real-time performance in Matter Tunnel, we carefully selected Hyperledger
			Fabric as our blockchain platform after extensive evaluation of various options.
			This choice was driven by Hyperledger Fabric's unique characteristics that
			align with the real-time requirements of IoT device communication.

			Hyperledger Fabric's high transaction processing capability stands out as
			a crucial feature for our implementation, supporting 2,000-20,000 transactions
			per second (TPS). This throughput is achieved through multiple channels
			that enable parallel transaction processing, effectively handling concurrent
			device communications while providing near-instantaneous transaction finality.
			The platform's configurable architecture can enhance performance by
			allowing optimization of network parameters, enabling adjustment of block creation
			time, supporting custom channel configurations for different device groups,
			and permitting fine-tuning of consensus mechanisms to match specific use case
			requirements.

			By leveraging Hyperledger Fabric's comprehensive capabilities, Matter
			Tunnel achieves the real-time performance necessary for effective IoT
			device control and monitoring. The platform's ability to handle high transaction
			volumes while maintaining low latency ensures that device interactions remain
			responsive and reliable, meeting the demanding requirements of modern IoT
			applications.
	\end{enumerate}
\end{document}