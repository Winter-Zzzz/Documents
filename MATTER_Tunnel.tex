%% Matter_Tunnel.tex
%% 2024/12/05
%% by Dongwook Kim, Jisu Shin, Giram Park, Seoyoon Jung

\documentclass[conference]{IEEEtran}

\usepackage{needspace}
\usepackage{enumitem}
\usepackage{graphicx}

\hyphenation{op-tical net-works semi-conduc-tor}

\begin{document}
\title{Matter Tunnel}

\author{ \IEEEauthorblockN{Dongwook Kim} \IEEEauthorblockA{\textit{College of Engineering} \\ \textit{Hanyang University}\\ \textit{Dept.of Information Systems}\\ Seoul, Korea \\ dongwook1214@gmail.com}
	\and \IEEEauthorblockN{Jisu Shin} \IEEEauthorblockA{\textit{College of Engineering} \\ \textit{Hanyang University}\\ \textit{Dept.of Information Systems}\\ Seoul, Korea \\ sjsz0811@hanyang.ac.kr}
	\and \IEEEauthorblockN{Giram Park} \IEEEauthorblockA{\textit{College of Engineering} \\ \textit{Hanyang University}\\ \textit{Dept.of Information Systems}\\ Seoul, Korea \\ kirammida@hanyang.ac.kr}
	\and \IEEEauthorblockN{Seoyoon Jung} \IEEEauthorblockA{\textit{College of Engineering} \\ \textit{Hanyang University}\\ \textit{Dept.of Information Systems}\\ Seoul, Korea \\ yoooonnn@naver.com}
}
\maketitle

\begin{abstract}
	Our team introduces `Matter Tunnel', which enables the Matter protocol to
	operate on a blockchain basis. Matter is a protocol that provides interoperability
	between IoT devices from various manufacturers, allowing control of multiple
	brands of IoT devices from a single application. However, due to current
	network constraints such as NAT and firewalls, a dedicated Matter hub is required
	when using IoT devices. Matter Tunnel resolves the current limitations of
	Matter by utilizing blockchain technology, operating as if creating a virtual
	private network between applications and IoT devices. Primarily, it eliminates the mandatory use of Matter hubs, enhancing user experience and extending the operational range of IoT devices. This innovation also improves device functionality support, allowing them to support diverse and specialized device functionalities
	beyond Matter's predefined device types. From an enterprise
	perspective, all device interactions and transactions are permanently recorded
	on the blockchain, providing businesses with reliable and immutable data for
	tracking device usage patterns and extracting valuable operational insights.
	Furthermore, Matter Tunnel ensures platform independence, freeing users from
	being locked into specific ecosystems. It strengthens security by enabling
	complete end-to-end encryption (E2EE) and enhances user privacy. Additionally,
	by lowering entry barriers, Matter Tunnel opens up opportunities for small development
	teams to participate in the Matter ecosystem. With these improvements, users
	can enjoy a more secure, versatile, and unrestricted IoT experience.
\end{abstract}

\begin{IEEEkeywords}
	Matter Tunnel, Matter, Blockchain, Matter hub, user experience, E2EE
\end{IEEEkeywords}

\begin{table}[h]
	\caption{Role Assignments}
	\def\arraystretch{1.24} \small
	\begin{tabular}{|p{1.8cm}|p{1.4cm}|p{4.4cm}|}
		\hline
		Roles                                                   & Name              & Task description and etc.                                                                                                                                                                                                                                                                                                                                  \\
		\hline
		Development \par manager \par Blockchain \par Developer & Dongwook \par Kim & The role involves designing and implementing solutions utilizing blockchain technology. This position sets the technical direction for projects and applies the latest blockchain trends and technologies. Responsibilities include developing smart contracts, optimizing blockchain protocols, managing team members' work, and developing their skills. \\
		\hline
	\end{tabular}
\end{table}

\needspace{10cm}

\begin{table}
	\def\arraystretch{1.24} \small
	\begin{tabular}{|p{1.8cm}|p{1.4cm}|p{4.4cm}|}
		\hline
		User, \par Customer, \par Front-end \par Developer & Jisu \par Shin    & From user and customer perspective, considers what features would enhance the IoT experience and how to improve user interaction. From a development perspective, designs and implements user interfaces for IoT applications, focusing on intuitive and responsive front-end solutions that integrate with Matter protocol and blockchain technology.                \\
		\hline
		User, \par Customer, \par Embedded \par Developer  & Giram \par Park   & From a user standpoint, evaluates how IoT devices can better serve everyday needs. As an Embedded developer, works on firmware and software for IoT devices compatible with the Matter protocol and proposed blockchain solution. Focuses on implementing user-centric features that enhance device functionality, improve reliability, and simplify setup processes. \\
		\hline
		User, \par Customer, \par AI \par Developer        & Seoyoon \par Jung & From a user perspective, studies how AI can assist with corporate decision-making needs. As an AI Developer, develops machine learning solutions to help companies make better data-driven decisions. Focuses on building AI models that analyze business data and provide clear insights for management teams.                                                       \\
		\hline
	\end{tabular}
\end{table}

\section{Introduction}

\subsection{Motivation}
The rapid advancement of Internet of Things (IoT) technology has led to significant
growth in the smart home market. However, compatibility issues among IoT devices
from various manufacturers have been a persistent challenge. To address this
problem, the Matter protocol was developed, offering an approach that enables
control of IoT devices from multiple brands through a single application.

Despite the introduction of the Matter protocol, the current implementation still
harbors several crucial issues. These problems prevent the full realization of
Matter's original goals: true interoperability, security, and protection of user
privacy.

Therefore, we believe a new approach is necessary to overcome these
limitations and maximize the potential of the Matter protocol.

\subsection{Problem Statement}
The current implementation of the Matter protocol presents the following key
issues.
\begin{enumerate}[itemsep=2ex, parsep=1ex]
	\item Mandatory use of Matter hubs
	      	      	      
	      Network limitations such as NAT and firewalls make direct P2P
	      communication between IoT devices and applications challenging in typical
	      households. This necessitates the use of dedicated Matter hubs.
	      	      	      
	\item Limited Operational Range
	      	      	      
	      As a consequence of the mandatory use of Matter hubs, the operational range
	      of Matter is confined to the home. This limitation restricts the potential
	      for broader IoT applications and remote device management beyond the
	      immediate household environment.
	      	      	      
	\item Device Functionality Constraints
	      	      	                  
	      Matter protocol defines Matter Application Cluster Library (MACL), limiting the execution of device-specific features. As modern IoT devices and appliances increasingly offer diverse and specialized functions, the protocol's standardized approach may prevent the utilization of unique or innovative device capabilities that fall outside the predefined device types and MACL.
	      	      	      
	\item Platform Dependency
	      	      	      
	      The need for platform-specific solutions, such as Apple's `HomeKit' and `HomePod'
	      or Google's `Google Home' and `Nest Hub', results in consumers being
	      locked into particular platforms.
	      	      	      
	\item Limited End-to-End Encryption (E2EE)
	      	      	      
	      The advantage of E2EE in the Matter protocol is confined to operation
	      within private networks, failing to ensure complete security throughout the
	      entire communication process.
	      	      	      
	\item Threats to User Privacy
	      	      	      
	      The communication structure that routes through cloud services potentially
	      threatens user privacy.
	      	      	      
	\item Centralized Authentication System
	      	      	      
	      Matter devices must be certified by a centralized root certificate
	      authority (CA), which makes it difficult for small development teams to participate
	      and limits consumer choices.
\end{enumerate}

These issues hinder the original purposes of the Matter protocol: interoperability,
security, and openness. Moreover, the existing centralized server-based IoT architecture
presents significant challenges for businesses.

\begin{enumerate}[itemsep=2ex, parsep=1ex]
	\item Data Reliability and Trust Issues
	      	      	      
	      The centralized server architecture creates vulnerability in data
	      integrity, making it challenging for businesses to maintain reliable records.
	      Without an immutable ledger system, businesses cannot ensure the authenticity
	      and accuracy of device interaction data, which is crucial for operation
	      analysis and service improvement.
	      	      	      
	\item Data Analysis and Tracking Problems
	      	      	      
	      Traditional centralized IoT systems lack comprehensive data tracking
	      mechanisms. This limitation makes it difficult for businesses to effectively
	      analyze device usage patterns, maintain accurate maintenance histories, and
	      extract valuable insights for business optimization. The fragmented nature
	      of data across different platforms and environments further complicates
	      the analysis process.
\end{enumerate}

Therefore, we have determined that a new approach is necessary to overcome these
limitations and fully realize the potential of the Matter protocol. We propose
a solution utilizing blockchain technology to address these challenges.

\subsection{Research on related technical elements}

\begin{enumerate}[itemsep=2ex, parsep=1ex]
	\item Matter
	      	      	      
	      We use a variety of IoT devices, and several manufacturers develop and sell
	      IoT devices with different names and appearances. It is Matter that
	      enables these various smart home devices to be connected and managed at
	      once.
	      	      	      
	      Matter is an IP-based smart home interworking standard that is compatible
	      with all devices, designed to overcome the manufacturer-dependent limitations
	      of smart home devices. It was launched in 2019 by four IoT giants Apple, Amazon,
	      Google, Samsung SmartThings, and the global association CSA, formerly the Zigbee
	      Alliance, and renamed Matter in 2021.
	      	      	      
	      Matter has the following technical features:
	      	      	      
	      Unlike existing ZigBee and Z-Wave, Matter operates based on IP protocols. Since
	      Matter is based on IP, which is a network layer protocol, the communication
	      protocol below it does not matter, and eventually all processing is done
	      at the application layer. In other words, the transmission method varies
	      depending on what the application is, but as long as IP is used, the
	      method is not important. Therefore, devices with the Matter logo can work
	      together regardless of brands or supported transmission protocols. In addition,
	      the reason why it is important to use IP is that IP protocols are already
	      proven in the market in terms of interoperability and security.
	      	      	      
	      Matter is interoperable between devices. Matter allows each device to interact
	      using the same protocol, even if it is from a different manufacturer. For example,
	      Samsung Electronics' products have been linked to SmartThings, and LG
	      Electronics only to the ThinQ platform, but now Samsung Electronics'
	      products can be connected to ThinQ. Matter is a very desirable standard from
	      a user's point of view because most homes use a mixture of products from multiple
	      brands.
	      	      	      
	      Matter supports both Wi-Fi and Thread, a low-power mesh network protocol, and
	      supports various network protocols, such as using BLE in the device
	      setting process.
	      	      	      
	      In addition, Matter has the characteristics of Multi-Admin, which uses the
	      same device in conjunction with multiple platforms, AES authentication prescribed
	      by NIST in the United States regarding data encryption, and PKI and
	      certificates for device authentication.
	      	      	      
	      An open ecosystem is being created with the introduction of Matter with
	      these characteristics, and the trend of automation and intelligence of residential
	      environments is spreading through integration with Generative AI technology.
	      Korea is also promoting active efforts to build and expand a smart home
	      ecosystem by preparing support plans in line with global trends. The Korean
	      government is expanding policy support by promoting `AI@Home', a project
	      centered on Matter and Generative AI, to support the creation of a smart home
	      ecosystem.
	      	      	      
	      However, privacy protection, application of smart home technology of existing
	      houses, and high installation costs are challenges that limit the growth
	      of the market, so it is necessary to proactively prepare countermeasures.
	      	      	      
	\item Network Constraints in P2P Communication
	      	      	      
	      In a Peer-to-Peer (P2P) structure, there is minimal reliance on always-on infrastructure
	      servers. Instead, the application allows pairs of intermittently connected
	      hosts, called peers, to communicate directly with each other. Peers are
	      desktops and laptops controlled by users rather than owned by service
	      providers, and most peers are located in homes, universities, and offices.
	      Since communication occurs directly between peers without passing through
	      a specific server, this structure is referred to as Peer-to-Peer.
	      	      	      
	      Network constraints in P2P communication, particularly due to NAT (Network
	      Address Translation) and firewalls, negatively impact user experience by introducing
	      additional complexity. NAT, which is used to map private IP addresses to public
	      ones, can prevent peers behind it from being directly accessible from
	      external networks, as it often blocks incoming connections. Firewalls
	      similarly restrict inbound traffic, further complicating direct peer-to-peer
	      communication. To address these issues, NAT traversal techniques such as STUN,
	      TURN, and ICE are commonly employed. Alternatively, users may need to manually
	      configure network settings, such as port forwarding, to bypass NAT and firewall
	      restrictions, allowing for direct communication between peers. These manual
	      configurations can be challenging for users, ultimately affecting the
	      overall user experience.
	      	      	      
	\item Matter Hub
	      	      	      
	      Matter Hub is a central component of the Matter ecosystem, designed to facilitate
	      seamless communication and interoperability between smart home devices from
	      various manufacturers. Matter aims to unify different smart home
	      technologies, allowing devices to work together regardless of brand.
	      	      	      
	      Smart Home Hubs serve as central controllers for smart home devices,
	      enabling communication between Matter-compatible devices from different
	      manufacturers. Samsung SmartThings and Amazon Echo are representative examples.
	      	      	      
	      Matter Hubs connect Matter devices to the internet and other networks,
	      allowing for remote access and control. Notable examples include Google Nest
	      Hub, which integrates with Google services, and Apple HomePod, which
	      utilizes Siri for voice commands.
	      	      	      
	      While Matter Hubs play a crucial role in enhancing interoperability within
	      the smart home ecosystem, it's important to note that using Matter devices
	      typically requires a home hub. Each application may dictate the specific
	      Matter Hub that must be used, which can strictly lock users into particular
	      platforms. This limitation highlights the need for greater flexibility and
	      broader compatibility in the Matter ecosystem to ensure a truly open and
	      user-friendly IoT environment.
	      	      	      
	\item Blockchain
	      	      	      
	      To effectively integrate blockchain technology into the IoT industry, it is
	      crucial to consider blockchains with high transaction processing speeds (TPS)
	      and enterprise-friendly features.
	      	      	      
	      Several blockchain platforms stand out for their high TPS capabilities,
	      including:
	      	      	      
	      Solana: Solana is an innovative platform designed for mainstream adoption.
	      The core development team, including co-founder Anatoly Yakovenko, focused
	      on scalability and efficiency based on their experience in building
	      telecommunications networks. By implementing Proof of Stake (PoS) and Proof
	      of History (PoH), they achieved a throughput of up to 65,000 TPS and
	      realized very low transaction fees (\$0.00025). It's also highly energy
	      efficient, a single Solana transaction uses 0.00051 kWh.
	      	      	      
	      XRP: XRP (Ripple) uses the RPCA consensus algorithm and provides
	      approximately 1,500 TPS throughput. Specialized in international remittance,
	      it has established partnerships with many banks and financial institutions,
	      featuring fast transaction completion times of 3-5 seconds.
	      	      	      
	      Hyperledger Fabric: Hyperledger Fabric is an enterprise blockchain
	      platform that provides 2,000-20,000 TPS throughput. Through various network
	      configurations, organizations can adjust the throughput and degree of
	      centralization according to their needs. For example, by modifying
	      parameters such as the number of organizations, ordering service nodes,
	      peer nodes, and channels, administrators can find the optimal balance
	      between performance and decentralization.
	      	      	      
	      Among enterprise-friendly blockchain platforms, the following are noteworthy:
	      	      	      
	      Hyperledger Fabric: Hyperledger Fabric features a modular architecture and
	      permissioned blockchain characteristics, supporting channel-based data partitioning.
	      It has various enterprise use cases including supply chain management, asset
	      tracking, identity management, and healthcare data management.
	      	      	      
	      Quorum: Quorum, developed by JP Morgan, is an enterprise blockchain based
	      on Ethereum, featuring enhanced privacy features and high throughput. It
	      supports private transactions, voting-based consensus mechanisms, role-based
	      access control, and multi-signature contracts.
	      	      	      
	      Hyperledger Besu: Hyperledger Besu is a Java-based blockchain platform
	      compatible with Ethereum. It supports both public and private networks and
	      provides enterprise-grade governance. It is being utilized in various
	      fields including financial services, supply chain management, digital
	      asset management, and inter-enterprise collaboration platforms.
	      	      	      
	      After careful consideration of these options, Hyperledger Fabric is judged
	      as the most suitable blockchain platform for the IoT industry. It offers a
	      combination of high TPS and enterprise-grade features that are essential for
	      large-scale IoT implementations. Furthermore, Hyperledger Fabric is compatible
	      with Monachain, a blockchain platform developed by LG CNS based on
	      Hyperledger Fabric. This compatibility allows for seamless integration and
	      immediate application in existing systems, potentially accelerating
	      adoption and reducing implementation barriers.
	      	      	      
	\item Arduino
	      	      	      
	      Arduino is an open-source electronics platform based on easy-to-use
	      hardware and software. In the context of Matter IoT, Arduino plays a
	      significant role due to its flexibility, ease of use, and strong community
	      support. When considering Arduino for Matter IoT applications, the
	      following aspects are crucial:
	      	      	      
	      Processing power: Matter protocol requires sufficient computational resources
	      to handle encryption and network communication.
	      	      	      
	      Connectivity options: Wi-Fi or Ethernet capability is essential for Matter,
	      as it's IP-based.
	      	      	      
	      Memory capacity: Adequate RAM and flash memory to run Matter stack and
	      application code.
	      	      	      
	      Power efficiency: For battery-operated IoT devices, low power consumption
	      is critical.
	      	      	      
	      Compatibility with Matter SDK: The board should be capable of running the Matter
	      SDK.
	      	      	      
	      Some Arduino boards suitable for Matter IoT projects include:
	      	      	      
	      Arduino Nano 33 IoT: This compact board features Wi-Fi connectivity and a
	      powerful SAMD21 microcontroller, making it suitable for small Matter devices.
	      	      	      
	      Arduino MKR WiFi 1010: With its low power consumption and robust Wi-Fi
	      capabilities, it's excellent for battery-operated Matter devices.
	      	      	      
	      Arduino Portenta H7: This high-performance board with dual-core processor
	      and multiple connectivity options is ideal for more complex Matter
	      applications.
	      	      	      
	      ESP32-S3: While not an official Arduino board, the ESP32-S3 is widely used
	      in the Arduino ecosystem and offers powerful processing capabilities, Wi-Fi
	      and Bluetooth connectivity, and ample memory.
	      	      	      
	      These boards offer various combinations of processing power, connectivity,
	      and memory, allowing developers to choose the most suitable option for their
	      specific Matter IoT application.
	      	      	      
	\item Web Assembly
	      	      	      
	      Web Assembly (Wasm) is a binary instruction format designed for efficient
	      execution in web browsers. It serves as a portable target for compilation of
	      high-level languages like C, C++, and Rust, enabling deployment on the web
	      for client and server applications.
	      	      	      
	      Key features and benefits of Web Assembly in the context of IoT and Matter
	      include:
	      	      	      
	      Language Versatility: Web Assembly allows developers to use languages like
	      C, C++, or Rust in web environments. This is particularly beneficial for
	      IoT applications, as these languages are commonly used in embedded systems
	      development.
	      	      	      
	      Performance: Wasm provides near-native performance, making it suitable for
	      computationally intensive tasks often required in IoT applications.
	      	      	      
	      Code Reusability: It enables the use of the same codebase across different
	      platforms - from embedded devices to web interfaces. This is especially valuable
	      for functions like encryption and decryption, where consistent
	      implementation across platforms is crucial.
	      	      	      
	      Frontend Capabilities: Web Assembly empowers frontend applications to perform
	      complex operations typically associated with backend or embedded environments.
	      This can include data processing, encryption, and other intensive
	      computations directly in the browser.
	      	      	      
	      In the context of Matter Tunnel, Web Assembly can play a significant role in
	      creating consistent, high-performance interfaces for controlling and
	      managing IoT devices across different platforms. It allows developers to implement
	      complex Matter protocol operations in web applications, maintaining
	      consistency with the implementations on the devices themselves. This consistency
	      across platforms is particularly valuable for ensuring that security measures,
	      device interactions, and data handling are uniform across the entire Matter
	      ecosystem.
	      	      	      
	\item gRPC
	      	      	      
	      gRPC is a modern open source high performance Remote Procedure Call (RPC)
	      framework that can run in any environment. It can efficiently connect services
	      in and across data centers with pluggable support for load balancing,
	      tracing, health checking and authentication.
	      	      	      
	      High Performance and Efficiency: gRPC is designed based on HTTP/2, allowing
	      it to support features like multiplexing, server push, and streaming.
	      These characteristics enable efficient management of connections between
	      multiple clients and servers, minimizing latency and optimizing bandwidth.
	      This ensures that high performance is maintained even in large-scale
	      distributed systems.
	      	      	      
	      Protocol Buffers: gRPC uses Protocol Buffers for service definition.
	      Protocol Buffers is a powerful binary serialization tool developed by
	      Google, allowing the definition of data structures and their conversion into
	      various programming languages. This increases the efficiency of data
	      transmission and maintains consistency in APIs.
	      	      	      
	      Easy Installation and Scalability: gRPC offers simple installation,
	      allowing developers to set up runtime and development environments with
	      just a single command. It also provides scalability that can handle millions
	      of RPCs per second, making it suitable for large-scale applications. This allows
	      developers to quickly build and operate services without complex infrastructure
	      setups.
	      	      	      
	      Cross-Language and Platform Support: gRPC works across multiple
	      programming languages and platforms. It can automatically generate idiomatic
	      client and server stubs for various languages such as Java, C++, Python,
	      Go, and Ruby. This facilitates collaboration between teams using different
	      tech stacks and enhances code reusability.
	      	      	      
	      Bi-Directional Streaming: gRPC supports bi-directional streaming between clients
	      and servers. This allows clients to send data to the server while
	      simultaneously receiving streamed data from the server. This feature is particularly
	      useful in applications that require real-time data transmission.
	      	      	      
	      Integrated Authentication and Security: gRPC enhances security by
	      integrating full pluggable authentication features at the HTTP/2 transport
	      layer. This simplifies the implementation of user authentication and authorization,
	      ensuring safe transmission of sensitive data.
	      	      	      
	      Due to these features, gRPC is widely used in various fields, including
	      microservices architecture, IoT applications, and mobile backend services.
	      By using gRPC, developers can build efficient and scalable systems, making
	      it easier to manage communication between services.
\end{enumerate}

\subsection{Research on related study}

\begin{enumerate}[itemsep=2ex, parsep=1ex]
	\item Benefits of Blockchain for Data Mining
	      	      	      
	      The integration of blockchain technology and data mining techniques for anomaly detection provides efficient methods through their combined application. Data stored on the blockchain can be treated as big data, and data mining techniques enable the extraction of hidden patterns and insights.
	      	      	      
	      This paper explores analytical approaches to blockchain data and practical applications, demonstrating how these technologies enhance anomaly detection and fraud prevention. Blockchain's transparent recording system facilitates data tracking and monitoring. It serves as an effective tool for corporate data analysis, with a key advantage being rapid detection of operational changes or fraudulent activities.
	      	      	      
	      Key applications from the literature include:
	      	      	      
	      \begin{enumerate}[itemsep=2ex, parsep=1ex]
	      	\item Analysis of Bitcoin Transaction Networks
	      	      	      	      	      	      
	      	      Zola et al. (2019) analyzed changes in Bitcoin transaction patterns by utilizing the time-series data of the blockchain. They used data from WalletExplorer and the Bitcoin mainnet over the past three years, calculating F1 scores through k-fold cross-testing. By analyzing the transaction data linked in chronological order on the blockchain, it is possible to detect cybersecurity threats and identify changes in behavioral patterns.
	      	      	      	      	      	      
	      	\item Blockchain Data Collection and Analysis
	      	      	      	      	      	      
	      	      Brinckman et al. (2019) presented techniques for crawling, collecting, and analyzing blockchain data. They demonstrated a method for clustering transactions and extracting account characteristics to identify fraudulent accounts, which serves as a good example of understanding the data structure of the blockchain and effectively analyzing it.
	      	      	      	      	      	      
	      	\item Time-Series Transaction Data Analysis
	      	      	      	      	      	      
	      	      Zhao et al. (2021) analyzed the entire dataset of the Ethereum blockchain from a temporal perspective. They utilized the ethereum blockchain dataset from the Bigquery Public Data Repository to examine changes in transaction patterns over time, comparing the accuracy of Random Forest and Logistic Regression, and visualizing the temporal evaluation of the collected data.
	      \end{enumerate}
	      	      	      
	      These application examples demonstrate that effective analysis is possible by leveraging the connected data structure and temporal characteristics of the blockchain. In particular, the data structure of the blockchain can be effectively utilized to identify patterns in sensitive transaction data and detect anomalous behaviors, which can be considered a significant advantage of blockchain-based data analysis.
	      	      	      
	\item Benefits of Blockchain for Data Integrity and Accessibility
	      	      	      
	      Blockchain technology has emerged as an innovative solution in healthcare data management, addressing important challenges in data security, integrity, and interoperability. Here are three representative implementations that demonstrate the applications of blockchain in healthcare:
	      	      	      
	      \begin{enumerate}[itemsep=2ex, parsep=1ex]
	      	\item MedRec
	      	      	      	      	      	      
	      	      MedRec is a blockchain project in healthcare that enables comprehensive management of medical data, including data provision by medical institutions, patient licensing, and data utilization by research institutes. MedRec 2.0 is implemented using Go-ethereum and Solidity languages, utilizing smart contracts on the Ethereum blockchain for intermediary-free data exchange. Like other blockchains, MedRec ensures security through its blockchain nature. Due to decentralization, data is maintained across all network nodes and stored in nodes of patients and their service providers. The blockchain's consensus mechanism prevents security issues from single-point vulnerabilities. Additionally, if one node attempts to modify a specific transaction in a block, that modified node becomes inconsistent with others and is excluded from consensus, maintaining record integrity.
	      	      	      	      	      	      
	      	\item Estonian e-Health
	      	      	      	      	      	      
	      	      The Estonian e-Health Foundation and Guardtime have strengthened security and patient monitoring by implementing KSI blockchain technology. The healthcare system integrates medical service data through X-Road, a data exchange platform, enabling comprehensive medical service data analysis. This system provides personal healthcare data integrity verification, with healthcare providers transmitting data integrity to the KSI server for permanent blockchain and offline recording. Through the ``e-prescription service", doctors and pharmacies can verify prescription integrity. Additionally, the ``electronic health registration service" allows doctors to access medical images like X-rays using only the patient's ID code. These systems enable secure storage, efficient sharing, and integrity verification of medical data.
	      	      	      	      	      	      
	      	\item Mediblock
	      	      	      	      	      	      
	      	      Mediblock is a blockchain-based integrated management platform for personal medical data that aims to integrate distributed patient medical data. Mediblock secures data by granting patients access to medical data and allowing only them to decrypt the entire data, while recording data hash values on the blockchain for integrity. It improves data reliability by restricting medical record creation to certified medical personnel, and ensures transparency by recording data access information and rights on the blockchain. Additionally, it enables safe data sharing through relay services and secondary backup storage, while facilitating secure data transactions through an encrypted data trading market within the platform.
	      \end{enumerate}
	      	      	      
	      These implementations demonstrate how blockchain technology enhances healthcare data management through improved integrity, security, accessibility, and transparency. These advantages can be applied to blockchain-based Matter tunnels, potentially replacing Matter hubs to enhance the reliability and efficiency of the Matter protocol.
	      	      	      
	\item Hyperledger Fabric Performance
	      	      	      
	      The performance of Hyperledger Fabric is often questioned, particularly regarding
	      transaction processing speeds (TPS) in comparison to other blockchain platforms.
	      The performance of a Fabric network is complex due to the involvement of
	      multiple organizations with varying hardware and networking
	      infrastructures, as well as factors such as the number of channels and chaincode
	      implementation.
	      	      	      
	      Hyperledger Fabric 2.x has introduced performance improvements over
	      version1.4, which is no longer in long-term support (LTS). Fabric 2.5 is the
	      latest LTS version and includes a new peer gateway service. This service, along
	      with the new gateway SDK, is expected to enhance the performance of
	      applications.
	      	      	      
	      \begin{enumerate}[itemsep=2ex, parsep=1ex]
	      	\item Hardware and Topology
	      	      	      	      	      	      
	      	      In Hyperledger Fabric, the topology used consists of two peer organizations
	      	      (PeerOrg0 and PeerOrg1), each with one peer node, along with a single ordering
	      	      service organization (OrdererOrg0) that utilizes Raft consensus with one
	      	      ordering service node. TLS is enabled on each node.
	      	      	      	      	      	      
	      	      All nodes utilized the same hardware configuration:
	      	      	      	      	      	      
	      	      - Intel(R) Xeon(R) Silver 4210 CPU @ 2.20GHz
	      	      	      	      	      	      
	      	      - 40 Cores made up of 2 CPUs. Each CPU has 10 physical cores supporting
	      	      20 Threads in total
	      	      	      	      	      	      
	      	      - 64Gb Samsung 2933Mhx Memory
	      	      	      	      	      	      
	      	      - MegaRAID Tri-Mode SAS3516 (MR9461-16i) disk controller
	      	      	      	      	      	      
	      	      - Intel 730 and DC S35x0/3610/3700 Series SSD attached to disk
	      	      controller
	      	      	      	      	      	      
	      	      - Ethernet Controller X710/X557-AT 10GBASE-T
	      	      	      	      	      	      
	      	      - Ubuntu 20.04
	      	      	      	      	      	      
	      	      All machines were connected to the same switch. Hyperledger Fabric was
	      	      deployed natively across three physical machines, meaning that the native
	      	      binaries were installed and executed without using container technologies
	      	      such as Docker or Kubernetes.
	      	      	      	      	      	      
	      	\item Hyperledger Fabric Application Configuration
	      	      	      	      	      	      
	      	      State Database: LevelDB was used as the state database.
	      	      	      	      	      	      
	      	      Gateway Service: The concurrency limit was set to 20,000.
	      	      	      	      	      	      
	      	      Application Channel: A single application channel was created with two
	      	      peers and an orderer, configured to use V1\_4 capabilities for lifecycle
	      	      deployment while other capabilities were set to V2\_0.
	      	      	      	      	      	      
	      	      Configuration: Only the application channel existed, no private data
	      	      was used, and default policies applied. No range or JSON queries were
	      	      conducted, and the network was TLS-enabled without mutual TLS.
	      	      	      	      	      	      
	      	      Chaincode: Go chaincode (fixed-asset-base) was deployed without the
	      	      Contract API, and an endorsement policy of “1 Of Any” was specified.
	      	      	      	      	      	      
	      	\item Load Generator
	      	      	      	      	      	      
	      	      Hyperledger Caliper 0.5.0 served as both the load generator and for generating
	      	      report outputs. It was configured to work with Fabric 2.4, utilizing the
	      	      peer Gateway Service to initiate and assess transactions, with all
	      	      transactions originating from a single organization directed to its
	      	      gateway peer. The load was defined based on the fixed-asset benchmark from
	      	      Hyperledger Caliper-Benchmarks.
	      	      	      	      	      	      
	      	      Four bare-metal machines were employed to host remote Caliper workers and
	      	      a single Caliper manager, which was responsible for generating the
	      	      load on the Hyperledger Fabric network. To create sufficient workload on
	      	      the Fabric network, multiple Caliper workers were necessary, corresponding
	      	      to the number of clients currently connected to the network. The
	      	      results section includes details on the number of Caliper workers
	      	      utilized.
	      	      	      	      	      	      
	      	\item Key Points
	      	      	      	      	      	      
	      	      The results presented here were generated using the latest builds of
	      	      Hyperledger Fabric 2.5, employing the default node and channel configuration
	      	      values. The block cutting parameters used were as follows:
	      	      	      	      	      	      
	      	      \textit{Block Cut Time: 2 seconds}
	      	      	      	      	      	      
	      	      \textit{Block Size: 500}
	      	      	      	      	      	      
	      	      \textit{Preferred Maximum Bytes: 2 MB}
	      	      	      	      	      	      
	      	      To avoid hitting concurrency limits while pushing enough workload
	      	      through, the gateway concurrency limit was set to 20,000.
	      	      	      	      	      	      
	      	      Only a single channel was utilized, meaning the peer did not leverage its
	      	      full resource potential.
	      	      	      	      	      	      
	      	      The chaincode was optimized for these tests, and real-world chaincode
	      	      is expected to perform less efficiently.
	      	      	      	      	      	      
	      	      The Caliper workload generator was also optimized for transaction
	      	      throughput, whereas real-world applications would involve client implementations
	      	      that may introduce latency.
	      	      	      	      	      	      
	      	      Transactions were sent to a single gateway peer from the same organization;
	      	      real-world scenarios would likely involve multiple organizations
	      	      sending transactions concurrently, potentially leading to higher TPS results.
	      	      	      	      	      	      
	      	      Using the gateway service allowed for better performance as blocks were
	      	      not received via the delivery service to confirm transaction completion,
	      	      enhancing both client and network performance compared to the legacy
	      	      node SDK.
	      	      	      	      
	      	      \clearpage
	      	      	      	      	      	      
	      	\item Result
	      	      	      	      	      	      
	      	      \begin{table}[h!]
	      	      	\caption{Hyperledger Fabric Benchmark Result}
	      	      	\def\arraystretch{1.4} \small
	      	      	\begin{tabular}{|p{1.7cm}|p{1.5cm}|p{1.5cm}|p{1.7cm}|}
	      	      		\hline
	      	      		Name                                        & Max \par Latency & Average \par Latency & Throughput \\
	      	      		\hline
	      	      		create \par asset 100                       & 2.13             & 0.33                 & 2946.7     \\
	      	      		\hline
	      	      		create \par asset 1000                      & 3.21             & 1.52                 & 2938.9     \\
	      	      		\hline
	      	      		read write \par assets previously read 100  & 0.21             & 0.06                 & 2544.3     \\
	      	      		\hline
	      	      		read write \par assets previously read 1000 & 0.26             & 0.11                 & 1527.0     \\
	      	      		\hline
	      	      	\end{tabular}
	      	      \end{table}
	      	      	      	      	      	      
	      	      \begin{enumerate}[itemsep=2ex, parsep=1ex]
	      	      	\item Blind Write of a Single Key 100 Byte Asset Size
	      	      	      	      	      	      	      	      	      
	      	      	      Caliper test configuration:
	      	      	      	      	      	      	      	      	      
	      	      	      - workers: 200
	      	      	      	      	      	      	      	      	      
	      	      	      - fixed-tps : 3000
	      	      	      	      	      	      	      	      	      
	      	      	      The TPS value represents the peak performance achieved during the tests.
	      	      	      Attempts to exceed this throughput resulted in unexpected failures
	      	      	      and a decrease in overall throughput.
	      	      	      	      	      	      	      	      	      
	      	      	\item Blind Write of a Single key 1000 Byte Asset Size
	      	      	      	      	      	      	      	      	      
	      	      	      Caliper test configuration:
	      	      	      	      	      	      	      	      	      
	      	      	      - workers: 200
	      	      	      	      	      	      	      	      	      
	      	      	      - fixed-tps : 3000
	      	      	      	      	      	      	      	      	      
	      	      	      The throughput remains the same as that observed with the 100-byte
	      	      	      blind write benchmark; however, latency increases.
	      	      	      	      	      	      	      	      	      
	      	      	\item Read Write of a Single Key 100 Byte Asset Size
	      	      	      	      	      	      	      	      	      
	      	      	      Caliper test configuration:
	      	      	      	      	      	      	      	      	      
	      	      	      - workers: 200
	      	      	      	      	      	      	      	      	      
	      	      	      - fixed-tps, tps: 2550
	      	      	      	      	      	      	      	      	      
	      	      	      The above results were achieved under the expectation of no failures.
	      	      	      The latency remained very low, indicating that the Fabric network was
	      	      	      reaching its capacity during this test.
	      	      	      	      	      	      	      	      	      
	      	      	\item Read Write of a Single Key 1000 Byte Asset Size
	      	      	      	      	      	      	      	      	      
	      	      	      Caliper test configuration:
	      	      	      	      	      	      	      	      	      
	      	      	      - workers: 200
	      	      	      	      	      	      	      	      	      
	      	      	      - fixed-tps, tps: 1530
	      	      \end{enumerate}
	      \end{enumerate}
	      	      	      
	      The above results were achieved under the expectation of no failures. The
	      latency remained very low, indicating that the Fabric network was reaching
	      its capacity during this test.
\end{enumerate}

\section{Requirements}

\subsection{Core Requirements}
The solution proposed in this study must meet the following key requirements

\begin{enumerate}[itemsep=2ex, parsep=1ex]
	\item Eliminate the mandatory use of Matter Hubs
	      	      	      
	      It must eliminate the mandatory use of Matter Hubs, enabling direct and secure
	      device-to-application communication without relying on intermediary hardware.
	      This removal of hub dependency not only reduces system complexity and cost
	      but also enhances system reliability by eliminating single points of
	      failure.
	      	      	      
	\item Extended Operational Range
	      	      	      
	      The solution should overcome the limitation of traditional Matter Hub based
	      systems, which confine device operation to a home network. It must enable
	      secure and efficient management and control of IoT devices from remote
	      locations, expanding the utility of Matter-compatible devices beyond the immediate
	      household environment. This extended range should not compromise security
	      or user privacy.
	      	      	      
	\item Enhanced Device Functionality Support
	      	      	                  
	      The solution must provide mechanisms to support diverse and specialized device functionalities beyond Matter's predefined device types. It should implement an expandable approach that allows manufacturers to define and integrate custom device capabilities. This enhancement enables the full utilization of modern IoT devices' innovative features without being constrained by standardized device type limitations, fostering technological advancement and product differentiation in the IoT ecosystem.
	      	      	      
	\item Enhanced data tracking mechanisms
	      	      	      
	      The solution must incorporate enhanced data tracking mechanisms that
	      provide comprehensive visibility into device operations, interactions.
	      This tracking system should maintain tamper-proof records of all device
	      activities, enabling businesses to analyze usage patterns, monitor
	      performance metrics, and optimize their operations effectively. The
	      implementation should support both real-time monitoring and historical
	      data analysis while maintaining user privacy and data security.
	      	      	      
	\item Improved system reliability
	      	      	      
	      To ensure system reliability and trust, all data interactions and transactions
	      must be recorded on an immutable ledger, creating a verifiable and
	      transparent history of device operations. This trustworthy data foundation
	      is crucial for both operational intelligence and regulatory compliance,
	      allowing businesses to make data-driven decisions with confidence. The system
	      should provide mechanisms for data verification and validation while maintaining
	      appropriate access controls and privacy measures.
	      	      	      
	\item Decentralization
	      	      	      
	      The solution should embrace decentralization by removing dependencies on
	      centralized certificate authorities (CAs) and platform-specific ecosystems.
	      This decentralization should establish a more democratic and open IoT ecosystem
	      where small developers and manufacturers can participate freely, fostering
	      innovation and competition. Furthermore, the solution allow IoT devices
	      from various manufacturers to interact seamlessly.
	      	      	      
	\item Enhanced End-to-End Encryption (E2EE)
	      	      	      
	      E2EE should be guaranteed even outside private networks, maintaining data confidentiality
	      throughout the entire communication process.
	      	      	      
	\item User Privacy Protection
	      	      	      
	      It should reduce reliance on centralized cloud services for communication
	      and minimize the collection and use of user data while managing it
	      transparently.
	      	      	      
	\item Real-time Performance
	      	      	      
	      The solution must support real-time communication and responsiveness, even
	      when utilizing blockchain technology. It should ensure that blockchain
	      integration does not introduce significant latency or delays in device
	      interactions. The system should maintain quick response times for user commands
	      and device state updates, while leveraging the benefits of blockchain for
	      enhanced security and decentralization.
	      	      	                  
\end{enumerate}
            
By meeting these requirements, the proposed solution is expected to overcome
the limitations of the current Matter hub-based Matter protocol and provide
a better user experience, security, and privacy.

\subsection{Development Requirements}
{\centering \textbf{Client} \par}
\begin{enumerate}[itemsep=2ex, parsep=1ex]
	\item User Authentication
	      	      	      
	      The login system should prioritize security, simplicity, and compatibility
	      with the Matter protocol. To achieve this, we propose implementing a login
	      mechanism based on asymmetric cryptography, specifically using the secp256k1
	      elliptic curve algorithm, which is also employed by Matter. Users can easily
	      log in by entering their secp256k1 private key instead of using social login
	      methods.
	      	      	      
	      \begin{enumerate}[itemsep=2ex, parsep=1ex]
	      	\item Sign Up
	      	      	      	      	      	      
	      	      Users can initiate the registration process by clicking the Sign Up
	      	      button on the login page of the application. During the sign-up process,
	      	      users will create their private key, which will serve as their unique identifier
	      	      for logging in. Generated private key will be securely stored in local
	      	      storage. If desired, users can retrieve their private key at any time
	      	      from the application's account management section. This feature allows
	      	      users to back up or transfer their private key to another device if needed.
	      	      	      	      	      	      
	      	\item Login
	      	      	      	      	      	      
	      	      Users can log in by entering their private key. Upon successful login,
	      	      a public key is derived from the private key, and users are directed to
	      	      a page where they can register Matter devices. If the entered key
	      	      doesn't match the required format, an error message starting ``Your key
	      	      is incorrectly formatted" will be displayed.
	      \end{enumerate}
	      	      	      
	\item Add Device
	      	      	      
	      To register a Matter-compatible device, the user clicks the ‘+’ button to add
	      a new device. User can scan the QR code or enter the setup code provided
	      by the device to proceed with the registration. The device information will
	      be stored in the local storage. Once the device is successfully registered,
	      the user gets registration confirmation message and will be directed to a screen
	      displaying the device status and features.
	      	      	      
	\item Remove Device
	      	      	      
	      To remove an unnecessary device, the user select the device and click the ‘Remove’
	      button. The user will be prompted to confirm choice before the device is
	      removed from the system. Upon confirming the removal, the system will
	      delete the device from the local storage and the user will receive a
	      message confirming the successful deregistration. If an error occurs during
	      the process, an error message will be provided.
	      	      	      
	\item Device Control
	      	      	      
	      A user-friendly interface will be designed for controlling each device, focusing
	      on intuitive navigation and clear functionality. The interface will
	      display available control options. When the user issues a command to control
	      a device, the command will be executed through communication with the
	      blockchain and the device. Feedback will be provided to the user upon successsful
	      command execution. User can set devices to operate automatically based on
	      specific time or conditions. The application will allow users to configure
	      and manage their automations.
	      	      	      
	\item Data Display
	      	      	      
	      Matter devices transmit a variety of signals to the application through
	      the Matter Tunnel. These signals are structured in various formats to accommodate
	      diverse data types and use cases. The formats include, but are not limited
	      to, JSON, binary data. Each format serves a specific purpose, allowing for
	      flexible and efficient data transmission.
	      	      	      
	      The application is responsible for processing these diverse signals and presenting
	      them to users in a manner that aligns with their respective data formats.
	      Upon receiving the signals, the application will parse and interpret the data
	      to ensure it is accurately represented. The transformed data will then be
	      displayed in a user-friendly and intuitive interface that enhances the
	      overall user experience.
	      	      	      
	      The application will be designed to automatically update the user
	      interface in real-time, reflecting any changes in the device status or incoming
	      data. This ensures that users have access to the most current information
	      available, allowing for informed decision-making and timely responses.
	      	      	      
	      By supporting various signal formats and providing a clear, interactive
	      display, the application aims to enhance the usability and effectiveness of
	      Matter devices within the connected ecosystem.
\end{enumerate}

{\centering \textbf{Dashboard} \par}
\begin{enumerate}[itemsep=2ex, parsep=1ex]
	\item Integrated Dashboard
	      	      	      
	      The Integrated Dashboard provides a unified view of the status of all
	      devices and the network. Through visual representations of real-time device
	      statuses, transaction logs, and key metrics, users can monitor and
	      understand system performance at a glance, enabling quick situation assessment.
	      	      	      
	\item Natural Language Query System
	      	      	      
	      The Natural Language Query System supports intuitive natural language
	      input for querying blockchain data. The AI model converts user questions into
	      precise data query commands, allowing users without programming knowledge
	      to access and analyze blockchain data. This interface enhances accessibility,
	      making data analysis easy for both technical and non-technical users.
	      	      	      
	\item Data Visualization and Analysis Tools
	      	      	      
	      Data Visualization and Analysis Tools improve the platform’s usability by
	      presenting blockchain data results in accessible visual formats. Charts,
	      graphs, and other visuals provide clear insights, while advanced analytics
	      functionalities let users analyze transaction logs, and real-time performance
	      indicators (KPIs) to generate meaningful insights for decision-making.
	      	      	      
	\item Insight and Report Service
	      	      	      
	      The Insight and Report Service features automatically generates daily, weekly,
	      and monthly operational reports. These reports allow decision-makers to track
	      performance trends, identify areas needing attention, and access actionable
	      insights and recommendations, supporting effective planning and strategic
	      problem-solving.
\end{enumerate}

\section{Development environment}

\subsection{software development platform}

\begin{enumerate}[itemsep=2ex, parsep=1ex]
	\item JavaScript
	      	      	              
	      \begin{figure}[h!]
	      	\centering
	      	\includegraphics[width=0.5\linewidth]{javascript.png}
	      	\caption{JavaScript}
	      	\label{fig:JavaScript}
	      \end{figure}
	      	      	      
	      JavaScript is a programming language used to make web pages dynamic, allowing
	      for content changes in response to user interactions. It evolved from
	      historically static web pages and is now utilized in both client-side and
	      server-side development, with various libraries and frameworks expanding its
	      functionality. JavaScript is interpreted by the browser, modifying the DOM
	      in response to user events on the client side and generating dynamic content
	      by interacting with databases on the server side. Additionally, when combined
	      with HTML and CSS, it enhances the UI of web applications and allows for
	      efficient task execution. As a client-side scripting language, JavaScript
	      is one of the core technologies of the World Wide Web. Its features can improve
	      the user experience of websites, from refreshing social media feeds to
	      animations and interactive map displays. For example, when browsing the
	      internet, if you encounter an image slideshow, a dropdown menu that
	      appears upon clicking, or dynamic color changes of objects on a webpage,
	      you are witnessing the effects of JavaScript in action. Its selection for this
	      project is driven by the team's familiarity with it, which enhances
	      efficiency and productivity.
	      	      
	      \vspace{5cm}
	      	      	      
	\item React
	      	      	      
	      \begin{figure}[h!]
	      	\centering
	      	\includegraphics[width=0.5\linewidth]{react.png}
	      	\caption{React}
	      	\label{fig:React}
	      \end{figure}
	      	      	      
	      React is a JavaScript library developed by Facebook, primarily used for building
	      user interfaces (UI). It has a component-based structure, allowing developers
	      to create reusable components for UI construction. React uses a Virtual
	      DOM to efficiently handle updates and optimize performance, making it widely
	      used in the front-end development of web applications. The decision to use
	      React for this project was influenced by the fact that the team conducted
	      a study on React together during their vacation, enhancing their familiarity
	      and readiness to implement it effectively.
	      	      	      
	\item Electron
	      	      	      
	      \begin{figure}[h!]
	      	\centering
	      	\includegraphics[width=0.5\linewidth]{electron.png}
	      	\caption{Electron}
	      	\label{fig:Electron}
	      \end{figure}
	      	      	      
	      Electron is a framework for building desktop applications using JavaScript,
	      HTML, and CSS. By embedding Chromium and Node.js into its binary, Electron
	      enables developers to maintain a single JavaScript codebase and create
	      cross-platform applications that work on Windows, macOS, and Linux. Popular
	      desktop applications like Slack and Visual Studio Code are developed using
	      Electron. Implementing the dashboard as a desktop application is
	      advantageous for local use, particularly when want to use AI function in local,
	      which makes Electron a suitable choice for this project. This decision reflects
	      the need for a robust and efficient local application to meet the project’s
	      requirements.
	      	      
	      \vspace{5cm}
	      	      	      
	\item Go
	      	      	      
	      \begin{figure}[h!]
	      	\centering
	      	\includegraphics[width=0.5\linewidth]{go.png}
	      	\caption{Go}
	      	\label{fig:Go}
	      \end{figure}
	      	      	      
	      Go (or Golang) is an open-source programming language developed by Google,
	      designed to be fast and concise while supporting concurrency, making it
	      ideal for network applications and server-side programming. Its straightforward
	      syntax and ease of error handling contribute to its popularity in projects
	      that require high performance and efficiency. Additionally, Go boasts a large
	      ecosystem of partners, communities, and tools, making it easy to learn and
	      fostering effective team collaboration. The decision to use Go for this project
	      is influenced by the requirement that Hyperledger Fabric must be developed
	      using Go, ensuring compatibility and optimal performance within the
	      blockchain framework.
	      	      	      
	\item gRPC
	      	      	      
	      \begin{figure}[h!]
	      	\centering
	      	\includegraphics[width=0.5\linewidth]{gRPC.png}
	      	\caption{gRPC}
	      	\label{fig:gRPC}
	      \end{figure}
	      	      	      
	      gRPC is a high-performance, open-source Remote Procedure Call (RPC) framework
	      initially developed by Google. It uses HTTP/2 for transport and Protocol
	      Buffers as the interface description language, enabling efficient communication
	      between distributed systems across different languages and platforms. gRPC
	      excels in scenarios requiring high-throughput and low-latency
	      communication, making it particularly suitable for microservices architectures.
	      The framework's strong typing system, bidirectional streaming capabilities,
	      and built-in support for authentication enhance the reliability and
	      security of service-to-service communication. The decision to use gRPC for
	      this project was influenced by its essential role in communicating with the
	      Hyperledger Fabric gateway. Since the Electron-based frontend needs to
	      interact with the Hyperledger Fabric network through its gateway, gRPC provides
	      the necessary protocol and tools to establish this communication efficiently
	      and securely.
	      	      
	      \vspace{5cm}
	      	                 
	\item C++
	      	      	      
	      \begin{figure}[h!]
	      	\centering
	      	\includegraphics[width=0.5\linewidth]{C++.png}
	      	\caption{C++}
	      	\label{fig:C++}
	      \end{figure}
	      	      	      
	      C++ is a high-performance, object-oriented programming language developed by
	      Bjarne Stroustrup as an extension of the C language. It is widely used in
	      various applications, including game engines, system software, IoT,
	      embedded systems, and graphics processing. C++ provides a clear program
	      structure and enables code reuse, which helps reduce development costs,
	      while also offering portability for creating applications that can adapt
	      to multiple platforms. Furthermore, C++ offers a high level of control
	      over system resources and memory. This project leverages C++ for building Arduino
	      and WebAssembly applications, capitalizing on its efficiency and
	      versatility in these domains.
	      	      	      
	\item Arduino
	      	      	      
	      \begin{figure}[h!]
	      	\centering
	      	\includegraphics[width=0.5\linewidth]{arduino.png}
	      	\caption{Arduino}
	      	\label{fig:Arduino}
	      \end{figure}
	      	      	      
	      Arduino is an open-source electronics platform based on easy-to-use
	      hardware and software, primarily used in IoT and embedded systems projects.
	      Arduino boards can read inputs—such as light from a sensor, a finger press
	      on a button, or a Twitter message—and turn them into outputs, like activating
	      a motor, lighting an LED, or publishing data online. Programmed using C/C++,
	      Arduino enables rapid prototyping by connecting various sensors and actuators,
	      and is also popular for educational and DIY projects. Users can control their
	      boards by sending instructions to the microcontroller, allowing for
	      flexible and dynamic applications. Arduino was chosen for this project because
	      it offers a simple way to develop embedded systems, streamlining the
	      development process and enhancing efficiency.
	      	      	      
	\item Python
	      	      	      
	      \begin{figure}[h!]
	      	\centering
	      	\includegraphics[width=0.5\linewidth]{python.png}
	      	\caption{Python}
	      	\label{fig:Python}
	      \end{figure}
	      	      	      
	      Python is a high-level programming language with concise and easy-to-read
	      syntax. It is used in various fields, including data science, artificial intelligence,
	      web development, automation, and scripting. Thanks to its rich libraries and
	      community support, Python enables rapid prototyping and highly productive
	      development. In this project, Python is chosen specifically for AI
	      development, leveraging its capabilities to create efficient and effective
	      AI solutions.
	      	      	      
	\item Visual Studio Code
	      	      	      
	      \begin{figure}[h!]
	      	\centering
	      	\includegraphics[width=0.5\linewidth]{vscode.png}
	      	\caption{VS Code}
	      	\label{fig:VSCode}
	      \end{figure}
	      	      	      
	      Visual Studio Code is a code editor redefined and optimized for building
	      and debugging modern web and cloud applications. Developed by Microsoft,
	      it is a free and open-source editor that supports a wide range of
	      programming languages, including JavaScript, Python, and C++. With
	      features like extensive extensibility through a marketplace of plugins, built-in
	      debugging tools, and seamless integration with version control systems
	      like Git, VS Code provides a user-friendly interface that enhances productivity.
	      Additionally, it is available on multiple platforms, including Windows, macOS,
	      and Linux, making it accessible to developers regardless of their operating
	      system. Visual Studio Code was chosen for this project because it is the
	      most commonly used code editor, offering familiarity and reliability for efficient
	      development
	      	      
	      \vspace{5cm}
	      	      	      
	\item GoLand
	      	      	      
	      \begin{figure}[h!]
	      	\centering
	      	\includegraphics[width=0.5\linewidth]{GoLand.png}
	      	\caption{GoLand}
	      	\label{fig:GoLand}
	      \end{figure}
	      	      	      
	      GoLand is an integrated development environment (IDE) specifically
	      designed for the Go programming language, developed by JetBrains. It
	      offers smart code assistance with advanced code completion, navigation,
	      and refactoring tools, making it easier to write clean and efficient code.
	      GoLand features a powerful integrated debugger for setting breakpoints and
	      inspecting variables, as well as built-in support for unit testing and code
	      coverage analysis. Additionally, it integrates seamlessly with version control
	      systems like Git, allowing developers to manage their repositories
	      directly within the IDE. With a customizable interface and cross-platform compatibility
	      for Windows, macOS, and Linux, GoLand enhances the development experience,
	      enabling developers to write, test, and deploy Go applications more effectively.
	      GoLand was chosen for this project to facilitate the use of the Go
	      programming language, providing the necessary tools and environment for optimal
	      development.
	      	      	      
	\item LaTeX
	      	      	      
	      \begin{figure}[h!]
	      	\centering
	      	\includegraphics[width=0.5\linewidth]{LaTeX.png}
	      	\caption{LaTeX}
	      	\label{fig:LaTeX}
	      \end{figure}
	      	      	      
	      LaTeX is a high-quality typesetting system. It includes features designed for
	      the production of technical and scientific documentation. LaTeX is the de
	      facto standard for the communication and publication of scientific
	      documents.
	      	      
	      \vspace{5cm}
	      	      	      
	\item GitHub
	      	      	      
	      \begin{figure}[h!]
	      	\centering
	      	\includegraphics[width=0.5\linewidth]{GitHub.png}
	      	\caption{GitHub}
	      	\label{fig:GitHub}
	      \end{figure}
	      	      	      
	      GitHub is a web-based platform that uses Git version control for managing
	      and sharing code repositories. It enables developers to collaborate on projects
	      by allowing them to track changes, manage branches, and resolve conflicts seamlessly.
	      With features like pull requests, code reviews, and issue tracking, GitHub
	      facilitates efficient team collaboration and project management.
	      Additionally, it hosts a vast repository of open-source projects,
	      providing developers with resources to learn from and contribute to. GitHub's
	      integration with various CI/CD tools and support for GitHub Actions enhances
	      its capabilities, making it an essential tool for modern software
	      development.
	      	      	      
	\item Notion
	      	      	      
	      \begin{figure}[h!]
	      	\centering
	      	\includegraphics[width=0.5\linewidth]{notion.png}
	      	\caption{Notion}
	      	\label{fig:Notion}
	      \end{figure}
	      	      	      
	      Notion is an all-in-one workspace that combines note-taking, task management,
	      databases, and collaboration tools, allowing teams to organize and share
	      information effectively. With its flexible structure, users can create
	      custom templates and pages tailored to their specific needs, promoting productivity
	      and collaboration. Notion's rich formatting options, including tables,
	      kanban boards, and calendars, enable users to visualize and manage their
	      work dynamically. Additionally, its real-time collaboration features allow
	      multiple users to edit and comment simultaneously, making it a powerful tool
	      for project management and team communication.
	      	      
	      \vspace{5cm}
	      	      	      
	\item macOS
	      	      	              
	      \begin{figure}[h!]
	      	\centering
	      	\includegraphics[width=0.5\linewidth]{mac-os.png}
	      	\caption{macOS}
	      	\label{fig:macOS}
	      \end{figure}
	      	      	      
	      macOS is a widely used operating system for software development, known for
	      its user-friendly interface and exceptional versatility. It equips
	      developers with essential tools and integrated development environments (IDEs)
	      for creating a variety of applications, including web, desktop, mobile,
	      and gaming software. The platform supports multiple programming languages and
	      frameworks, offering the flexibility to adapt to specific project
	      requirements. Its intuitive design simplifies the setup of development environments
	      and project management. Additionally, an active macOS developer community fosters
	      collaboration and knowledge sharing. With continuous updates, developers have
	      access to the latest technologies and tools, enabling them to modernize
	      their applications effectively. Overall, macOS is recognized as a crucial
	      platform for software development, playing a significant role in turning innovative
	      ideas into reality.
\end{enumerate}

\subsection{Computer resources}

\begin{table}[h!]
	\caption{Computer Resources}
	\def\arraystretch{1.4} \small
	\begin{tabular}{|p{1.8cm}|p{2.7cm}|p{3.1cm}|}
		\hline
		Name              & Computer \par Resource                 & Version of OS, SW    \\
		\hline
		Dongwook \par Kim & Apple M3 Pro Chip \par 18GB RAM memory & macOS Sequoia 15.0.1 \\
		\hline
		Jisu Shin         & Apple M1 Chip \par 16GB RAM memory     & macOS Sequoia 15.0.1 \\
		\hline
		Giram park        & Apple M2 Chip \par 16GB RAM memory     & macOS Sequoia 15.0.1 \\
		\hline
		Seoyoon Jung      & Apple M2 Chip \par 8GB RAM memory      & macOS Sequoia 15.0.1 \\
		\hline
	\end{tabular}
\end{table}

\par

\subsection{Cost Estimation}

Although it is different from the actual application in the industry, we
envision a test network operating two Hyperledger Fabric peers and four orderers
on a single computer. For this configuration, we plan to use an AWS EC2 t2.medium
instance (2vCPU, 4GB RAM). This t2.medium instance is deemed suitable for a
test network of this scale, as it meets the minimum specifications required for
running peers and orderer while providing a cost-effective option for development
and testing purposes. According to the AWS pricing calculator, operating a t2.medium
instance in the Seoul region with a long-term commitment would cost approximately
\$18.47 per month. Adding the cost of a required 50GB EBS volume at approximately
\$4.56, the total estimated monthly cost would be \$23.03. This represents the
minimum cost for establishing a test environment, and an actual production
environment would likely require higher-specification instances and additional
infrastructure configuration. However, if we can utilize the company's
existing underutilized server resources, we expect to significantly reduce these
cloud cost.

\subsection{Software in use}

\begin{enumerate}[itemsep=2ex, parsep=1ex]
	\item ADEPT
	      	      	      
	      ADEPT, which stands for Autonomous Decentralized Peer-to-Peer Telemetry,
	      is a blockchain platform designed for IoT devices that operates on a peer-to-peer
	      basis. The idea behind ADEPT was introduced in 2015 as a result of collaboration
	      between Samsung and IBM. It incorporates technologies like BitTorrent, Telehash,
	      and Ethereum. ADEPT organizes IoT devices based on their capacities,
	      allowing them to independently manage, analyze, and share their data. This
	      platform is being implemented in wearable technology and household appliances.
	      For instance, Samsung's smart washing machine employs ADEPT technology to automatically
	      order essential supplies, such as detergent, when they are running low.
	      	      	      
	\item IoT Chain
	      	      	      
	      IoT Chain operates on blockchain technology and incorporates various
	      mechanisms like PBFT (Practical Byzantine Fault Tolerance), DAG (Directed
	      Acyclic Graph), SPV (Simple Payment Verification), and CPS (Cyber Physical
	      System). Its primary goal is to improve security within the IoT ecosystem
	      while utilizing ICT (IoT Chain Token) for accessing IoT products. By leveraging
	      the decentralized security of conventional blockchains, IoT Chain
	      overcomes challenges related to transaction speed and scalability through
	      PBFT and DAG. The architecture consists of a main chain and a side chain;
	      the side chain executes smart contracts using coins generated from the
	      main chain. The main chain employs PBFT for rapid transaction validation,
	      while the side chain utilizes DAG for efficient transaction processing. SPV
	      allows payment verification by checking only the headers of blocks, rather
	      than all their components, which reduces verification fees and decreases user
	      overhead. IoT Chain finds applications in shared economies and smart home
	      technologies. In November 2018, initiatives were launched to create a
	      developer ecosystem, with plans to publicly release IoT Chain in December.
	      	      	      
	\item SLOCK.IT
	      	      	      
	      SLOCK.IT, a startup based in Germany, focuses on creating a sharing
	      economy infrastructure utilizing Ethereum technology. They are in the process
	      of developing the Universal Sharing Network, which integrates an automated
	      payment system with Ethereum. This platform allows individuals to share and
	      trade unused resources like homes or cars via blockchain technology,
	      ensuring trust between parties. SLOCK.IT provides a smart lock feature,
	      allowing users to unlock their assets for others by paying with tokens to
	      execute Ethereum smart contracts. Additionally, users can control the keys
	      required for transactions through a mobile application.
	      	      	      
	\item JD.COM
	      	      	      
	      JD.com offers blockchain gateway services, blockchain node services, and
	      blockchain consensus network services. The platform utilizes a BFT-like consensus
	      algorithm and employs an authentication protocol to manage the number of accesses
	      to the blockchain network. The system consists of three types of peers: consensus
	      peers, gateway peers, and IoT devices. Gateway peers operate within the
	      middleware layer to integrate inputs and protocols from the lower layers.
\end{enumerate}

\subsection{Task distribution }

\begin{table}[h!]
	\caption{Task distribution}
	\def\arraystretch{1.4} \small
	\begin{tabular}{|p{3cm}|p{4.6cm}|}
		\hline
		Name         & Task                   \\
		\hline
		Dongwook Kim & Blockchain Development \\
		\hline
		Jisu Shin    & Front-end Development  \\
		\hline
		Giram Park   & Embedded Development   \\
		\hline
		Seoyoon Jung & AI Development         \\
		\hline
	\end{tabular}
\end{table}

\section{Specifications}

\subsection{Core Requirements Specifications}

\begin{enumerate}[itemsep=2ex, parsep=1ex]
	\item Hub Elimination and Range Extension
	      	      
	      \begin{figure} [h!]
	      	\centering
	      	\includegraphics[width=0.8\linewidth]{blockchain_queue.png}
	      	\caption{Blockchain Queue}
	      	\label{fig:blockchain_queue}
	      \end{figure}
	      	      	      
	      To eliminate the mandatory use of Matter Hubs and extend operational range,
	      we propose replacing traditional Matter hubs and cloud services with Hyperledger          Fabric's chaincode functionality. This transformation fundamentally changes how Matter
	      devices communicate and operate, freeing them from the physical constraints
	      of home networks.
	      	      	      
	      The core of this solution lies in implementing a message queue system within
	      the blockchain. Instead of relying on a physical hub for communication, each
	      Matter device interacts with a dedicated queue in the blockchain. This
	      queue serves as a virtual communication channel, enabling devices to operate
	      beyond the traditional home network boundaries.
	      	      	      
	      \begin{figure}[h!]
	      	\centering
	      	\includegraphics[width=0.8\linewidth]{KeyValueBasedBlockchainQueue.png}
	      	\caption{Key\-Value based Blockchain Queue}
	      	\label{fig:KeyValueBasedBlockchainQueue}
	      \end{figure}
	      	      	      
	      However, implementing a traditional queue structure in Hyperledger Fabric
	      would be inefficient due to the complexity of transaction operations. Reading
	      and writing to an array-based queue would require reading the entire array
	      for each push operation, creating unnecessary overhead. To optimize this process,
	      we propose using a Key-Value Store structure where the key is formatted as
	      \emph{devicePK-index} (device public key combined with a index) and
	      the value contains the message payload.
	      	      	      
	      This architecture provides several advantages:
	      \begin{enumerate}[itemsep=2ex, parsep=1ex]
	      	\item Eliminates the need for physical Matter hubs by virtualizing their
	      	      functionality through blockchain
	      	      	      	      	      	      
	      	\item Enables device operation from any location with internet connectivity
	      	      	      	      	      	      
	      	\item Maintains secure and reliable communication through blockchain's inherent
	      	      security features
	      	      	      	      	      	      
	      	\item Optimizes performance through efficient key-value based message handling
	      \end{enumerate}
	      	      	      
	\item Protocol Flexibility and QR Code Innovation
	      	      	      
	      To support diverse device functionalities and provide greater flexibility, we innovate the protocol and QR code structure. Instead of relying on predefined application clusters in Matter protocol, our solution implements function definitions within QR codes, enabling dynamic functionality support.
	      	      	      
	      \begin{figure}[h!]
	      	\centering
	      	\includegraphics[width=0.9\linewidth]{MatterTunnelQRCode.png}
	      	\caption{Matter Tunnel QR Code}
	      	\label{fig:MatterTunnelQRCode}
	      \end{figure}
	      	      	      
	      The QR code structure consists of three main components: a 33-byte public key (PK), a 16-byte passcode, and function definitions. Each function definition comprises an 18-byte function name and 2 bytes representing parameter and return value types. Matter Tunnel supports four basic types - void, string, number, and boolean - each represented by 2 bits. This efficient encoding allows for up to seven parameters and one return value within the 2-byte type definition.
	      	      	      
	      \begin{figure}[h!]
	      	\centering
	      	\includegraphics[width=1\linewidth]{MatterTunnelTX.png}
	      	\caption{Matter Tunnel TX Format}
	      	\label{fig:MatterTunnelTXFormat}
	      \end{figure}
	      	      	      
	      The Matter Tunnel transaction format similarly reflects this flexibility, consisting of a 64-byte signature, 18-byte function name, 33-byte public key, 8-byte timestamp, and encrypted arguments. By including function names directly in the transaction format rather than using opcodes, the solution simplifies data analysis while maintaining extensibility. This approach effectively eliminates the constraints of Matter's predefined device types, allowing manufacturers to implement custom functionalities while ensuring seamless integration with the Matter ecosystem.
	      	      	                  
	\item Data Analytics and System Reliability
	      	      	      
	      Our AI Dashboard revolutionizes how organizations interact with blockchain data by making complex data analysis accessible to everyone, regardless of their technical background. At its core, the system transforms everyday language questions into precise blockchain queries, allowing business managers, analysts, and decision-makers to extract valuable insights without writing a single line of code.
	                
	      For example, a business manager can simply ask ``Show transactions from \{src\_pk\} to \{pk\}" and the system automatically converts this natural language question into appropriate technical queries, handling all the complex filtering and data retrieval operations behind the scenes. This breakthrough eliminates the traditional requirement for programming expertise or specialized blockchain knowledge, democratizing access to critical data.
	      
	      The system achieves this through a sophisticated AI model trained on 5000 carefully crafted examples of natural language questions and their corresponding technical queries. To ensure the system understands a wide variety of ways people might ask questions, we implemented advanced language processing techniques that help the AI recognize different phrasings of the same question.
	      
	      This direct interaction approach revolutionizes data analytics by eliminating multiple intermediaries present in traditional systems. Conventional approaches require data analysts to process raw data and relay servers to transmit processed information, creating multiple points where data could be corrupted or manipulated by malicious actors. Each intermediary step not only introduces potential security vulnerabilities but also raises questions about data authenticity.
	      
	      By connecting directly to the blockchain through secure gRPC connections, our AI Dashboard completely eliminates these risks. Instead of trusting multiple intermediaries, users receive data straight from the blockchain itself, ensuring complete data authenticity. The system combines this direct access with natural language processing, allowing decision-makers to query blockchain data through simple conversations while maintaining the highest level of data integrity. Whether you're requesting transaction histories, analyzing device patterns, or monitoring system status, you can trust that the data you're viewing is exactly what's recorded on the blockchain - unaltered and authentic.
	      
	      	      	      
	\item Decentralization
	      	      	      
	      Our approach to decentralization fundamentally reimagines the Matter
	      protocol's architecture while maintaining its core benefits. Unlike the traditional
	      Matter protocol that relies on a centralized Certificate Authority (CA) for
	      device certification, Matter Tunnel eliminates this requirement while
	      preserving protocol compatibility. By removing the centralized authentication
	      system, we lower barriers to entry for device manufacturers and smaller
	      development teams, fostering innovation and competition in the IoT ecosystem.
	      	      	      
	      This decentralized approach maintains the Matter protocol's ability to control
	      multiple vendors' devices through a single application, ensuring that the key
	      benefit of interoperability remains intact. The elimination of platform
	      dependencies further enhances true decentralization, freeing users from vendor
	      lock-in and creating a more open IoT environment.
	      	      
	      \vspace{3cm}
	      	      	      
	\item Enhanced Security and Privacy Protection
	      	      	      
	      \begin{figure}[h!]
	      	\centering
	      	\includegraphics[width=1\linewidth]{TraditionalMatter.png}
	      	\caption{Traditional Matter}
	      	\label{fig:TraditionalMatter}
	      \end{figure}
	      	      	      
	      \begin{figure}[h!]
	      	\centering
	      	\includegraphics[width=0.8\linewidth]{Matter Tunnel E2EE.png}
	      	\caption{Matter Tunnel E2EE}
	      	\label{fig:MatterTunnelE2EE}
	      \end{figure}
	      	      	      
	      Our solution significantly enhances End-to-End Encryption (E2EE) and user
	      privacy protection by fundamentally restructuring the communication
	      architecture of Matter devices. Traditional Matter implementations rely on
	      a cloud-based communication model where applications communicate with cloud
	      services, and Matter hubs poll these services for updates. This structure inherently
	      compromises both E2EE and privacy. Matter Tunnel addresses these
	      limitations through a blockchain-based approach.
	      	      	      
	      Key Security and Privacy Enhancements:
	      	      	      
	      \begin{enumerate}[itemsep=2ex, parsep=1ex]
	      	\item Direct Blockchain Communication
	      	      	      	      	      	      
	      	      By eliminating cloud service intermediaries, this system enables direct
	      	      and encrypted communication between applications and devices. The
	      	      removal of vulnerable points in the communication chain enhances
	      	      security, while ensuring true end-to-end encryption throughout the entire
	      	      process.
	      	      	      	      	      	      
	      	\item Enhanced Privacy Protection
	      	      	      	      	      	      
	      	      This system maintains privacy by only exposing device public keys on
	      	      the blockchain, while keeping sensitive information like IP addresses and
	      	      location data completely private. This approach enables anonymous
	      	      device operation, significantly reducing the attack surface for potential
	      	      privacy breaches.
	      	      	      	      	      	      
	      	\item Secure Device Registration and Authentication
	      	      	      	      	      	                      
	      	      The system implements a robust security mechanism through a predefined `register' 
	      	      function on the blockchain where users must verify device ownership by providing 
	      	      the correct 128-bit passcode. This establishes a secure transaction relationship 
	      	      between the user and device. With the passcode's substantial bit length, the 
	      	      probability of an adversary successfully guessing it is negligible 
	      	      $\left(\frac{1}{2}\right)^{128}$, effectively preventing unauthorized access at the fundamental level. 
	      	      This approach creates an inherently secure system that prevents attacks 
	      	      by design rather than through reactive measures.
	      \end{enumerate}
	      	      	      
	      By leveraging blockchain technology and existing Matter security features,
	      our solution creates a more robust and private IoT ecosystem. The combination
	      of anonymous operation, secure message counting, and direct blockchain communication
	      ensures that both E2EE and user privacy are maintained at the highest possible
	      level.
	      	      	      
	\item Real-Time Performance with Hyperledger Fabric
	      	      	      
	      To ensure real-time performance in Matter Tunnel, we carefully selected Hyperledger
	      Fabric as our blockchain platform after extensive evaluation of various options.
	      This choice was driven by Hyperledger Fabric's unique characteristics that
	      align with the real-time requirements of IoT device communication.
	      	      	      
	      Hyperledger Fabric's high transaction processing capability stands out as
	      a crucial feature for our implementation, supporting 2,000-20,000 transactions
	      per second (TPS). This throughput is achieved through multiple channels
	      that enable parallel transaction processing, effectively handling concurrent
	      device communications while providing near-instantaneous transaction finality.
	      The platform's configurable architecture can enhance performance by
	      allowing optimization of network parameters, enabling adjustment of block creation
	      time, supporting custom channel configurations for different device groups,
	      and permitting fine-tuning of consensus mechanisms to match specific use case
	      requirements.
	      	      	      
	      By leveraging Hyperledger Fabric's comprehensive capabilities, Matter
	      Tunnel achieves the real-time performance necessary for effective IoT
	      device control and monitoring. The platform's ability to handle high transaction
	      volumes while maintaining low latency ensures that device interactions remain
	      responsive and reliable, meeting the demanding requirements of modern IoT
	      applications.
\end{enumerate}

\begin{figure}[h!]
	\centering
	\includegraphics[width=1\linewidth]{comunicationOverview.png}
	\caption{Matter Tunnel Communication Overview}
	\label{fig:MatterTunnelCommunicationOverview}
\end{figure}

\subsection{Development Requirements Specifications}
\vspace{0.5cm}
{\centering \textbf{Client} \par}
\vspace{0.5cm}

\begin{enumerate}[itemsep=2ex, parsep=1ex]
	\item Entry \& Tutorial
	      
	      \begin{figure}[h!]
	      	\centering
	      	\includegraphics[width=0.5\linewidth]{MatterTunnelEntry.png}
	      	\caption{Entry}
	      	\label{fig:MatterTunnelEntry}
	      \end{figure}
          
	      \begin{table}[h!]
	      	\def\arraystretch{1.24} \small
	      	\begin{tabular}{|p{1.2cm}|p{2.5cm}|p{4.0cm}|}
	      		\hline
	      		ID  & Name               & Description                                                                                                                                                                                                                                                  \\
	      		\hline
	      		001 & MatterTunnel-Entry & When the application is launched, this component displays the initial entry page with the Matter Tunnel logo. It serves as the first touchpoint for users and should maintain visibility until the application completes loading its core data and services. \\
	      		\hline
	      	\end{tabular}
	      \end{table}
	      
	      \begin{figure}[h!]
	      	\centering
	      	\includegraphics[width=1\linewidth]{MatterTunnelTutorial.png}
	      	\caption{Tutorials}
	      	\label{fig:MatterTunnelTutorials}
	      \end{figure}
	                
	      \begin{table}[h!]
	      	\def\arraystretch{1.24} \small
	      	\begin{tabular}{|p{1.2cm}|p{2.5cm}|p{4.0cm}|}
	      		\hline
	      		ID  & Name                  & Description                                                                                                                                                                                                                                                                                                                                                              \\
	      		\hline
	      		002 & MatterTunnel-Tutorial & A comprehensive tutorial screen that appears after the initial entry, remaining active until the user completes all tutorial steps or chooses to skip. The component guides users through multiple slides introducing Matter Tunnel's core functionalities and features, helping new users understand the platform's key advantages before they begin using the service. \\
	      		\hline
	      	\end{tabular}
	      \end{table}
	      
            \clearpage
          
	      \begin{figure}[h!]
	      	\centering
	      	\includegraphics[width=1\linewidth]{MatterTunnelTutorialButtons.png}
	      	\caption{Tutorial Buttons}
	      	\label{fig:MatterTunnelTutorialButtons}
	      \end{figure}
	      	          
	      \begin{table}[h!]
	      	\def\arraystretch{1.24} \small
	      	\begin{tabular}{|p{1.2cm}|p{2.5cm}|p{4.0cm}|}
	      		\hline
	      		ID  & Name                                  & Description                                                                                                                                                                                                                                         \\
	      		\hline
	      		003 & MatterTunnel-Tutorial-Skip            & A component that allows users to bypass the entire tutorial sequence and directly proceed to the login page. This feature is designed for experienced users who are already familiar with the system and wish to access their accounts immediately. \\
	      		\hline
	      		004 & MatterTunnel-Tutorial-Next            & A navigation component that enables users to progress through multiple tutorial pages sequentially. It maintains its functionality until the user reaches the final tutorial page, guiding them through each step of the introduction process.      \\
	      		\hline
	      		005 & MatterTunnel-Tutorial-NavigateToLogin & A component that appears on the final tutorial page, directing users to transition from the tutorial completion to the login page. This represents the end of the tutorial flow and the beginning of the actual application usage.                  \\
	      		\hline
	      	\end{tabular}
	      \end{table} 
    
	\item Authentication
	      \begin{enumerate}[itemsep=2ex, parsep=1ex]
	      	          
	      	\item Login
            
	      	      \begin{table}[h!]
	      	      	\def\arraystretch{1.24} \small
	      	      	\begin{tabular}{|p{1.2cm}|p{2.5cm}|p{4.0cm}|}
	      	      		\hline
	      	      		ID  & Name                    & Description                                                                                                                                                                                                                                                                                                                                                                                                                              \\
	      	      		\hline
	      	      		006 & MatterTunnel-Login-Page & The authentication page that allows users to access the Matter Tunnel service. This component contains two primary functions: a login form for existing users to enter their credentials and a sign-up option for new users who want to create an account. The page features a clean, minimalist design with a text input field and a primary action button for login, along with a secondary option to navigate to the sign-up process. \\
	      	      		\hline
	      	      	\end{tabular}
	      	      \end{table}

                  \begin{figure}[h!]
	      		\centering
	      		\includegraphics[width=0.75\linewidth]{MatterTunnelLogin.png}
	      		\caption{Login Page}
	      		\label{fig:enter-label}
	      	\end{figure}
	      	      
	      	      \begin{figure}[h!]
	      	      	\centering
	      	      	\includegraphics[width=0.75\linewidth]{MatterTunnelLoginError.png}
	      	      	\caption{Login Error}
	      	      	\label{fig:MatterTunnelLoginError}
	      	      \end{figure}
	      	      	      	      
	      	      \begin{table}[h!]
	      	      	\def\arraystretch{1.24} \small
	      	      	\begin{tabular}{|p{1.2cm}|p{2.5cm}|p{4.0cm}|}
	      	      		\hline
	      	      		ID  & Name                     & Description                                                                                                                                                                                                                                                                                                                \\
	      	      		\hline
	      	      		007 & MatterTunnel-Login-Error & The system must validate user input during login. Both when the input field is empty and when the key does not match the required format (exactly 64 hexadecimal characters using 0-9 and A-F, pattern: \verb|/^[0-9a-fA-F]{64}$/|), display the error message ``Your key is incorrectly formatted" below the input field. \\ \\
	      	      		\hline
	      	      	\end{tabular}
	      	      \end{table}
	      	      	      	      
	      	      \clearpage
	      	      	      	      
	      	\item SignUp
	      	      
	      	      \begin{figure}[h!]
	      	      	\centering
	      	      	\includegraphics[width=0.75\linewidth]{MatterTunnelSignUpInitial.png}
	      	      	\caption{SignUp Initial}
	      	      	\label{fig:MatterTunnelSignUpInitial}
	      	      \end{figure}
	      	                  
	      	      \begin{table}[h!]
	      	      	\def\arraystretch{1.24} \small
	      	      	\begin{tabular}{|p{1.2cm}|p{2.5cm}|p{4.0cm}|}
	      	      		\hline
	      	      		ID  & Name                        & Description                                                                                                                                                                                                                                                                                                                                                                    \\
	      	      		\hline
	      	      		008 & MatterTunnel-SignUp-Initial & Upon accessing the Sign-Up box under the Login section, users are presented with an initial prompt encouraging them to register if they are not yet members. The ``Sign Up" button is displayed, inviting users to start the sign-up process. This component serves to inform users that they can proceed to registration to enjoy the features of the Matter Tunnel platform. \\
	      	      		\hline
	      	      	\end{tabular}
	      	      \end{table}
	      	      
	      	      \begin{figure}[h!]
	      	      	\centering
	      	      	\includegraphics[width=0.75\linewidth]{MatterTunnelSignUpPrivateKey.png}
	      	      	\caption{SignUp PrivateKey}
	      	      	\label{fig:MatterTunnelSignUpPrivateKey}
	      	      \end{figure}
	      	                        
	      	      \begin{table}[h!]
	      	      	\def\arraystretch{1.24} \small
	      	      	\begin{tabular}{|p{1.2cm}|p{2.5cm}|p{4.0cm}|}
	      	      		\hline
	      	      		ID  & Name                           & Description                                                                                                                                                                                                                                                                                                                           \\
	      	      		\hline
	      	      		009 & MatterTunnel-SignUp-PrivateKey & After clicking the ``Sign Up" button, users are directed to a screen where a private key is generated for them. This key is shown in a 64-character hexadecimal format, ensuring its precision and security. The private key is crucial for the user's future access to Matter Tunnel, and users are encouraged to store it securely. \\
	      	      		\hline
	      	      	\end{tabular}
	      	      \end{table}
	      	      
	      	      \begin{figure}[h!]
	      	      	\centering
	      	      	\includegraphics[width=0.8\linewidth]{MatterTunnelSignUpCompletion.png}
	      	      	\caption{SignUp Completion}
	      	      	\label{fig:MatterTunnelSignUpCompletion}
	      	      \end{figure}
	      	                        
	      	      \begin{table}[h!]
	      	      	\def\arraystretch{1.24} \small
	      	      	\begin{tabular}{|p{1.2cm}|p{2.5cm}|p{4.0cm}|}
	      	      		\hline
	      	      		ID  & Name                           & Description                                                                                                                                                                                                                                                                                                                                                               \\
	      	      		\hline
	      	      		010 & MatterTunnel-SignUp-Completion & Once the ``Sign Up" button is pressed, the private key is generated and displayed on the screen. Simultaneously, a ``Sign Up Completed" message appears, indicating that the registration process has been successfully completed. This stage reassures the user that they are now fully registered and ready to proceed to the next steps in the Matter Tunnel platform. \\
	      	      		\hline
	      	      	\end{tabular}
	      	      \end{table}
	      	      
	      	      \begin{figure}[h!]
	      	      	\centering
	      	      	\includegraphics[width=0.8\linewidth]{MatterTunnelSignUpCopy.png}
	      	      	\caption{Matter Tunnel SignUp Copy}
	      	      	\label{fig:MatterTunnelSignUpCopy}
	      	      \end{figure}
	      	      	      	      
	      	      \begin{table}[h!]
	      	      	\def\arraystretch{1.24} \small
	      	      	\begin{tabular}{|p{1.2cm}|p{2.5cm}|p{4.0cm}|}
	      	      		\hline
	      	      		ID  & Name                     & Description                                                                                                                                                                                                                                                                                                              \\
	      	      		\hline
	      	      		011 & MatterTunnel-SignUp-Copy & This component allows users to copy their generated private key by clicking the ``Copy" button. A confirmation message is displayed once the key is successfully copied to the clipboard, ensuring that users can easily save their credentials for future use and proceed with logging into the Matter Tunnel platform. \\
	      	      		\hline
	      	      	\end{tabular}
	      	      \end{table}
	      	      	      	      
	      	      \clearpage
	      	      	      	      
	      	\item Account Management
	      	                  
	      	      \begin{figure}[h!]
	      	      	\centering
	      	      	\includegraphics[width=0.75\linewidth]{MatterTunnelAccountPrivateKey.png}
	      	      	\caption{Account PrivateKey}
	      	      	\label{fig:MatterTunnelAccountPrivateKey}
	      	      \end{figure}
	      	                  
	      	      \begin{table}[h!]
	      	      	\def\arraystretch{1.24} \small
	      	      	\begin{tabular}{|p{1.2cm}|p{2.5cm}|p{4.0cm}|}
	      	      		\hline
	      	      		ID  & Name                            & Description                                                                                                                                                                                                                                              \\
	      	      		\hline
	      	      		012 & MatterTunnel-Account-PrivateKey & This component displays the user's private key. It is a critical part of the user's identity within Matter Tunnel, and users can view it to manage their credentials. The key is displayed securely, ensuring it is visible only to the authorized user. \\
	      	      		\hline
	      	      	\end{tabular}
	      	      \end{table}

                  \vspace{30cm}
	      	                        
	      	      \begin{figure}[h!]
	      	      	\centering
	      	      	\includegraphics[width=0.75\linewidth]{MatterTunnelAccountCopyButton.png}
	      	      	\caption{Account Copy Button}
	      	      	\label{fig:MatterTunnelAccountCopyButton}
	      	      \end{figure}
	      	      	      	      
	      	      \begin{table}[h!]
	      	      	\def\arraystretch{1.24} \small
	      	      	\begin{tabular}{|p{1.2cm}|p{2.5cm}|p{4.0cm}|}
	      	      		\hline
	      	      		ID  & Name                            & Description                                                                                                                                                                                                                                                                                                                                       \\
	      	      		\hline
	      	      		013 & MatterTunnel-Account-CopyButton & A ``Copy Private Key" button is provided below the private key. When users click the button, the private key is copied to their clipboard. After the action is completed, an alert message ``Private key copied" appears at the top of the screen, confirming that the key has been successfully copied and can be securely stored for later use. \\
	      	      		\hline
	      	      	\end{tabular}
	      	      \end{table}

                  \vspace{5cm}
	      	      
	      	      \begin{figure}[h!]
	      	      	\centering
	      	      	\includegraphics[width=0.75\linewidth]{MatterTunnelAccountLogoutButton.png}
	      	      	\caption{Account Logout Button}
	      	      	\label{fig:MatterTunnelAccountLogoutButton}
	      	      \end{figure}
	      	      	      	      
	      	      \begin{table}[h!]
	      	      	\def\arraystretch{1.24} \small
	      	      	\begin{tabular}{|p{1.2cm}|p{2.5cm}|p{4.0cm}|}
	      	      		\hline
	      	      		ID  & Name                              & Description                                                                                                                                                                                                                                                                                           \\
	      	      		\hline
	      	      		014 & MatterTunnel-Account-LogoutButton & The ``Logout" button allows users to exit their current session. When clicked, the system logs the user out and redirects them to the Login page. Upon successful logout, a ``Logout Completed" message appears below the ``Login" button, confirming that the user has been successfully logged out. \\
	      	      		\hline
	      	      	\end{tabular}
	      	      \end{table}
	      \end{enumerate}

            \clearpage
	      	      
	\item Main Interface
	      \begin{enumerate}[itemsep=2ex, parsep=1ex]
	      	\item Device list
	      	      
	      	      \begin{figure}[h!]
	      	      	\centering
	      	      	\includegraphics[width=1\linewidth]{MatterTunnelDeviceList.png}
	      	      	\caption{Device List}
	      	      	\label{fig:MatterTunnelDeviceList}
	      	      \end{figure}
	      	                  
	      	      \begin{table}[h!]
	      	      	\def\arraystretch{1.24} \small
	      	      	\begin{tabular}{|p{1.2cm}|p{2.5cm}|p{4.0cm}|}
	      	      		\hline
	      	      		ID  & Name                     & Description                                                                                                                                                                                                                                                                                                                                                                                                                                             \\
	      	      		\hline
	      	      		015 & MatterTunnel-Device-List & This component displays a list of devices registered by the user within the Matter Tunnel platform. If no devices are added, the screen will prompt the user to add a device in order to experience various services provided by Matter Tunnel. This list serves as the main interface for device management. The list is interactive, allowing users to view details of their devices and navigate to further actions like adding or deleting devices. \\
	      	      		\hline
	      	      	\end{tabular}
	      	      \end{table}

                  \vspace{7cm}
	      	      
	      	      \begin{figure}[h!]
	      	      	\centering
	      	      	\includegraphics[width=0.25\linewidth]{MatterTunnelNavigateToDeviceAdditionButton.png}
	      	      	\caption{Navigate To Device Addition Button}
	      	      	\label{fig:MatterTunnelNavigateToDeviceAdditionButton}
	      	      \end{figure}
	      	                        
	      	      \begin{table}[h!]
	      	      	\def\arraystretch{1.24} \small
	      	      	\begin{tabular}{|p{1.2cm}|p{2.5cm}|p{4.0cm}|}
	      	      		\hline
	      	      		ID  & Name                                             & Description                                                                                                                                                                                                                                                                                                                                             \\
	      	      		\hline
	      	      		016 & MatterTunnel-NavigateTo DeviceAddition Button(+) & This button is located on the Device List screen and allows users to add new devices to their Matter Tunnel account. When clicked, the user is redirected to the QR code scanning screen, where they can scan a QR code to register a new device. This action enhances the user experience by expanding the available devices within the Matter Tunnel. \\
	      	      		\hline
	      	      	\end{tabular}
	      	      \end{table}
	      	      
	      	      \begin{figure}[h!]
	      	      	\centering
	      	      	\includegraphics[width=0.25\linewidth]{MatterTunnelNavigateToDeviceDeletionButton.png}
	      	      	\caption{Navigate To Device Deletion Button}
	      	      	\label{fig:MatterTunnelNavigateToDeviceDeletionButton}
	      	      \end{figure}
	      	      	      	      
	      	      \begin{table}[h!]
	      	      	\def\arraystretch{1.24} \small
	      	      	\begin{tabular}{|p{1.2cm}|p{2.5cm}|p{4.0cm}|}
	      	      		\hline
	      	      		ID  & Name                                             & Description                                                                                                                                                                                                                                                                                                                                                                                     \\
	      	      		\hline
	      	      		017 & MatterTunnel-NavigateTo DeviceDeletion Button(-) & This button is located on the Device List screen, allowing users to delete one or more devices from their Matter Tunnel account. When clicked, the user can select devices to delete by using checkboxes displayed on the Device List. This action provides users with an easy and efficient way to manage their devices by removing unnecessary ones, helping maintain a clutter-free account. \\
	      	      		\hline
	      	      	\end{tabular}
	      	      \end{table}
	      	      	      	      
	      	      \vspace{10cm}
	      	      
	      	      \begin{figure}[h!]
	      	      	\centering
	      	      	\includegraphics[width=0.5\linewidth]{MatterTunnelDeviceComponent.png}
	      	      	\caption{Device Component}
	      	      	\label{fig:enter-label}
	      	      \end{figure}
	      	      	      	      
	      	      \begin{table}[h!]
	      	      	\def\arraystretch{1.24} \small
	      	      	\begin{tabular}{|p{1.2cm}|p{2.5cm}|p{4.0cm}|}
	      	      		\hline
	      	      		ID  & Name                                            & Description                                                                                                                                                                                                                                                                                                                                                                                                                                                    \\
	      	      		\hline
	      	      		018 & MatterTunnel-DeviceComponent                    & Each device in the list is represented by a Device Component. This component includes the device icon, device name, and the public key associated with the device, represented by a QR code. Additionally, there is a NavigateToDeviceControlButton, which allows users to navigate to the Device Control screen for managing the device. The device component provides an overview of each device and serves as an interactive element for device management. \\
	      	      		\hline
	      	      		019 & MatterTunnel-NavigateTo DeviceControl Button(:) & This button is embedded within the Device Component and allows users to navigate to the Device Control screen. By clicking this button, users can control and configure specific settings for the selected device. The button provides quick access to device management functions.                                                                                                                                                                            \\
	      	      		\hline
	      	      	\end{tabular}
	      	      \end{table}
                  
	      	      \begin{figure}[h!]
	      	      	\centering
	      	      	\includegraphics[width=1\linewidth]{MatterTunnelNavigationBar.png}
	      	      	\caption{Matter Tunnel Navigation Bar}
	      	      	\label{fig:enter-label}
	      	      \end{figure}

                  \vspace{10cm}
	      	      	      	      
	      	      \begin{table}[h!]
	      	      	\def\arraystretch{1.24} \small
	      	      	\begin{tabular}{|p{1.2cm}|p{2.5cm}|p{4.0cm}|}
	      	      		\hline
	      	      		ID  & Name                       & Description                                                                                                                                                                                                                                                                                                                                                                                                                                                                                                                                                                                      \\
	      	      		\hline
	      	      		020 & MatterTunnel-NavigationBar & The Navigation Bar is a persistent component at the bottom of the screen, providing users with easy access to different sections of the app. It includes buttons for navigating to the Device List and Account Management pages. Clicking the Home button redirects users back to the Device List screen, while the Account button takes them to the Account Management screen. This navigation ensures that users can seamlessly switch between managing their devices and adjusting their account settings. However, the Navigation Bar is not visible during the Login and Sign-Up processes. \\
	      	      		\hline
	      	      	\end{tabular}
	      	      \end{table}
	      	      	      	      
	      	\item Device addition
	      	      
	      	      \begin{figure}[h!]
	      	      	\centering
	      	      	\includegraphics[width=0.7\linewidth]{MatterTunnelDeviceAdditionScreen.png}
	      	      	\caption{Device Addition Screen}
	      	      	\label{fig:DeviceAdditionScreen}
	      	      \end{figure}

                  \clearpage
	      	      
	      	      \begin{figure}[h!]
	      	      	\centering
	      	      	\includegraphics[width=0.75\linewidth]{MatterTunnelDeviceAdditionScreenComplete.png}
	      	      	\caption{Device Addition Complete Screen}
	      	      	\label{fig:DeviceAdditionCompleteScreen}
	      	      \end{figure}
	      	      \begin{table}[h!]
	      	      	\def\arraystretch{1.24} \small
	      	      	\begin{tabular}{|p{1.2cm}|p{2.5cm}|p{4.0cm}|}
	      	      		\hline
	      	      		ID  & Name                              & Description                                                                                                                                                                                                                                                                                               \\
	      	      		\hline
	      	      		021 & MatterTunnel-DeviceAddition Screen & This component displays the Device Addition screen, which allows users to add a new device to their account. When the ``+" button on the Device List screen is clicked, the user is navigated to this screen, where they can scan a QR code and enter a name for the device to complete the registration. \\
	      	      		\hline
	      	      	\end{tabular}
	      	      \end{table}

                  \vspace{5cm}
	      	      
	      	      \begin{figure}[h!]
	      	      	\centering
	      	      	\includegraphics[width=0.7\linewidth]{MatterTunnelDeviceQRScanScreen.png}
	      	      	\caption{Device QR Scan Screen}
	      	      	\label{fig:DeviceQRScanScreen}
	      	      \end{figure}
	      	      
	      	      \begin{figure}[h!]
	      	      	\centering
	      	      	\includegraphics[width=0.8\linewidth]{MatterTunnelDeviceQRScanScreenError.png}
	      	      	\caption{Device QR Scan Screen Error}
	      	      	\label{fig:MatterTunnelDeviceQRScanScreenError}
	      	      \end{figure}
	      	                        
	      	      \begin{table}[h!]
	      	      	\def\arraystretch{1.24} \small
	      	      	\begin{tabular}{|p{1.2cm}|p{2.5cm}|p{4.0cm}|}
	      	      		\hline
	      	      		ID  & Name                            & Description                                                                                                                                                                                                                                                                                                                                                                                                                                                             \\
	      	      		\hline
	      	      		022 & MatterTunnel-DeviceQRScan Screen & This screen is displayed for users to scan a QR code for adding a device. Once the QR code is successfully scanned, the screen changes to black, and a message “QR Code Scan Complete!” appears. “Scan again” button is also available at the bottom of the screen, allowing users to scan a new QR code if needed. If the QR code has already been scanned and the device is already registered, an alert message appears saying ``Already Registered Device!" \\
	      	      		\hline
	      	      	\end{tabular}
	      	      \end{table}

                  \clearpage
	      	      
	      	      \begin{figure}[h!]
	      	      	\centering
	      	      	\includegraphics[width=0.75\linewidth]{MatterTunnelDeviceNameInput.png}
	      	      	\caption{Device Name Input}
	      	      	\label{fig:MatterTunnelDeviceNameInput}
	      	      \end{figure}
	      	      	      	      
	      	      \begin{table}[h!]
	      	      	\def\arraystretch{1.24} \small
	      	      	\begin{tabular}{|p{1.2cm}|p{2.5cm}|p{4.0cm}|}
	      	      		\hline
	      	      		ID  & Name                         & Description                                                                                                                                                                                                                                                           \\
	      	      		\hline
	      	      		023 & MatterTunnel-DeviceNameInput & After successfully scanning a QR code, this input field allows users to name the device. The input field becomes active, enabling users to assign a unique name to the device they are registering. Once the name is entered, the ``Register" button will be enabled. \\
	      	      		\hline
	      	      	\end{tabular}
	      	      \end{table}
	      	      
	      	      \begin{figure}[h!]
	      	      	\centering
	      	      	\includegraphics[width=0.75\linewidth]{MatterTunnelDeviceRegisterButton.png}
	      	      	\caption{Device Register Button}
	      	      	\label{fig:MatterTunnelDeviceRegisterButton}
	      	      \end{figure}
	      	      
	      	      \begin{figure}[h!]
	      	      	\centering
	      	      	\includegraphics[width=0.8\linewidth]{MatterTunnelDeviceRegisterButtonAlert.png}
	      	      	\caption{Device Register Alert}
	      	      	\label{fig:DeviceRegisterAlert}
	      	      \end{figure}
	      	      	      	      
	      	      \begin{table}[h!]
	      	      	\def\arraystretch{1.24} \small
	      	      	\begin{tabular}{|p{1.2cm}|p{2.5cm}|p{4.0cm}|}
	      	      		\hline
	      	      		ID  & Name                               & Description                                                                                                                                                                                                                                                  \\
	      	      		\hline
	      	      		024 & MatterTunnel-DeviceRegister Button & Once a device name is entered, the ``Register" button becomes active. Clicking this button triggers the addition of the device to the system. When the device is successfully registered, an alert message appears saying ``Device Successfully Registered." \\
	      	      		\hline
	      	      	\end{tabular}
	      	      \end{table}
	      	      
	      	      \begin{figure}[h!]
	      	      	\centering
	      	      	\includegraphics[width=0.25\linewidth]{MatterTunnelDeviceAdditionCancelButton.png}
	      	      	\caption{Device Addition Cancel Button}
	      	      	\label{fig:DeviceAdditionCancelButton}
	      	      \end{figure}

                  \vspace{5cm}
	      	      	      	      
	      	      \begin{table}[h!]
	      	      	\def\arraystretch{1.24} \small
	      	      	\begin{tabular}{|p{1.2cm}|p{2.5cm}|p{4.0cm}|}
	      	      		\hline
	      	      		ID  & Name                                        & Description                                                                                                                                                                                                  \\
	      	      		\hline
	      	      		025 & MatterTunnel-DeviceAddition CancelButton(X) & The ``Cancel" button allows users to exit the Device Addition screen without adding a new device. Clicking this button returns users to the Device List screen, canceling the addition process at any point. \\
	      	      		\hline
	      	      	\end{tabular}
	      	      \end{table}
	      	      	      	              
	      	\item Device deletion
	      	      \begin{figure}[h!]
	      	      	\centering
	      	      	\includegraphics[width=0.5\linewidth]{MatterTunnelDeviceDeletionScreen.png}
	      	      	\caption{Device Deletion Screen}
	      	      	\label{fig:DeviceDeletionScreen}
	      	      \end{figure}
	      	      	      	      
	      	      \begin{table}[h!]
	      	      	\def\arraystretch{1.24} \small
	      	      	\begin{tabular}{|p{1.2cm}|p{2.5cm}|p{4.0cm}|}
	      	      		\hline
	      	      		ID  & Name                              & Description                                                                                                                                                                                                                                                                                                                                                                                                                                                                                                                                                                                        \\
	      	      		\hline
	      	      		026 & MatterTunnel-DeviceDeletionScreen & This screen is displayed when the user clicks the ``-" button to initiate the device deletion process. The checkboxes appear at the top-right corner of each device component, and the ``Cancel" and ``Delete" buttons replace the previous button. The user can select one or more devices for deletion. After selecting the devices, the ``Delete" button becomes active. The user can confirm the deletion, and a confirmation message is shown once the devices are successfully deleted. If the user chooses to cancel, the selection is cleared, and no changes are made to the device list. \\
	      	      		\hline
	      	      	\end{tabular}
	      	      \end{table}
	      	      
	      	      \begin{figure}[h!]
	      	      	\centering
	      	      	\includegraphics[width=1\linewidth]{MatterTunnelDeviceDeletionCheckbox.png}
	      	      	\caption{Device Deletion Checkbox}
	      	      	\label{fig:DeviceDeletionCheckbox}
	      	      \end{figure}
	      	      	      	      
	      	      \begin{table}[h!]
	      	      	\def\arraystretch{1.24} \small
	      	      	\begin{tabular}{|p{1.2cm}|p{2.5cm}|p{4.0cm}|}
	      	      		\hline
	      	      		ID  & Name                                & Description                                                                                                                                                                                                                                                                            \\
	      	      		\hline
	      	      		027 & MatterTunnel-DeviceDeletion Checkbox & This checkbox appears at the top-right corner of each device component on the Device Deletion screen. Users can select one or more devices for deletion by checking the corresponding checkboxes. The checkboxes become active once the device list is displayed in the deletion mode. \\
	      	      		\hline
	      	      	\end{tabular}
	      	      \end{table}

                  \vspace{5cm}
	      	      
	      	      \begin{figure}[h!]
	      	      	\centering
	      	      	\includegraphics[width=0.25\linewidth]{MatterTunnelDeviceDeletionCancelButton.png}
	      	      	\caption{Device Deletion Cancel Button}
	      	      	\label{fig:DeviceDeletionCancelButton}
	      	      \end{figure}
	      	      	      	      
	      	      \begin{table}[h!]
	      	      	\def\arraystretch{1.24} \small
	      	      	\begin{tabular}{|p{1.2cm}|p{2.5cm}|p{4.0cm}|}
	      	      		\hline
	      	      		ID  & Name                                     & Description                                                                                                                                                                                                                            \\
	      	      		\hline
	      	      		028 & MatterTunnel-DeviceDeletion CancelButton & This button allows users to cancel the device deletion process. It is displayed on the Device Deletion screen alongside the ``Delete" button. If clicked, the user's selection is cleared, and no changes are made to the device list. \\
	      	      		\hline
	      	      	\end{tabular}
	      	      \end{table}
	      	      
	      	      \begin{figure}[h!]
	      	      	\centering
	      	      	\includegraphics[width=0.25\linewidth]{MatterTunnelDeviceDeletionButton.png}
	      	      	\caption{Device Deletion Button}
	      	      	\label{fig:DeviceDeletionButton}
	      	      \end{figure}
	      	      
	      	      \begin{figure}[h!]
	      	      	\centering
	      	      	\includegraphics[width=0.8\linewidth]{MatterTunnelDeviceDeletionAlert.png}
	      	      	\caption{Device Deletion Alert}
	      	      	\label{fig:DeviceDeletionAlert}
	      	      \end{figure}

                   \vspace{5cm}
	      	      	      	      
	      	      \begin{table}[h!]
	      	      	\def\arraystretch{1.24} \small
	      	      	\begin{tabular}{|p{1.2cm}|p{2.5cm}|p{4.0cm}|}
	      	      		\hline
	      	      		ID  & Name                               & Description                                                                                                                                                                                                                            \\
	      	      		\hline
	      	      		029 & MatterTunnel-DeviceDeletio nButton & This button becomes active after the user selects one or more devices for deletion. When clicked, it initiates the deletion of the selected devices, and a confirmation message appears notifying the user of the successful deletion. \\
	      	      		\hline
	      	      	\end{tabular}
	      	      \end{table}
	      	      	      	      
	      	\item Device control
	      	      
	      	      \begin{figure}[h!]
	      	      	\centering
	      	      	\includegraphics[width=0.75\linewidth]{MatterTunnelDeviceControlScreen.png}
	      	      	\caption{Device Control Screen}
	      	      	\label{fig:DeviceControlScreen}
	      	      \end{figure}
                  
	      	      \begin{table}[h!]
	      	      	\def\arraystretch{1.24} \small
	      	      	\begin{tabular}{|p{0.5cm}|p{1.8cm}|p{5.4cm}|}
	      	      		\hline
	      	      		ID  & Name                              & Description                                                                                                                                                                                                                                                                                                                                                                                                                                                                                                                                                   \\
	      	      		\hline
	      	      		030 & MatterTunnel-DeviceControl Screen & This screen provides users with the option to control device functions within the Matter Tunnel system. It features two main components: Feedback and Device Functions. The Feedback section displays the feedback received. The Device Functions section lists the available actions. When a function is executed, a transaction is sent to the blockchain, and the corresponding feedback is displayed.\\
	      	      		\hline
	      	      	\end{tabular}
	      	      \end{table}

                  \clearpage
	      	      
	      	      \begin{figure}[h!]
	      	      	\centering
	      	      	\includegraphics[width=0.75\linewidth]{MatterTunnelDeviceFeedbackSection.png}
	      	      	\caption{Device Feedback Section 1}
	      	      	\label{fig:DeviceFeedbackSection1}
	      	      \end{figure}
	      	      
	      	      \begin{figure}[h!]
	      	      	\centering
	      	      	\includegraphics[width=0.75\linewidth]{MatterTunnelDeviceFeedbackSection2.png}
	      	      	\caption{Device Feedback Section 2}
	      	      	\label{fig:DeviceFeedbackSection2}
	      	      \end{figure}
	      	      	      	      
	      	      \begin{table}[h!]
	      	      	\def\arraystretch{1.24} \small
	      	      	\begin{tabular}{|p{1.2cm}|p{2.5cm}|p{4.0cm}|}
	      	      		\hline
	      	      		ID  & Name                                & Description                                                                                                                                                                                                                                                                                                                                                                                       \\
	      	      		\hline
	      	      		031 & MatterTunnel-DeviceFeedback Section & This section displays the feedback received from the server after executing a function. It includes details such as the device icon, device name, and the feedback content. While the function is being executed, the feedback section shows the message ``Processing Transaction...". Once the function is completed, the appropriate feedback is displayed based on the function’s execution. \\
	      	      		\hline
	      	      	\end{tabular}
	      	      \end{table}

                  \begin{figure}[h!]
	      	      	\centering
	      	      	\includegraphics[width=0.8\linewidth]{DeviceFunctionError1.png}
	      	      	\caption{Device Function Error 1}
	      	      	\label{fig:enter-label}
	      	      \end{figure}

                  \begin{figure}[h!]
	      	      	\centering
	      	      	\includegraphics[width=0.8\linewidth]{DeviceFunctionError2.png}
	      	      	\caption{Device Function Error 2}
	      	      	\label{fig:enter-label}
	      	      \end{figure}

                  \begin{figure}[h!]
	      	      	\centering
	      	      	\includegraphics[width=0.8\linewidth]{DeviceFunctionError3.png}
	      	      	\caption{Device Function Error 3}
	      	      	\label{fig:enter-label}
	      	      \end{figure}
                  
	      	      \begin{figure}[h!]
	      	      	\centering
	      	      	\includegraphics[width=0.75\linewidth]{MatterTunnelDeviceFunction.png}
	      	      	\caption{Device Function}
	      	      	\label{fig:DeviceFunction}
	      	      \end{figure}
	      	      	      	      
	      	      \begin{table}[h!]
	      	      	\def\arraystretch{1.24} \small
	      	      	\begin{tabular}{|p{1cm}|p{2.2cm}|p{4.5cm}|}
	      	      		\hline
	      	      		032 & MatterTunnel-DeviceFunction Name          & This component displays the name of the function retrieved from the device’s QR code. Each device has different functions, and this field shows the specific function available for that device. The function name serves as an indicator of what action will be executed once the user interacts with it.                                                                                           \\
	      	      		\hline
	      	      		033 & MatterTunnel-DeviceFunction Arguments     & This section shows the arguments required for the function, which may vary based on the function itself. If the function requires inputs, users can provide them here. If the required number of arguments is not entered, the message ``Please Enter All Parameters" will appear. If the arguments do not match the expected format, an error message ``Please Enter Valid Format" will be displayed. \\
	      	      		\hline
	      	      		034 & MatterTunnel-DeviceFunction ExecuteButton & This button allows users to execute the function after providing the necessary arguments. Once clicked, a transaction is sent to the server to perform the requested function, and feedback is shown in the Feedback section. The button is only enabled once the required inputs are correctly filled out.                                                                                            \\
	      	      		\hline
	      	      	\end{tabular}
	      	      \end{table}
	      \end{enumerate}
\end{enumerate}

\clearpage

{\centering \textbf{Dashboard} \par}

\begin{enumerate}[itemsep=2ex, parsep=1ex]
	\item Dashboard Screen

            \begin{figure}[h!]
                \centering
                \includegraphics[width=1\linewidth]{DashboardScreen.png}
                \caption{Dashboard Screen}
                \label{fig:DashboardScreen}
            \end{figure}
    
	      \begin{table}[h!]
	      	\def\arraystretch{1.24} \small
	      	\begin{tabular}{|p{1.2cm}|p{2.5cm}|p{4.0cm}|}
	      		\hline
	      		ID  & Name             & Description                                                                                                                                                                                                              \\
	      		\hline
	      		001 & Dashboard-Screen & The main full-screen interface of the dashboard, consisting of a navigation bar, four key components, and a refresh button. These elements allow users to interact with and monitor blockchain transactions effectively. \\
	      		\hline
	      	\end{tabular}
	      \end{table}
	      	      	      	      
	\item Dashboard Components
	      \begin{enumerate}[itemsep=2ex, parsep=1ex]
	      	\item Transactions Per Second Chart

\begin{figure}[h!]
    \centering
    \includegraphics[width=1\linewidth]{DashboardTransactionsPerSec.png}
    \caption{Transactions Per Sec}
    \label{fig:enter-label}
\end{figure}

\vspace{5cm}
            
	      	      \begin{table}[h!]
	      	      	\def\arraystretch{1.24} \small
	      	      	\begin{tabular}{|p{1.2cm}|p{2.5cm}|p{4.0cm}|}
	      	      		\hline
	      	      		ID  & Name                         & Description                                                                                                                                                                                                                                                                                                                                                                                 \\
	      	      		\hline
	      	      		002 & Dashboard-TransactionsPerSec & A real-time line chart that displays the number of blockchain transactions processed per second. The chart dynamically updates every second to show the latest transaction rates. When the user hovers the mouse over any point on the chart, the number of transactions processed at that specific time is displayed, offering insights into the transaction rate for the selected moment. \\
	      	      		\hline
	      	      	\end{tabular}
	      	      \end{table}
	      	      	      	      
	      	\item Transactions by PK Chart

            \begin{figure}[h!]
                \centering
                \includegraphics[width=1\linewidth]{DashboardTransactionsByPK.png}
                \caption{Transactions By PK}
                \label{fig:enter-label}
            \end{figure}
            
	      	      \begin{table}[h!]
	      	      	\def\arraystretch{1.24} \small
	      	      	\begin{tabular}{|p{1.2cm}|p{2.5cm}|p{4.0cm}|}
	      	      		\hline
	      	      		ID  & Name                       & Description                                                                                                                                                                                                                                   \\
	      	      		\hline
	      	      		003 & Dashboard-TransactionsByPK & A bar chart that visualizes the transaction distribution based on Public Keys. When hovering over each bar, the exact number of transactions for the corresponding PK is shown, allowing users to analyze the transaction count for each key. \\
	      	      		\hline
	      	      	\end{tabular}
	      	      \end{table}

                  \clearpage
	      	      	      	      
	      	\item Total Transactions and Users

                \begin{figure}[h!]
                    \centering
                    \includegraphics[width=1\linewidth]{Dashboard-Total.png}
                    \caption{Dashboard Total}
                    \label{fig:enter-label}
                \end{figure}
            
	      	      \begin{table}[h!]
	      	      	\def\arraystretch{1.24} \small
	      	      	\begin{tabular}{|p{1.2cm}|p{2.5cm}|p{4.0cm}|}
	      	      		\hline
	      	      		ID  & Name            & Description                                                                                                                                                                                                                                    \\
	      	      		\hline
	      	      		004 & Dashboard-Total & Displays a comprehensive summary of the total blockchain transactions and the overall number of registered users. This component provides a quick, easy-to-understand overview of key metrics related to transaction volume and user activity. \\
	      	      		\hline
	      	      	\end{tabular}
	      	      \end{table}
	      	      	      	      
	      	\item Transaction Management and Natural Language Processing

            \begin{figure}[h!]
                \centering
                \includegraphics[width=1\linewidth]{DashboardNLQueryTool.png}
                \caption{Dashboard NL Query Tool}
                \label{fig:enter-label}
            \end{figure}

            \vspace{5cm}
            
	      	      \begin{table}[h!]
	      	      	\def\arraystretch{1.24} \small
	      	      	\begin{tabular}{|p{1.2cm}|p{2.5cm}|p{4.0cm}|}
	      	      		\hline
	      	      		ID  & Name                  & Description                                                                                                                                                                                                                                                                                                                                                                                                             \\
	      	      		\hline
	      	      		005 & Dashboard-NLQueryTool & A tool that allows users to manage and query blockchain transaction details using natural language. Users can input queries such as ``Show all transactions for PK X in the last week", and results are displayed visually for easy interpretation. This tool enables the management of blockchain transactions across the IoT industry and allows users to query transaction data through natural language processing. \\
	      	      		\hline
	      	      	\end{tabular}
	      	      \end{table}
	      	      	      	      
	      	\item Refresh Button

            \begin{figure}[h!]
                \centering
                \includegraphics[width=0.15\linewidth]{RefreshButton.png}
                \caption{Refresh Button}
                \label{fig:enter-label}
            \end{figure}
            
	      	      \begin{table}[h!]
	      	      	\def\arraystretch{1.24} \small
	      	      	\begin{tabular}{|p{1.2cm}|p{2.5cm}|p{4.0cm}|}
	      	      		\hline
	      	      		ID  & Name                    & Description                                                                                                                                                                                                                  \\
	      	      		\hline
	      	      		006 & Dashboard-RefreshButton & A button designed to refresh all charts and data components on the dashboard. Upon clicking, all visualizations are updated to reflect the most recent blockchain information, ensuring users have access to real-time data. \\
	      	      		\hline
	      	      	\end{tabular}
	      	      \end{table}
	      \end{enumerate}
	      	      
	\item Dashboard Navigation Menu

    \begin{figure}[h!]
        \centering
        \includegraphics[width=0.7\linewidth]{Navigation.png}
        \caption{Navigation}
        \label{fig:enter-label}
    \end{figure}
	      \begin{table}[h!]
	      	\def\arraystretch{1.24} \small
	      	\begin{tabular}{|p{1.2cm}|p{2.5cm}|p{4.0cm}|}
	      		\hline
	      		ID  & Name                 & Description                                                                                                                                                                                                                                                                                                     \\
	      		\hline
	      		007 & Dashboard-Navigation & Navigation bar provides access to key system functions including Dashboard, Products, Customers, Orders, Analytics, Marketing, Discounts, Payouts, Statements, Calendar, and Storefront. Dashboard is the primary view, displaying essential metrics and visualizations for monitoring blockchain transactions. \\
	      		\hline
	      	\end{tabular}
	      \end{table}
\end{enumerate}

\clearpage

\section{Architecture Design \& Implementation}
\subsection{Overall Architecture}

Matter Tunnel's architecture consists of several interconnected modules designed to provide a secure, and efficient IoT management system. Each module serves a specific purpose while maintaining robust communication with other components.

\begin{figure} [h!]
	\centering
	\includegraphics[width=\linewidth]{OverallArchitecture.png}
	\caption{Overall Architecture}
	\label{fig:Overall Architecture}
\end{figure}

\begin{enumerate}[itemsep=2ex, parsep=1ex]
	\item{Core Modules}
			
	\begin{enumerate}
		\item Client Application
		      \begin{itemize}[itemsep=0.5pt, parsep=0.5pt]
		      	\item Developed using React for cross-platform compatibility
		      	\item Provides intuitive user interface for device management
		      	\item Integrates WebAssembly-compiled Matter Tunnel utilities
		      	\item Communicates with the gateway for blockchain interaction
		      \end{itemize}
		      		      		      
		\item Matter Tunnel Utility
		      \begin{itemize}[itemsep=0.5pt, parsep=0.5pt]
		      	\item Core functionality implemented in C++
		      	\item Dual deployment approach:
		      	      \begin{itemize}[itemsep=0.5pt, parsep=0.5pt]
		      	      	\item Direct C++ implementation for IoT devices
		      	      	\item WebAssembly compilation for web-based clients
		      	      \end{itemize}
		      	\item Ensures consistent behavior across different platforms
		      	\item Handles device communication and protocol management
		      \end{itemize}
		      		      		      
		\item Gateway
		      \begin{itemize}[itemsep=0.5pt, parsep=0.5pt]
		      	\item Serves as intermediary between clients/devices and blockchain
		      	\item Simplifies development process for client and device implementations
		      	\item Provides standardized API for blockchain interaction
		      	\item Manages connection pooling and request routing
		      \end{itemize}
		      		      		      
		\item Hyperledger Fabric Network
		      \begin{itemize}[itemsep=0.5pt, parsep=0.5pt]
		      	\item Implements decentralized device management
		      	\item Maintains immutable record of device interactions
		      	\item Provides high-performance transaction processing
		      	\item Supports multiple channels for scalability
		      \end{itemize}
		      		      		      
		\item Dashboard Application
		      \begin{itemize}[itemsep=0.5pt, parsep=0.5pt]
		      	\item Built with Electron for desktop performance
		      	\item Connects directly to blockchain via gRPC
		      	\item Provides comprehensive monitoring and analytics interface
		      	\item Integrates with AI component through IPC
		      \end{itemize}
		      		      		      
		\item AI Analysis Component
		      \begin{itemize}[itemsep=0.5pt, parsep=0.5pt]
		      	\item Based on fine-tuned T5 model
		      	\item Processes natural language queries
		      	\item Communicates with dashboard through simple IPC
		      	\item Generates blockchain queries from natural language
		      \end{itemize}
	\end{enumerate}
			
	\item{Key Technical Decisions}
			
	The architecture reflects several key technical decisions made to optimize system performance and usability:
			
	\begin{enumerate}
		\item Development Approach
		      \begin{itemize}[itemsep=0.5pt, parsep=0.5pt]
		      	\item React for cross-platform client development
		      	\item Electron for high-performance dashboard
		      	\item C++ for core Matter Tunnel functionality
		      	\item WebAssembly for web integration
		      \end{itemize}
		      		      		      
		\item Communication Protocols
		      \begin{itemize}[itemsep=0.5pt, parsep=0.5pt]
		      	\item gRPC for blockchain communication
		      	\item IPC for AI-Dashboard interaction
		      	\item Gateway API for client-blockchain abstraction
		      \end{itemize}
		      		      		      
		\item AI Implementation
		      \begin{itemize}[itemsep=0.5pt, parsep=0.5pt]
		      	\item T5 model fine-tuning for specific use case
		      	\item Natural language processing for query generation
		      	\item Integration with blockchain data analysis
		      \end{itemize}
	\end{enumerate}
\end{enumerate}

This architecture enables Matter Tunnel to provide a robust, scalable, and user-friendly IoT management system while maintaining high security and performance standards.
    
\subsection{Blockchain Architecture}

Matter Tunnel's blockchain infrastructure is built on Hyperledger Fabric, implementing a robust and scalable architecture designed for enterprise-grade IoT device management. The network consists of multiple components organized to ensure high availability, fault tolerance, and efficient transaction processing.

\begin{figure}[h!]
	\centering
	\includegraphics[width=1\linewidth]{BlockchainArchitecture.png}
	\caption{Blockchain Architecture}
	\label{fig:Blockchain Architecture}
\end{figure}

\begin{enumerate}[itemsep=2ex, parsep=1ex]
	\item{Network Components}
			
	\begin{enumerate}
		\item Peer Organizations
		      \begin{itemize}[itemsep=0.5pt, parsep=0.5pt]
		      	\item Two peer nodes (peer1, peer2) maintained by separate organizations
		      	\item Each peer maintains a copy of the ledger
		      	\item Responsible for endorsing transactions and maintaining state
		      	\item Hosts chaincode for device management and communication
		      \end{itemize}
		      		      		      
		\item Ordering Service
		      \begin{itemize}[itemsep=0.5pt, parsep=0.5pt]
		      	\item Four ordering nodes (orderer1-4) in Byzantine Fault Tolerance configuration
		      	\item Implements crash fault tolerance and Byzantine fault tolerance
		      	\item Ensures consensus among network participants
		      	\item Can continue operation even if f nodes fail, where f = (n-1)/3
		      \end{itemize}
		      		      		      
		\item Gateway Service
		      \begin{itemize}[itemsep=0.5pt, parsep=0.5pt]
		      	\item Provides simplified access point for end systems
		      	\item Manages connection pooling and load balancing
		      	\item Handles authentication and authorization
		      	\item Optimizes network communication
		      \end{itemize}
		      		      		      
		\item End Systems
		      \begin{itemize}[itemsep=0.5pt, parsep=0.5pt]
		      	\item Multiple user applications and devices
		      	\item Can connect either through gateway or direct gRPC
		      	\item Maintains flexibility in connection methods
		      	\item Supports various client implementations
		      \end{itemize}
	\end{enumerate}
			
	\item{Communication Patterns}
			
	\begin{enumerate}
		\item Gateway-mediated Communication
		      \begin{itemize}[itemsep=0.5pt, parsep=0.5pt]
		      	\item Simplified access through gateway API
		      	\item Reduced connection overhead
		      	\item Centralized authentication management
		      	\item Optimal for standard client applications
		      \end{itemize}
		      		      		      
		\item Direct gRPC Communication
		      \begin{itemize}[itemsep=0.5pt, parsep=0.5pt]
		      	\item High-performance direct peer connection
		      	\item Suitable for advanced applications
		      	\item Full access to Fabric SDK capabilities
		      	\item Used by dashboard and specialized clients
		      \end{itemize}
	\end{enumerate}
			
	\item{Fault Tolerance}
			
	\begin{enumerate}
		\item Byzantine Fault Tolerance
		      \begin{itemize}[itemsep=0.5pt, parsep=0.5pt]
		      	\item Four-node ordering service ensures BFT
		      	\item Tolerates up to f = 1 Byzantine failure
		      	\item Maintains consensus under malicious behavior
		      	\item Provides strong consistency guarantees
		      \end{itemize}
		      		      		      
		\item High Availability
		      \begin{itemize}[itemsep=0.5pt, parsep=0.5pt]
		      	\item Multiple peers ensure continuous operation
		      	\item Redundant ordering service nodes
		      	\item No single point of failure
		      	\item Automatic failover capabilities
		      \end{itemize}
	\end{enumerate}
\end{enumerate}

This blockchain architecture provides the foundation for secure, scalable, and reliable IoT device management while maintaining flexibility in how clients connect to and interact with the network.


\subsection{Directory Organization}

Matter Tunnel consists of six GitHub repositories: blockchain, IoT\_embedded, AI\_dashboard, matter\_tunnel\_utils, client, Documents.

\begin{enumerate}[itemsep=2ex, parsep=1ex]
	\item The blockchain repository handles blockchain-related implementation built on Hyperledger Fabric. This repository contains smart contracts and protocols designed to enable secure device-user communication between applications and IoT devices, ensuring complete end-to-end encryption(E2EE) and enhanced security.
	      	      	          
	\item The IoT\_embedded repository is dedicated to embedded systems development for IoT devices. It contains software implementations for devices compatible with the Matter Tunnel protocol and the proposed blockchain solution. This repository focuses on implementing features that enhance device functionality.
	      	      	          
	\item The AI\_dashboard repository creates and manages dashboard interfaces for monitoring and controlling IoT devices connected through the Matter Tunnel System. This repository implements both AI services and frontend visualization components to provide a comprehensive view of IoT device operations and blockchain network status. It integrates data processing capabilities with user-friendly interfaces, enabling real-time monitoring, visualization of transactions.
	      	      	          
	\item The matter\_tunnel\_utils repository provides essential utilities and tools for handling Matter tunnel protocol operations. It implements the core functionality that enables direct P2P communication between applications and IoT devices on the blockchain.
	      	      	          
	\item The client repository focuses on developing the front-end application that provides user interfaces for IoT device control. This repository contains code for implementing intuitive and responsive front-end solutions that integrate with both the Matter protocol and blockchain technology, ensuring a smooth user experience without requiring a Matter hub.
	      	      	          
	\item The Documents repository maintains comprehensive project documentation, including technical specifications, architecture designs, and implementation details of the Matter Tunnel system.
\end{enumerate}

\begin{table}[h]
	\caption{Directory Organization - blockchain}
	\def\arraystretch{1.24} \small
	\begin{tabular}{|p{5cm}|p{2.5cm}|}
		\hline
		Directory                           & File Name \\
		\hline
		blockchain/matter\_tunnel           & go.mod    \\
		/application\_gateway               & go.sum    \\
		                                    & main.go   \\
		\hline
		blockchain/matter\_tunnel/chaincode & go.mod    \\
		                                    & go.sum    \\
		                                    & main.go   \\
		\hline
	\end{tabular}
\end{table}

\begin{table}[h]
	\caption{Directory Organization - IoT embedded}
	\def\arraystretch{1.24} \small
	\begin{tabular}{|p{5cm}|p{2.5cm}|}
		\hline
		Directory                   & File Name          \\
		\hline
		IoT\_embedded/IoT\_embedded & main.cpp           \\
		                            & matter\_tunnel.cpp \\
		\hline
	\end{tabular}
\end{table}

\vspace{10cm}

\begin{table}[h]
	\caption{Directory Organization - AI\_dashboard}
	\def\arraystretch{1.24} \small
	\begin{tabular}{|p{4.5cm}|p{3cm}|}
		\hline
		Directory                         & File Name                \\
		\hline
		AI\_dashboard/ai/src/config       & init.py                  \\
		                                  & model\_config.py         \\
		\hline
		AI\_dashboard/ai/src/utils        & init.py                  \\
		                                  & data\_loader.py          \\
		                                  & filter\_data.py          \\
		                                  & preprocessing.py         \\
		                                  & query\_augmenter\_nl\_pa \\
		                                  & ug.py                    \\
		                                  & query\_augmenter.py      \\
		                                  & inference.py             \\
		                                  & train.py                 \\
		\hline
		AI\_dashboard/frontend/components & BarChartComponent.js     \\
		                                  & ChatBox.js               \\
		                                  & Sidebar.js               \\
		                                  & TPSChart.js              \\
		                                  & chartDimensions.js       \\
		                                  & MetricCards.js           \\
		\hline
		AI\_dashboard/frontend/dataset    & getTransactions.js       \\
		\hline
	\end{tabular}
\end{table}

\begin{table}[h]
	\caption{Directory Organization - matter\_tunnel\_utils}
	\def\arraystretch{1.24} \small
	\begin{tabular}{|p{4.4cm}|p{3cm}|}
		\hline
		Directory                         & File Name                \\
		\hline
		matter\_tunnel\_utils/IoT\_crypto & main.cpp                 \\
		                                  & matter\_tunnel.cpp       \\
		                                  & wasm\_matter\_tunnel.cpp \\
		\hline
		Matter\_tunnel\_utils/qr\_maker   & qr\_maker.py             \\
		\hline
	\end{tabular}
\end{table}

\vspace{10cm}

\begin{table}[h]
	\caption{Directory Organization - client}
	\def\arraystretch{1.24} \small
	\begin{tabular}{|p{5cm}|p{2.5cm}|}
		\hline
		Directory                            & File Name          \\
		\hline
		client/src/common                    & matter\_tunnel.js  \\
		\hline
		client/src/modules/auth/components   & Login.js           \\
		                                     & SignUp.js          \\
		                                     & auth.module.css    \\
		                                     & user.js            \\
		\hline
		client/src/modules/auth/hooks        & useLogin.js        \\
		                                     & useSignUp.js       \\
		\hline
		client/src/modules/auth/states       & authSlice.js       \\
		\hline
		client/src/modules/core/states       & store.js           \\
		\hline
		client/src/modules/device/components & AddDevice.js       \\
		                                     & BottomNav.js       \\
		                                     & DeviceList.js      \\
		                                     & RemoveDevice.js    \\
		                                     & UpdateDevice.js    \\
		\hline
		client/src/modules/device/hooks      & useRemoveDevice.js \\
		\hline
		client/src/modules/device/states     & deviceSlice.js     \\
		\hline
	\end{tabular}
\end{table}

\subsection{Module 1: Blockchain}

\subsubsection{Purpose}
For implementing Matter Tunnel, we utilized Hyperledger Fabric. The blockchain module serves as a crucial component in resolving network constraints in P2P communication and ensuring secure communication between Matter devices and applications. This enables IoT devices to communicate with each other without requiring a Matter hub.

\subsubsection{Functionality}
This module supports bypass of NAT constraints by acting as a virtual private network. It enables end-to-end encrypted communication between IoT devices and applications. The module provides a platform-independent communication structure and implements a decentralized authentication system for user privacy protection.

\subsubsection{Location of source code}
https://github.com/Winter-Zzzz/blockchain/matter\_tunnel

\subsubsection{Class component}
\begin{itemize}
	\item application\_gateway folder: This folder contains Go files that act as a gateway to convert the gRPC protocol to an HTTP/REST API, allowing client applications to easily communicate with the blockchain network.
	\item chaincode folder: This is a folder containing smart contract implementation files that define communication rules in Hyperledger Fabric and managing transactions.
	\item go.mod: This is a file for managing dependencies in GoLang. All modules used in GoLang are maintained in the go.mod file.
	\item go.sum: This file lists down the checksum of direct and indirect dependency requires along with the version. It is to be mentioned that the go.mod file is enough for a successful build. The checksum present in go.sum file is used to validate the checksum of each of direct and indirect dependency to confirm that none of them has been modified.
\end{itemize}

\subsection{Module 2: IoT\_embedded}

\subsubsection{Purpose}
We developed a virtual IoT device implementation. This approach allowed us to simulate MatterTunnel-compatible devices without requiring physical hardware. The software-based virtual devices implement the same communication protocols and functionalities as real IoT devices would in the Matter Tunnel ecosystem, enabling thorough testing and validation of the system's capabilities in a controlled environment.

\subsubsection{Functionality}
This module manages the IoT device's connection to the blockchain network by gateway, handles device-specific operations, and implements the Matter protocol specifications. It enables secure P2P communication and integrates with the blockchain module for data transfer.

\subsubsection{Location of source code}
https://github.com/Winter-Zzzz/IoT\_embedded/IoT\_embedded

\subsubsection{Class component}
\begin{itemize}
	\item main.cpp: This is the main entry point file containing device initialization and core control logic.
	\item matter\_tunnel.cpp: This is a file implementing Matter Tunnel protocol specifications.
\end{itemize}

\subsection{Module 3: AI\_dashboard}

\subsubsection{Purpose}
For visualizing and analyzing blockchain metrics and transaction data, we developed a dashboard using Electron. The AI dashboard module provides an intuitive interface for monitoring system performance, displaying statistical data, and managing blockchain transactions. This module helps understanding the system's behavior and performance through various visual components.

\subsubsection{Functionality}
This module visualizes transaction statistics, provides real-time monitoring capabilities, and offers interactive data visualization components. It includes features for displaying blockchain transactions, and key metrics through various chart types.

\subsubsection{Location of source code}
https://github.com/Winter-Zzzz/AI\_dashboard

\subsubsection{Class component}
\begin{itemize}
	\item ai folder: A folder containing AI implementations and models for data analysis.
	      \begin{itemize}
	      	\item config folder: Configuration files for AI models.
	      	      \begin{itemize}
	      	      	\item init.py: Initialization file for the config module.
	      	      	\item model\_config.py: Configuration file for AI model parameters.
	      	      \end{itemize}
	      	\item utils folder: Utility functions for data processing.
	      	      \begin{itemize}
	      	      	\item data\_loader.py: Loading and managing data.
	      	      	\item filter\_data.py: Data filtering operations.
	      	      	\item preprocessing.py: Data preprocessing functions.
	      	      	\item query\_augmenter\_nlpaug.py: NLP-based query augmentation.
	      	      	\item query\_augmenter.py: General query augmentation.
	      	      	\item inference.py: Model inference operations.
	      	      	\item train.py: Model training functions.
	      	      \end{itemize}
	      \end{itemize}
	\item components folder: Reusable UI components for a dashboard.
	      \begin{itemize}
	      	\item BarChartComponent.js: Bar chart visualizations.
	      	\item ChatBox.js: Communication interface component.
	      	\item Sidebar.js: Sidebar component.
	      	\item TPSChart.js: TPS Chart visualizations.
	      	\item chartDimensions.js: Utility file for chart sizing.
	      	\item MetricCards.js: Displaying total transactions and users in card style.
	      \end{itemize}
	\item dataset folder: File for handling blockchain transaction data processing and analytics.
	      \begin{itemize}
	      	\item getTransactions.js: Fetches and processes blockchain transaction data.
	      \end{itemize}
\end{itemize}

\subsection{Module 4: matter\_tunnel\_utils}

\subsubsection{Purpose}
For supporting Matter Tunnel's core functionality, we developed utility modules. It provides essential tools and utilities, including IoT device cryptography handling and QR code generation capabilities. This module serves as a supporting layer for the main Matter Tunnel implementation.

\subsubsection{Functionality}
This module handles cryptographic operations for IoT devices, manages WebAssembly implementations of Matter Tunnel, and provides tools for generating QR codes used in device pairing and configuration processes.

\subsubsection{Location of source code}
https://github.com/Winter-Zzzz/matter\_tunnel\_utils

\subsubsection{Class component}
\begin{itemize}
	\item IoT\_crypto folder: This is a folder containing cryptographic \& Matter tunnel implementations.
	      \begin{itemize}
	      	\item main.cpp: This is the main entry point file that contains many tests.
	      	\item matter\_tunnel.cpp: This is a file implementing Matter Tunnel cryptographic functions.
	      	\item wasm\_matter\_tunnel.cpp: This is a WebAssembly implementation of Matter Tunnel functions.
	      \end{itemize}
	\item qr\_maker folder: This is a folder for QR code generation utilities.
	      \begin{itemize}
	      	\item qr\_maker.py: This is a file for generating QR codes for device configuration.
	      \end{itemize}
\end{itemize}

\subsection{Module 5: client}

\subsubsection{Purpose}
The client module serves as the frontend interface for Matter Tunnel, providing user authentication, device management, and interaction capabilities. It enables users to control their IoT devices through a web-based interface without requiring a traditional Matter hub, leveraging the blockchain infrastructure for secure communication.

\subsubsection{Functionality}
This module implements user authentication and registration, device management (adding, removing, and updating devices), and provides a seamless interface for controlling Matter-compatible devices. It utilizes React hooks for state management and implements end-to-end encryption for secure communication with the blockchain network.

\subsubsection{Location of source code}
https://github.com/Winter-Zzzz/client/src

\subsubsection{Class component}
\begin{itemize}
	\item common folder: This is a folder containing shared utilities and components used across the application.
	      \begin{itemize}
	      	\item matter\_tunnel.js: This is a file for core functionality implementation for Matter protocol integration.
	      \end{itemize}
	\item modules folder: This is a folder containing the core functional units of the application.
	      \begin{itemize}
	      	\item auth folder: This is a folder containing all authentication-related components, hooks, and state management.
	      	      \begin{itemize}
	      	      	\item components folder: This is a folder containing React components related to authentication features.
	      	      	      \begin{itemize}
	      	      	      	\item Login.js: This is a file that renders the login form and handles user authentication logic.
	      	      	      	\item SignUp.js: This is a file implementing user registration interface and validation.
	      	      	      	\item auth.module.css: This is a CSS file containing styles specific to authentication components.
	      	      	      	\item user.js: This is a file that manages user authentication state.
	      	      	      \end{itemize}
	      	      	\item hooks folder: This is a folder containing files for authentication-related functionality.
	      	      	      \begin{itemize}
	      	      	      	\item useLogin.js: This is a file that manages login state and provides login-related functions.
	      	      	      	\item useSignUp.js: This is a file that handles registration logic.
	      	      	      \end{itemize}
	      	      	\item states folder: This is a folder containing files for managing authentication-related state using Redux.
	      	      	      \begin{itemize}
	      	      	      	\item authSlice.js: This is a file that manages authentication state.
	      	      	      \end{itemize}
	      	      \end{itemize}
	      	\item device folder: This is a folder managing all device-related functionality and user interface components.
	      	      \begin{itemize}
	      	      	\item components folder: This is a folder containing components for device management.
	      	      	      \begin{itemize}
	      	      	      	\item AddDevice.js: This is a file that provides interface for adding new devices.
	      	      	      	\item BottomNav.js: This is a navigation component file.
	      	      	      	\item DeviceList.js: This is a file displaying the list of connected devices.
	      	      	      	\item RemoveDevice.js: This is a file handling device removal functionality.
	      	      	      	\item UpdateDevice.js: This is a file managing device update operations.
	      	      	      \end{itemize}
	      	      	\item hooks folder: This is a folder containing custom hooks for device operations.
	      	      	      \begin{itemize}
	      	      	      	\item useRemoveDevice.js: This is a file managing device removal logic.
	      	      	      \end{itemize}
	      	      	\item states folder: This is a folder managing device-related state using Redux.
	      	      	      \begin{itemize}
	      	      	      	\item deviceSlice.js: This is a file managing the state of devices.
	      	      	      \end{itemize}
	      	      \end{itemize}
	      	\item core folder: This is a folder containing core application state management.
	      	      \begin{itemize}
	      	      	\item states folder: This is a folder managing global application state.
	      	      	      \begin{itemize}
	      	      	      	\item store.js: This is a file for the Redux store.
	      	      	      \end{itemize}
	      	      \end{itemize}
	      \end{itemize}
\end{itemize}

\section{Use Cases}
\subsection{Use case 1: Client}

\begin{enumerate}[itemsep=2ex, parsep=1ex]
	\item{Initial Experience}

\begin{figure}[h!]
    \centering
    \includegraphics[width=0.5\linewidth]{initialExperience.png}
    \caption{initial experience 1}
    \label{fig:enter-label}
\end{figure}

\begin{figure}[h!]
    \centering
    \includegraphics[width=1\linewidth]{initialExperience2.png}
    \caption{initial experience 2}
    \label{fig:enter-label}
\end{figure}
            
	When users first launch the application, they are greeted with the Matter Tunnel logo on the entry screen while the system loads. Following this, new users are presented with a comprehensive tutorial that introduces core functionalities through multiple slides. Users can navigate through the tutorial using the ``Next" button or choose to skip it entirely. The tutorial concludes by directing users to the login page to begin using the application.
	        
	\item{Login}

    \begin{figure}[h!]
        \centering
        \includegraphics[width=0.9\linewidth]{login.png}
        \caption{login}
        \label{fig:enter-label}
    \end{figure}
	        
	The Login page enables users to access the system by entering their private key for authentication. Users must input a 64-character hexadecimal key (using characters 0-9 and A-F) for validation. If the key format is incorrect, the system displays an error message ``Your key is incorrectly formatted". This secure authentication method ensures that only authorized users can access their device management interface.

    \clearpage
	        
	\item{Sign Up}
	\begin{enumerate}
		\item Sign Up button

        \begin{figure}[h!]
            \centering
            \includegraphics[width=0.9\linewidth]{signUpButton.png}
            \caption{SignUp Button}
            \label{fig:enter-label}
        \end{figure}
        
        The user needs its own private key to use program. They can create new private key by clicking ``Sign up" button on the login page. The system then generates and displays a new private key.
		                  
		\item Copy button

        Users can copy using the ``Copy" button. After creating their own key, they can proceed to the login screen and attempt to log in with the newly created private key.

        \vspace{5cm}

        \begin{figure}[h!]
            \centering
            \includegraphics[width=0.9\linewidth]{copyButton.png}
            \caption{Copy Button}
            \label{fig:enter-label}
        \end{figure}
		                  
		\item A ``Sign Up Completed" message confirms successful registration.

        \begin{figure}[h!]
            \centering
            \includegraphics[width=0.7\linewidth]{SignUpCompletedMessage.png}
            \caption{SignUp Completed Message}
            \label{fig:enter-label}
        \end{figure}
	\end{enumerate}

    \clearpage
	        
	\item{Main Page}
	\begin{enumerate}
		\item QR Code recognition

        \begin{figure}[h!]
            \centering
            \includegraphics[width=0.9\linewidth]{QRCodeRecognition.png}
            \caption{QR Code Recognition}
            \label{fig:enter-label}
        \end{figure}
        
        On the main page, users can add new devices to their system by clicking `+' button. After clicking the button, users enter the scanning interface where they can capture the device's QR code. Upon successful scanning, the screen displays ``QR Code Scan Complete!" A ``Rescan" option is available if needed. If the scanned device is already registered, the system shows ``Already Registered Device!" warning.
		                  
		\item Device Addition

        \begin{figure}[h!]
            \centering
            \includegraphics[width=1\linewidth]{DeviceAddition.png}
            \caption{Device Addition}
            \label{fig:enter-label}
        \end{figure}
        
        After QR scan, users can input a custom name for the device and the ``Register" button becomes active. The system provides immediate feedback through notifications, either confirming successful registration with ``Device Registration Complete" or warning if the device is already registered. Users can cancel the addition process at any time using the X button.
		                  
		\item Device List Management

        \begin{figure}[h!]
            \centering
            \includegraphics[width=0.9\linewidth]{DeviceList.png}
            \caption{Device List Management}
            \label{fig:enter-label}
        \end{figure}
        
        The main screen displays all registered devices in a square card format, with each card showing the device icon, name, and public key obtained from the QR code. A three-dot menu button : on each card provides access to device controls. When accessed, users can view device feedback from the server and execute available device. The function section includes input fields for required arguments and displays ``Processing Transaction..." during execution. Users can easily monitor and manage their connected devices from this central interface.

        \clearpage
		                  
		\item Device Deletion
        
        \begin{figure}[h!]
            \centering
            \includegraphics[width=0.9\linewidth]{DeviceDeletion.png}
            \caption{Device Deletion}
            \label{fig:enter-label}
        \end{figure}
        
        Users can initiate device removal by clicking the `-' button, which activates deletion mode. This displays checkboxes on each device card, allowing users to select multiple devices for simultaneous deletion. The interface shows ``Delete" and ``Cancel" buttons during deletion mode. The deletion process requires at least one device selection and confirms the removal through a notification message.
	\end{enumerate}
	        
	\item{Navigation}

    \begin{figure}[h!]
        \centering
        \includegraphics[width=0.9\linewidth]{Navigation1.png}
        \caption{Navigation}
        \label{fig:enter-label}
    \end{figure}
	        
	The application features a bottom navigation bar containing Home and Account buttons. The Home button returns users to the Device List screen. The Account Button directs account management page. This navigation bar remains visible throughout the application except during login and signup processes.
	        
	\item{Account Management}

    \begin{figure}[h!]
        \centering
        \includegraphics[width=0.9\linewidth]{AccountManagement1.png}
        \caption{Account Management 1}
        \label{fig:enter-label}
    \end{figure}

    \begin{figure}[h!]
        \centering
        \includegraphics[width=0.9\linewidth]{AccountManagement2.png}
        \caption{Account Management 2}
        \label{fig:enter-label}
    \end{figure}
	        
	When users click the Account button, they are directed to the account management page where they can view their public key information and manage their private key settings. Users can copy their private key using the ``Copy Private Key" button. It displays a ``Private key copied" alert upon successful copying. The page also includes a ``Logout" button that, when clicked, logs the user out and redirects them to the login page with a ``Logout Completed" confirmation message.
\end{enumerate}

\subsection{Use case 2: Enterprise}
    
\begin{figure}[h!]
	\centering
	\includegraphics[width=1\linewidth]{DashboardScreen.png}
	\caption{Dashboard Full Screen}
	\label{fig:Dashboard_full}
\end{figure}
    
The Matter Tunnel Dashboard is a comprehensive monitoring system designed to provide real-time visibility into transaction processing and system performance. The primary users of this dashboard are system administrators and performance analysts who need to track and analyze transaction throughput across different time periods and processing nodes.
\begin{enumerate}[itemsep=2ex, parsep=1ex]
	\item TPS chart
	      	      	      
	      \begin{figure}[h!]
	      	\centering
	      	\includegraphics[width=1\linewidth]{DashboardTransactionsPerSec.png}
	      	\caption{TPS Chart}
	      	\label{fig:TPSChart}
	      \end{figure}
	      	      	      
	      CopyThis is a real-time Transactions Per Second (TPS) line graph that refresh automatically to show the most recent 18-hour-period. The system plots actual transaction throughput as it occurs, allowing operators to observe performance trends and respond to any anomalies immediately.
	      	      	      
	\item PK transactions chart
	      	      	      
	      \begin{figure}[h!]
	      	\centering
	      	\includegraphics[width=1\linewidth]{DashboardTransactionsByPK.png}
	      	\caption{PK Transactions Chart}
	      	\label{fig:PKTransactionsChart}
	      \end{figure}
	      	      	      
	      This horizontal bar chart displays transaction distribution across distinct PKs (Primary Keys). This visualization updates in real-time to reflect the current processing load across different nodes. The bars will dynamically adjust as transaction volumes shift between PKs, enabling immediate detection of load balancing issues or processing bottlenecks.
	      	      	      
	\item Total Transaction\&User Card
	      	      	      
	      \begin{figure}[h!]
	      	\centering
	      	\includegraphics[width=1\linewidth]{Dashboard-Total.png}
	      	\caption{Total Transaction \& User Card}
	      	\label{fig:TotalTransactionUserCard}
	      \end{figure}
	      	      	      
	      This part of dashboard displays key transaction statistics through metric cards. The first card shows total transactions, and the second card shows total users. These cards provide administrators and operators with immediate visibility into the system's current usage and activity levels, enabling quick assessment of blockchain network participation and transaction volume.

        \clearpage

        \item Natural Language Query
	      	      	      
	      \begin{figure}[h!]
	          \centering
	          \includegraphics[width=1\linewidth]{NaturalLanguageQuery.png}
	          \caption{Natural Language Query}
	          \label{fig:enter-label}
	      \end{figure}
	      	      	      
	      This chat-style interface transforms blockchain transaction data into an easily readable format, presenting information in familiar message bubbles. Each transaction displays key details including timestamp, public keys, and function type, allowing users to track and monitor blockchain activities as naturally as reading a conversation. It provides clear transaction identification headers, making it simple for users to follow specific transaction threads and understand the flow of blockchain activities.
          
\end{enumerate}

\section{Discussion}
    
\subsection{Technical Difficulties}

During the development of Matter Tunnel, we encountered several technical hurdles that needed to be addressed. In our initial approach, we attempted to build WebAssembly from C++ code. While we successfully built simple C++ code to WebAssembly, we faced significant challenges when trying to link OpenSSL. Although we managed to build the required libraries as WebAssembly static libraries, the linking process produced various errors. Despite consulting numerous resources including Stack Overflow, we couldn't find a working solution. Due to time constraints, we ultimately decided to abandon the WebAssembly approach and instead implement native code for each platform (C++ and JavaScript).

Working with byte and bit operations across different platforms proved challenging, particularly in ensuring JavaScript code matched C++ functionality exactly. We encountered numerous encoding issues, which we resolved by either converting problematic sections to base16 within functions or utilizing Latin-1 encoding where appropriate.

The AI component also presented technical challenges, particularly due to computational resource limitations. Using Google Colab's GPU restrictions meant we had limited processing power and time for model training. This constraint affected our ability to implement extensive data augmentation techniques and train our models on diverse patterns. While we developed a synthetic data generation pipeline for natural language query processing, the GPU limitations prevented us from exploring more sophisticated augmentation methods that could have improved our model's robustness. Despite these constraints, we implemented an iterative training approach and efficient caching mechanisms to optimize our model's performance within the available resources.

\subsection{Non-technical Difficulties}

The most significant non-technical challenge was coordinating between different team members working on various components of the system. Each developers had different perspectives and priorities, which sometimes led to integration challenges and communication barriers. To address these challenges, we implemented regular meetings, and the increased frequency of communication proved highly effective, as it allowed team members to better understand each other’s constraints and requirements.

Through consistent effort to maintain open communication, we were able to transform our initial challenges into opportunities for team growth and project improvement. The solutions we implemented not only resolved our immediate challenges but also established a strong foundation for future development. As a result, we successfully delivered Matter Tunnel with improved team dynamics and a more cohesive development process than when we started.


\subsection{Conclusion}

Matter Tunnel successfully demonstrated the feasibility of enabling communication between IoT devices and users without requiring additional hardware like Matter hubs. This breakthrough not only freed devices from spatial constraints but also expanded the range of supported functionalities beyond Matter's predefined application cluster specification. The implementation significantly enhanced security and privacy through blockchain-based end-to-end encryption and decentralization. From an enterprise perspective, the system substantially improved reliability by providing immutable transaction records and comprehensive data analytics capabilities, enabling more informed decision-making through transparent and trustworthy operational insights.

Looking ahead, Matter Tunnel shows great promise in revolutionizing the IoT industry by providing a more open, secure, and flexible device management solution. Future development will focus on successfully implementing WebAssembly builds to enhance cross-platform compatibility and expanding platform support to accommodate a wider range of devices and use cases. Additionally, we aim to improve the dashboard's AI capabilities by incorporating advanced data augmentation techniques such as back translation, which will enhance the natural language processing system's performance and versatility in handling various query patterns.

\end{document}